\section{Ejercicio 3}
\subsection{Interpretación del enunciado}
\par{En cierta provincia se ubican varias fábricas de ladrillos pertenecientes
  una misma empresa. La empresa se encarga de distribuír  lo s ld rillos de l
as fábricas a sus clientes. Para ello, utiliza las rtas provinciales que conec                                                                                                                                                                                                                                                                                         
distintos puntos de la provincia en los que puede haber una fábrica o un
cliente. Sin embargo, las rutas no están preparadas para soportar el peso de
los camiones que transportan los ladrillos, así que estas deben ser
fortalecidas. El costo de fortalecer cada ruta es proporcional a la longitud de
la misma. La empresa quiere fortalecer determinadas rutas para poder distribuír
los ladrillos a sus clientes sin problemas, intentando reducir lo más posible
el costo de dicha remodelación. El objetivo del ejercicio es desarrollar un
algoritmo que determine el conjunto de rutas que deben ser fortalecidas para
poder continuar la distribución (debe haber una ruta fortalecida entre cada
cliente y al menos una fábrica), cuyo costo de fortalecimiento sea mínimo.}
\subsection{Resolución}
\par{Se modelará la situación a través de un grafo en el que cada nodo
representará un cliente o una fábrica y cada arista entre un par $v1$, $v2$
de nodos representará una ruta entre los clientes o fábricas correspondientes a
$v1$ y $v2$. Para modelar el costo de fortalecer cada ruta, se le asignará a
cada arista el peso equivalente a la longitud de su correspondiente ruta. Si
bien el grafo podría no ser conexo, Se asume que existe un camino simple entre
cada cliente y al menos una fábrica, ya que se parte de que se puede satisfacer
la demanda de todos los clientes. Se busca obtener un conjunto de rutas tal que
exista un camino entre cada cliente y al menos una fábrica, es decir que
debemos encontrar un subgrafo en el grafo modelado. Como se busca reducir el
costo total sería conveniente que dicho subgrafo esté compuesto por árboles.
Este subgrafo constaría de una componente conexa por cada fábrica, sin dejar
clientes afuera. Cada componente conexa tendría que ser un árbol generador
del subgrafo inducido por los nodos de dicha componente. De esta forma se está
asegurando mantener la mínima cantidad de rutas necesarias. De todos los
posibles subgrafos bosque, debemos tomar el de menor peso para reducir el costo
total de fortalecer las rutas. En definitiva hay que encontrar un bosque
generador (conjunto de árboles generadores) de peso mínimo en el grafo modelado,
con la restricción de que haya exactamente una fábrica en cada árbol y ningún
cliente quede afuera.

A continuación, se presenta el pseudocódigo del algoritmo:\\

\begin{algorithm}[H]
	\caption{Algoritmo de Ejercicio 3}
	\KwData{Grafo G=(V,E) con nodos clientes y nodos fábrica, tal que para todo cliente, hay una fábrica alcanzable desde el.}
	
	$G.aristas \longleftarrow$ ordenar$\_$por$\_$peso$\_$asc($G.aristas$)\\
	\textbf{vector<arista>} $res \longleftarrow \emptyset$\\
	\ForEach{$e \in G.aristas$}{
		\If{!forma$\_$ciclo($e$, $res$) $\land$ no une dos componentes con una fábrica en cada una}{
		$res \longleftarrow res \cup e$\\
		}
	}
	\textbf{return} $res$\\
	
\end{algorithm}

}

\subsection{Demostración de Correctitud}


\textbf{La solución es válida:\\}
    Demostración:
    Todo cliente puede ser provisto por al menos una fábrica ya que hay una
    fábrica por componente conexa.
    Veremos que el algoritmo retorna un grafo $G $con cada cliente que pertenece a una componente
    conexa(en la que hay una fábrica):\\
    Supongamos que hay un cliente $c$ que no está en ninguna componente conexa. Esto es equivalente a afirmar
    que $c \notin V(G)$. Como cada cliente en el grafo de entrada del algoritmo, tiene un camino a alguna fábrica
    $\Longrightarrow$ existe un camino entre cualquier cliente y alguna fábrica.}

\textbf{Demostrar que es óptimo}

    \par{
%   \textbf{Demostrar que existe una solución óptima con una fábrica por componente conexa:}
    
    Notar que la solución debe cumplir que en cada componente conexa hay al menos una fábrica. Veremos que si un grafo $G$ contiene
    más de una fábrica en una componente conexa, existe un subgrafo (con el mismo conjunto de vertices y menos aristas)
    de $G'$ con una sola fábrica por componente conexa con menor peso que el de $G$.
    
    \textbf{Propiedad:} Sea una $G$ el grafo de una solución con más
    de una fábrica en alguna componente conexa, veremos que si hay una sola fábrica en esa componente, el costo de la nueva solución
    quitandole una arista \footnote{Esta arista que quitamos pertence a un camino entre dos fábricas}tiene igual o menor peso.\\
    
    Demostración:
    Sea $P$ un camino entre dos fábricas y sea $G'$ el grafo resultante de quitarle una arista $e$ (que une a los nodos $u,v$)
    perteneciente a un camino entre dos fábricas:$ P = C_1 \leftrightarrow (u,v) \leftrightarrow C_2$.
    Veamos que $G'$ sigue siendo solución posible.
    
    Sea un nodo cliente que no está en la componente conexa a la que pertenece $e$. Luego, ese cliente tiene a una fábrica alcanzable,
    ya que no quitamos ninguna arista de su componente conexa y como $G$ era solución, había una fábrica en esa componente.
    Sea un nodo cliente $c$ que estaba en la componente conexa a la que pertenece $e$.
    En ese caso como $G$ era un grafo solución, existia en $G$ un camino entre un c y una fábrica. Luego en ese camino, pasaba por
    $u$, o por $v$ (los nodos incidentes a la arista removida).Sin perdida de generalidad, supongamos que inicidía en v.
    
    Como $e$ era una arista que pertenecía a un camino entre dos fábricas se puede crear un camino
    que llegue desde $c$ a una fábrica: $ c \leftrightarrow v \leftrightarrow C_2$. Luego $G'$ tambien es solución.
    Además como quitamos una arista, que tiene peso positivo (todas lo tienen), el costo
    de la solución es menor o igual a la solución anterior.
    
    Esta quita de aristas se puede repetir hasta tener un grafo con una fábrica por componente conexa. Este grafo
    sera solución optima pues no se le pueden quitar aristas: si se remueve una arista de una componente conexa (que es un arbol) 
    la cual contiene una sola fábrica , se va a desconectar la componente generando asi la imposibilidad
    para algun cliente de alcanzar a la fábrica que estaba en la componente conexa (antes de quitar la arista).\\}

% 	\par{\textbf{Demostrar que usa la cantidad mínima de aristas:}
% 	Sea T una solución con una fábrica por componente conexa(cc) y c aristas (la cantidad de cliente).
% 	Como cada cc es una AGM, al quitar una arista de una cc, pasan a haber 2 ccs, y como sólo había una fábrica por cc,
% 	los clientes de alguna de las 2 ccs formadas no podrá ser provisto.
% 	$\Rightarrow$ La cantidad de aristas de la solución es mínima.}

  \par{
  \textbf{El peso de la solución es mínimo:\\} 
  Demostración:\\
  Sea $T$ la solución devuelta por nuestro algoritmo y sea $e$ una arista del grafo original tal que $e \notin T$.
  Agrego la arista $e = (u,v)$ a $T$ generando así al grafo $T'$:

  \textbf{Caso 1, $u$ y $v$ pertenecen a la misma cc en T: } 
  Cada cc del grafo devuelto por nuestro algoritmoes un AGM y, por ser árbol,
  $T+e$ contiene un ciclo incluído en la cc de $u$ y $v$. Sea $C$ tal ciclo, tomo $f$
  arista en $C$. Defino $T' = T+e-f$ con peso $p(T') = p(T) + p(e) - p(f)$.
  Como nuestro algoritmo siempre selecciona las aristas de peso minimo que no forman ciclo
  $$p(f) \leq p(e)$$ , luego
  $$0 \leq p(e) - p(f) \Rightarrow p(T') \geq P(T)$$. Es preciso aclarar que $T+e-f$ es conexo, por lo tanto todos los clientes
  tienen a una fábrica alcanzable.

  \textbf{Caso 2, $u$ y $v$ pertenecen a distintas cc en T: }
  Las ccs de $u$ y $v$ ahora son una misma cc con dos fábricas. Esta cc es un árbol porque es conexa (obvio) y la cantidad de
  aristas es la suma de la cantidad de aristas es $m = m1+m2+1 = n1-1 + n2-1 +1 = n1+n2-1 = n-1$.
  Luego puedo quitar una arista que pertenece al camino entre las dos fábricas \footnote{Es preciso aclarar que puedo suponer esto de la arista
  que quitamos ya que nuestro algoritmo no devuelve grafos con componentes conexas con más de una fábrica.}
  que actualmente están conectadas con el  camino
  $ P = C_1 \leftrightarrow (u,v) \leftrightarrow C_2$. Luego cualquier cliente del grafo resultante de remover la arista $e$
  tiene un camino hasta alguna fábrica:
  
  Sea un nodo cliente que no está en la componente conexa a la que pertenece $e$. Luego, ese cliente tiene a una fábrica alcanzable,
  ya que no quitamos ninguna arista de su componente conexa y como $T$ era solución, había una fábrica en esa componente.
  
  Sea un nodo cliente $c$ que estaba en la componente conexa a la que pertenece $e$.
  En ese caso como $T$ era un grafo solución, existia en $T$ un camino entre un c y una fábrica. Luego en ese camino, pasaba por
  $u$, o por $v$ (los nodos incidentes a la arista removida).Sin perdida de generalidad, supongamos que inicidía en v.
    
  Como $e$ era una arista que pertenecía a un camino entre dos fábricas se puede crear un camino
  que llegue desde $c$ a una fábrica: $ c \leftrightarrow v \leftrightarrow C_2$. Luego $T'$ tambien es solución.
    
  Como nuestro algoritmo siempre selecciona las aristas de peso minimo que no forman ciclo
  $$p(f) \leq p(e)$$ , luego
  $$0 \leq p(e) - p(f) \Rightarrow p(T') \geq P(T)$$.
    
% 		Como la cantidad de aristas
% 		de la solución actual es una más que la mínima puedo sacar una arista de la nueva cc para obtener una solución mejor que
% 		T+e, siempre y cuando mantenga que las dos ccs que se formen al quitar una arista de la nueva cc, tengan una fábrica.
% 		Sea f una arista distinta de e tal que al quitarla de la nueva cc, se formen dos ccs con una fábrica cada una (Si no hubiese
% 		una f distinta de e, tendría que sacar e y volvería a quedar la misma solución, por lo tanto sería solución única).
% 		el peso de f no puede ser mayor al peso de e porque si lo fuera, en lugar de escoger f, el algoritmo habría escogido e.\\
}


    $\Rightarrow$ Cualquier solución del tipo $T+e-f$ tiene peso mayor o igual al peso de $T \Rightarrow$ Nuestra solución es óptima. \box

}

\end{enumerate}

%\subsection{Complejidad}
%La complejidad del algoritmo es $O(m+n)$, ya que se fija en todas las aristas

\subsection{Cota de Complejidad}

\subsection{Implementación}

\begin{lstlisting}
grafo kruskal(){
  disjointSet ds(this->cantNodos);
  vector<arista> res;
  /*Ordeno la lista de aristas para Kruskal */ // ----> O(n logn)
  sort(this->aristas.begin(), this->aristas.end()); 
  
  int i = 0;	//indice de arista que itero
  int j = 0;	//Cantidad de aristas que agregue
  int cantClientes = cantNodos - this->cantFabricas;
  
  while (j < cantClientes){
    int root1 = ds.root(this->aristas[i].nodo1)->valor;
    int root2 = ds.root(this->aristas[i].nodo2)-> valor;
    if((root1 != root2) 	//estan en distintas componente conexas y...
    // no pasa que ambos tengan una fabrica
    && !((root1 < this->cantFabricas) && (root2 < this->cantFabricas)) ){ 
      if (root1 < this->cantFabricas) {  //Si el primero es una fabrica...
	      ds.simple_join(root2, root1);      
      } else {   //Si el segundo es fabrica o si ninguno lo es.
	      ds.simple_join(root1, root2);	     
      }
      j++;
      res.push_back(this->aristas[i]);
    }
    i++;
  }
  
  /* Armo el grafo para devolver */
  grafo g_res(this->cantNodos);
  for(int i = 0; i < res.size(); ++i)
	  g_res.asociar(res[i].nodo1+1, res[i].nodo2+1, res[i].peso);
  
  return g_res;
}
\end{lstlisting}
\newpage


\subsection{Testing de Correctitud}

Los tests expuestos a continuación fueron diseñados con el fin de verificar
diferentes casos particulares que pudimos identificar. Para cada test vamos
a exponer la entrada, la salida y, en caso de que sea necesario, una
justificaci\'on de la correctitud de la soluci\'on.\\

\noindent\textbf{Test$\#$1}\\
\textbf{Caracterización:} Una fábrica sin clientes.\\
\textbf{Input:}\\ \texttt{1 0 0}
\begin{center}
\begin{dot2tex}
graph graphname{
	rank=same;
	rankdir="LR";
	splines=line;
	1;
}
\end{dot2tex}
\end{center}
\textbf{Output:} \texttt{0 0}\\
\textbf{Status:} Ok, como no hay clientes, no hay rutas que fortalecer.\\

\noindent\textbf{Test$\#$2}\\
\textbf{Caracterización:} Varias fábricas sin clientes.\\
\textbf{Input:}\\ \texttt{6 0 8\\1 2 5\\1 3 7\\2 3 3\\2 4 6\\3 4 8\\
4 5 8\\4 6 4\\6 1 2}
\begin{center}
\begin{dot2tex}
graph graphname{
	rank=same;
	rankdir="LR";
	splines=line;
	1 -- 2 [label=5];
	1 -- 3 [label=7];
	3 -- 2 [label=3];
	4 -- 2 [label=6];
	4 -- 3 [label=8];
	5 -- 4 [label=8];
	4 -- 6 [label=4];
	1 -- 6 [label=2];
}
\end{dot2tex}
\end{center}
\textbf{Output:} \texttt{0 0}\\
\textbf{Status:} Ok, como no hay clientes, no hay rutas que fortalecer.\\



\noindent\textbf{Test$\#$3}\\
\textbf{Caracterización:} Varias fábricas y un cliente.\\
\textbf{Input:}\\ \texttt{4 1 8\\1 2 5\\1 3 7\\2 3 3\\2 4 6\\3 4 8\\
4 5 8\\4 1 4\\5 2 2}
\begin{center}
\begin{dot2tex}
graph graphname{
	rank=same;
	rankdir="LR";
	splines=line;
	1 -- 2 [label=5];
	1 -- 3 [label=7];
	3 -- 2 [label=3];
	4 -- 2 [label=6];
	4 -- 3 [label=8];
	5 -- 4 [label=8];
	4 -- 1 [label=4];
	2 -- 5 [label=2];
}
\end{dot2tex}
\end{center}
\textbf{Output:} \texttt{ 2 1 5 2}\\
\textbf{Status:} Ok, como hay un solo cliente (el nodo 5), elige la ruta de costo minimo que conecta
 a ese cliente con una de las fábricas(el nodo 2) adyacentes.\\

\noindent\textbf{Test$\#$4}\\
\textbf{Caracterización:} Varios clientes y una fábrica.\\
\textbf{Input:}\\ \texttt{1 4 8\\1 2 5\\1 3 7\\2 3 3\\2 4 6\\3 4 8\\
4 5 8\\4 1 4\\5 2 2}
\begin{center}
\begin{dot2tex}
graph graphname{
	rank=same;
	rankdir="LR";
	splines=line;
	1 -- 2 [label=5];
	1 -- 3 [label=7];
	3 -- 2 [label=3];
	4 -- 2 [label=6];
	4 -- 3 [label=8];
	5 -- 4 [label=8];
	4 -- 1 [label=4];
	2 -- 5 [label=2];
}
\end{dot2tex}
\end{center}
\textbf{Output:} \texttt{14 4 5 2 2 3 4 1 1 2 }\\
\textbf{Status:} Ok, como hay una sola fábrica debe conectar a todo el grafo(fábrica y clientes) con el menor costo posible.
Es decir debe hacer el AGM del grafo. Notar que lo hace ya que devuelve lo que haria Kruskal, siempre eligiendo los nodos de menor peso
que no generan circuitos.\\

\newpage
\subsection{Testing de Performance}

Para realizar el test, generamos grafos aleatorios, con pesos aleatorios (entre 0 y 100). Decimos que los grafos son aleatorios, porque nuestro algoritmo va agrandando alguna componente conexa uniendo vertices
correspondientes a los clientes con componenetes conexas que contienen una fabrica. Cabe destacar que esas componenetes conexas pueden ser simplemente una fábrica.

Realizamos un gráfico comparando la sucesión de tiempos obtenida con una función cuadrática.
La función $f = C * n^2$, donde C es una constante. Se midió el tiempo con n desde 1 hasta 1000 con saltos de a 10 (con 10 mediciones por cada n).\\

\begin{figure}[H]
\centering
\includegraphics{imgs/logo_dc.jpg}
\caption{Test Performance: Tiempo(s) vs Cantidad de Clientes y Fábricas.}
\end{figure}
