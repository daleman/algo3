\nonstopmode
\documentclass[10pt,a4paper]{article}
\usepackage[utf8]{inputenc} % para poder usar tildes en archivos UTF-8
\usepackage[spanish]{babel} % para que comandos como \today den el resultado en castellano
\usepackage{a4wide} % márgenes un poco más anchos que lo usual
\usepackage{color}
\usepackage{gnuplottex}
%\usepackage{ccfonts,eulervm}
\usepackage{dot2texi}
\usepackage{tikz}
\usetikzlibrary{shapes,arrows}
\usepackage[T1]{fontenc}
\usepackage{listings}
\usepackage{xcolor}
\usepackage{amsmath}
\lstset { %
    language=C++,
    %backgroundcolor=\color{black!5}, % set backgroundcolor
                   basicstyle=\ttfamily,
                keywordstyle=\color{blue}\ttfamily,
                stringstyle=\color{red}\ttfamily,
                commentstyle=\color{green}\ttfamily,
                morecomment=[l][\color{magenta}]{\#}
}
\usepackage{float}
\usepackage{fancyhdr}
\pagestyle{fancy}
\thispagestyle{fancy}
\addtolength{\headheight}{1pt}
\lhead{AED3}
\rhead{TP2}
\usepackage[ruled,vlined,linesnumbered]{algorithm2e}
\usepackage[conEntregas]{caratula}
\renewcommand*{\algorithmcfname}{Algoritmo}

\begin{document}

\titulo{Trabajo Práctico II}
\subtitulo{Grupo 5}

\fecha{\today}

\materia{Algoritmos y Estructuras de Datos III}

\integrante{Aleman, Damian Eliel}{377/10}{damianealeman@gmail.com}
\integrante{Amil, Diego Alejandro}{68/09}{amildie@gmail.com}
\integrante{Barabas, Ariel}{775/11}{ariel.baras@gmail.com}
\integrante{Fern\'andez, Gonzalo Pablo}{836/10}{ralo4155@hotmail.com}

\maketitle


\tableofcontents

\newpage
\section{Introducción}
El presente informe apunta a documentar el desarrollo del Trabajo Práctico número 2 de la materia Algoritmos y Estructuras de Datos III, cursada correspondiente al segundo cuatrimestre del año 2013. Este trabajo pr\'actico consiste en la realización de un análisis teórico-experimental de un conjunto de problemas propuestos por la cátedra. Se requiere, para cada uno de los tres problemas, la implementación de un algoritmo que satisfaga criterios tanto de correctitud como de complejidad temporal.

Vamos a exponer, para cada uno de los problemas, los siguientes apartados:

\begin{itemize}
\item Una interpretación del enunciado, detallando ejemplos y/o casos particulares.
\item Una solución propuesta.
\item Un pseudocódigo que describa la implementación de dicha solución, junto con una explicación de su correctitud y una justificación de su complejidad.
\item Un apartado de testing, tanto de correctitud como de performance.
\end{itemize}


\newpage
\section{Pautas de Implementación}
El lenguaje elegido para la implementación de los algoritmos es \texttt{C++}. De ser necesario vamos a utilizar la biblioteca estandar del mismo y aclarar los costos de las operaciones en cuestión. En el caso de tener que implementar una clase propia para simplificar el código o proveer de cierto encapsulamiento, los costos de los métodos de la misma serán verificados y justificados. La estructura de directorios que utilizaremos para la implementación será la siguiente para todos los ejercicios:

\begin{verbatim}
\codigo
	 timer.h
	 tests.cpp
     \ejx
          ejx.cpp
          ejx.h
          Makefile
\end{verbatim}

El archivo \texttt{timer.h} contiene las funciones necesarias pera medir el
tiempo de ejecución de nuestros programas. Vamos a usar la función
\texttt{clock$\_$gettime} de la librería \texttt{time.h}. Estas funciones son
idénticas para las mediciones en todos los ejercicios. El archivo
\texttt{tests.cpp} contiene el código que testea los ejercicios. A este
programa se le pasan parámetros que utiliza para generar instancias de los
ejercicos y, con las funciones de \texttt{timer.h} mide el tiempo que tardan
en ejecutarse. Las funciones pedidas por la cátedra se encuentran en el
archivo \texttt{ejx.h}, y se incluyen en \texttt{ejx.cpp}. Este archivo
trabaja con entrada y salida standard de manera que, para ejecutar un programa
con su respectivo conjunto de casos, será suficiente con direccionarlo por
consola escribiendo \texttt{./ejx < ejx.in}. Obviaremos mencionar detalles
referentes a la carga de datos en las implementaciones. Cada archivo
\texttt{ejx.h} define los structs \texttt{entrada$\_$ejx} (inicializado a través
de la función \texttt{inicializar$\_$ejx} a partir de la entrada standard) y
\texttt{salida$\_$ejx} (devuelto por la función \texttt{resolver$\_$ejx}, la cual
toma una instancia del tipo \texttt{entrada$\_$ejx}). También se definen las
funciones \texttt{imprimir$\_$ejx}, la cual imprime una instancia de tipo
\texttt{salida$\_$ejx} por la salida standard y \texttt{generar$\_$instancia$\_$ejx}
la cual genera una instancia aleatoria de tipo \texttt{entrada$\_$ejx} a partir
de un parámetro como tamaño de la instancia. El código de \texttt{ejx.cpp}
lee entradas con \texttt{inicializar$\_$ejx} hasta recibir una entrada inválida.
Cada entrada la resuelve con \texttt{resolver$\_$ejx} y luego imprime el
resultado con \texttt{imprimir$\_$ejx}. El código de \texttt{tests.cpp} genera
instancias aleatorias con \texttt{generar$\_$instancia$\_$ejx}, las resuleve con
\texttt{resolver$\_$ejx} y luego imprime los tiempos de ejecución medidos.\\

Para modelar los ejercicios 2 y 3 decidimos implementar una clase grafo mediante listas de adyacencia y desarrollar los algoritmos necesarios como métodos de la misma. Contamos además con métodos auxiliares para simplificar el código, como por ejemplo \texttt{asociar(v,u,p)} que, dados dos nodos $v$ y $u$ crea una arista haciéndolos adyacentes con un peso $p$; o \texttt{vecinos(v)}, que devuelve un vector de todos los nodos que son vecinos a $v$.

Por ejemplo, el programa:

\begin{lstlisting}
	grafo g(3);
	g.asociar(1,2,7);
	g.asociar(2,3,4);
	g.asociar(3,1,2);
	grafo a = g.kruskal();
\end{lstlisting}

Nos va a crear un grafo \texttt{g} de tres nodos, va a crearle 3 aristas con pesos 7, 4 y 2, y luego va a generar un grafo \texttt{a} consistiendo en el AGM de \texttt{g}. 

\newpage
\section{Ejercicio 1}
\subsection{Interpretación del enunciado}
\par{Se tiene una imprenta en la que cada día deben realizarse determinados
trabajos. Los trabajos son realizados por las máquinas de la imprenta. Esta
imprenta cuenta con dos máquinas idénticas. Los trabajos son distintos, pero
pueden ser realizados por cualquiera de la dos máquinas siempre que se repete
un orden dado. Las máquinas deben prepararse para realizar determinados
trabajos y estas preparaciones tienen un costo, el cuál depende tanto del
trabajo a realizar como del estado de la máquina. Teniendo los trabajos
ordenados y los costos de preparación de las máquinas para cada trabajo, se
debe tomar cada trabajo, elegir la máquina en la que se realizará ese trabajo,
prepararla, y realizar el trabajo. El objetivo del ejercicio es desarrollar un
algoritmo que determine la mejor distribución de los trabajos entre las
máquinas que reduzca el costo total de la operación, es decir la suma de los
costos de cada preparación.}

\subsection{Resolucion}
\par{Sea $p(j)$ el problema de obtener la distribución óptima de $j$ trabajos
entre $2$ máquinas, definimos el problema derivado $p'(j,i)$ con $0<i<j$, el
cual consiste en obtener la distribución óptima de $j$ trabajos entre $2$
máquinas con la restricción de que los últimos trabajos realizados en cada
máquina sean $t_j$ y $t_i$. Notar que es indiferente en qué máquina se ejecuta
cada uno de estos. Definimos entonces la $distribucionOptima(j,i$)
como la distribución de costo mínimo con $j$ trabajos y el
trabajo $t_i$ como último trabajo de una de las máquinas. Notar que cualquier
distribución de $j$ trabajos siempre va a tener a $t_j$ como último trabajo en
una de las máquinas. Luego, la solución para el problema $p(j)$ es la solución
de costo mínimo de $p(j,i)$ para todo $i$, es decir:}

$$p(j) = \min_{\forall i} p(j,i) $$

El problema $p'$ se resolvió con programación dinámica mediante $distribucionOptima(j,i)$, que se calcula de la siguiente manera:
$$distribucionOptima(j,i) = \left\{
\begin{array}{c l}
 costo(0,1) & j = 1\\
 distribucionOptima(j-1,0) + costo(j-1,j) & i = 0 \land j > 1\\
 distribucionOptima(j-1,i) + costo(j-1,j) & 1 \le i < j-1 \land j > 1\\
 \displaystyle \min_{\forall k} distribucionOptima(j-1,k) + costo(k,j) & i = j-1 \land j > 1\\
\end{array}
\right$$

\par{Notar que siempre $i < j$. La entrada para ambos problemas es la cantidad
de trabajos y los costos de preparación de las máquinas para cada trabajo
definidos como:}
\begin{equation*}
$costo($i$,$j$) = Costo de preparar una máquina para realizar el trabajo $t_j$ luego
de haber realizado el trabajo $t_i$ (o no haber realizado ningún trabajo si
$i=0$). $0 \leq i < j
\end{equation*}\\
\par{A continuación se muestra el pseudocódigo de la solución para $p(j)$,
resolviendo $p'(j,i)$.}\\
\begin{algorithm}[H]
	\caption{Algoritmo de Ejercicio 1}
	\KwData{\textbf{int} $cantTrabajos$, $costos$}
	$distribucionOptima(1,0) \longleftarrow costo(0,1)$\\
	\For{$j \in \{2..cantTrabajos\}$}{
		
		$distribucionOptima(j,0) \longleftarrow distribucionOptima(j-1,0) + costo(j-1,j)$\\
		$distribucionOptima(j,j-1) \longleftarrow distribucionOptima(j-1,0) + costo(0,j)$
		
		\For{$i \in \{1..j-2\}$}{
	
			$distribucionOptima(j,i) \longleftarrow distribucionOptima(j-1,i) + costo(j-1,j)$\\
			$distribucionOptima(j,j-1) \longleftarrow min(distribucionOptima(j,j-1), distribucionOptima(j-1,i) + costo(i,j))$\\
		}
	}
	\textbf{return} $ \displaystyle \min_{\substack{\forall i}} distribucionOptima(cantTrabajos,i) $\\
\end{algorithm}

\subsection{Demostración de Correctitud}

\textbf{Propiedad 1 -}  \emph{ Existe un $i$ tal que la solución del problema $p'(j,i)$
es solución de $p(j)$. }\\

\textbf{Demostración:} Es inmediato ver que alguna solución del problema $p(j)$ pertenece al
conjunto de las soluciones de $p(j,i)$ para algún $i$. Dado que $p(j)$ no tiene
la restricción del trabajo realizado como último en la otra máquina, la solución al
problema $p(j)$ es la solución de $p(j,i)$ con menor costo:
$$p(j) = \min_{\forall i} p(j,i)$$

\textbf{Propiedad 2 -} \emph{$ (\forall i < j) \quad distribucionOptima(j,i) $ devuelve una de las distribuciones con menor costo dados
j trabajos e i como último trabajo.}\\

\textbf{Demostración:}
Vamos a demostrar por inducción en j.


\textbf{Caso base. Si $j = 1:$}
Queremos ver que $ (\forall i < j) \quad distribucionOptima(1,i) $ devuelve una de las distribuciones con menor costo dados
$1$ trabajo e i como ultimo trabajo. Notar que siempre $i < j$ , por lo tanto $i$ debe ser $0$. 
Hay sólo una forma de configurar un trabajo en las dos máquinas, que este trabajo sea el único de una máquina
y la otra máquina este libre de trabajos, es decir que el costo de la distribución optima es igual a el costo de preparar
el trabajo $1$ luego del trabajo $0$ que es los mismo que \\
$distribucionOptima(1,0) = costo(0,1)$.
Luego, queda demostrado el caso base.\\


\textbf{Paso Inductivo:} Se dividirá la demostración en dos casos :\\

\textbf{Caso $i <  j-1$:} 

Definimos a $ distribucionOptima(j,i) $ = 
$distribucionOptima($j$-1,$i$) + costo($j$-1,$j$) & $\\
Supongamos que no es la distribución optima y lleguemos a un absurdo. Decir que es la distribución es óptima es lo mismo que afirmar que
que no hay ninguna distribución de $j$ trabajos con $i$ como último trabajo que cueste menos que $ distribucionOptima(j,i) $.
Sea $d_2$ la distribución que cuesta menos que $ distribucionOptima(j,i) $ y tiene j trabajos e i como último trabajo.
Luego por como se 
define la preparación de los trabajos,en $d_2$ el trabajo $j$ está al final de una máquina (y como $(i < j-1) \Rightarrow i \neq j $ )
y el trabajo $i$ está en la otra máquina. De esta manera, si quitamos el trabajo $j$, obtenemos una configuración de $j-1$ trabajos con $i$
como último trabajo, llamemosla $d_{2'}$.

\begin{align*}
\text{Por lo tanto}
\qquad costo (d_2) &= costo (d_{2'}) + costo(j-1,j) \\
\text{y como }
 \qquad costo(d_2) &<  distribucionOptima(j,i) \\ 
\text{entonces}
 \qquad costo (d_{2'}) &< distribucionOptima(j-1,i). Absurdo. 
\end{align*}

\textbf{Caso $i = j-1:$}

Definimos a $ distribucionOptima(j,i) $ =  
\displaystyle \min_{\substack{\forall i \in \{0..j-1\}}} \ $distribucionOptima($j$-1,$i$) + costo($i$,$j$)$
 & \\
 
Supongamos que no es la distribución optima y lleguemos a un absurdo.
Sea $d_2$ la distribución que cuesta menos que $ distribucionOptima(j,i) $ y tiene $j$ trabajos e $i$ como último trabajo.
Luego por como se 
define la preparación de los trabajos,en $d_2$ el trabajo $j$ está al final de una máquina (y como $(i = j-1) \Rightarrow i \neq j $ )
y el trabajo $i$ está en la otra máquina.
De esta manera, si quitamos el trabajo j, obtenemos una configuración de $j-1$ trabajos con $j-1$
y algún k (con $ 0 \le k < j-1$) como últimos trabajos en las dos máquinas, llamemosla $d_{2'}$.

\begin{align*}
\text{Por lo tanto}
\qquad costo (d_2) &= costo (d_{2'}) + costo(k,j)  \text{para algun k (con }  0 \le k < j-1) \\
\text{y como }
 \qquad costo(d_2) &<  distribucionOptima(j,i) \\ 
\text{entonces}
 \qquad costo (d_{2'}) &< distribucionOptima(j-1,k). 
\end{align*}

Absurdo pues por la definición
\displaystyle \min_{\substack{\forall k \in \{0..j-1\}}} \ $distribucionOptima($j$-1,$k$) + costo($k$,$j$)$. \\

\textbf{Corolario de Propiedad 2 -} \emph{$ distribucionOptima(cantTrabajos,i) $ devuelve una de las distribuciones con menor costo dados
cantTrabajos trabajos e i como ultimo trabajo.)}\\

Ahora veremos que el algoritmo retorna la distribución óptima según como la definimos.\\
\textbf{Caso $j = 1:$} En la línea $1$ del pseudocódigo se asigna la distribución óptima según como la definimos:
$$costo(1,0)$$
\textbf{Para $j > 1$:}
\textbf{Caso $i = 0$:} En la línea $3$ del pseudocódigo se asigna la distribución óptima según como la definimos:
$$distribucionOptima(j-1,0) + costo(j-1,j)$$
\textbf{Caso $1 \le i < j-1:$} En la línea $6$ del pseudocódigo se asigna la distribución óptima según como la definimos:
$$distribucionOptima(j-1,i) + costo(j-1,j)$$
\textbf{Caso $i = j-1:$} Sea $g(i) = distribucionOptima(j-1,i) + costo(i,j).$
En este caso se inicializa $distribucionOptima(j,j-1)$ en la linea $4$ con $g(0)$.
Luego en el ciclo interno se compara el valor actual de $distribucionOptima(j,j-1)$ con los valores
$g(i)$ para todo $ 1 \le i \le j-2 $. 


\subsection{Cota de Complejidad}
Para demostrar la complejidad del algoritmo, vemos que la incialización de la matriz de costos
tiene una complejidad de $O(n^2)$, con $n$ la cantidad de trabajos, ya que completa una matriz de $n^2$ elementos con dos ciclos anidados.

Luego, el algoritmo completa la matriz de $distribucionOptima$, con dos ciclos anidados. En cada uno de estos ciclos,
las variables de iteración se van incrementando de a uno y cada una debe iterar
en un numero menor o igual a la cantidad de trabajos.
\textbf{Cada asignación tiene una complejidad de $O(1)$, ya que al momento de la asignación (en la iteración $j$)
$distribucionOptima(j-1,i)$ ya está calculada y guardada en la matriz.}\\

\begin{algorithm}[H]
	\caption{Algoritmo de Ejercicio 1}
	\KwData{\textbf{int} $cantTrabajos$, $costos$}
	$distribucionOptima(1,0) \longleftarrow costo(0)(1)$	         \tcc*[r]{$O(1)$}
	\For{$j \in \{2..cantTrabajos\}$ \tcc*[r]{$O(n)$}}{	
		
		$distribucionOptima(j,0) \longleftarrow distribucionOptima(j-1)(0) + costo(j-1)(j)$\\ 	
		$distribucionOptima(j,j-1) \longleftarrow distribucionOptima(j-1)(0) + costo(0)(j)$		
		
		\For{$i \in \{1..j-2\}$ \tcc*[r]{$O(n)$} }{	
	
			$distribucionOptima(j,i) \longleftarrow distribucionOptima(j-1,i) + costo(j-1,j)$\\	${O(1)}$
			$distribucionOptima(j,j-1) \longleftarrow min(distribucionOptima(j,j-1), distribucionOptima(j-1)(i) + costo(i)(j))$\\
		}
	}
	\textbf{return} $ \displaystyle \min_{\substack{\forall i}} distribucionOptima(cantTrabajos,i) $	\tcc*[r]{$O(n)$}
\end{algorithm}

Luego la complejidad del algoritmos es $ O(n^2) + n * O(n) + O(n) = O(n^2).$

\subsection{Implementación}

\begin{lstlisting}
salida_ej1 resolver_ej1(entrada_ej1 entrada) {
  int cantTrabajos = entrada.n;
  vector<vector<int> > costo = entrada.matriz;

  //En la posicion i j está el costo de la configuración optima para
  //los primeros i trabajos con el trabajo j como ultimo de la otra máquina.
  vector < vector<par> > 
	distribucionOptima(cantTrabajos+1,vector<par> (cantTrabajos) );
  distribucionOptima[1][0] = make_pair(costo[0][1],0);
  
  for ( int j = 2; j <= cantTrabajos; ++j ) {
    distribucionOptima[j][0] = make_pair (distribucionOptima[j-1][0].first 
					+ costo[j-1][j],j-1);
    distribucionOptima[j][j-1] = make_pair (distribucionOptima[j-1][0].first
					+ costo[0][j],0);
    for ( int i = 1; i < j-1 ; ++i ) {
      distribucionOptima[j][i] = make_pair(distribucionOptima[j-1][i].first
						    + costo[j-1][j],j-1);
      int costoAlternativo = distribucionOptima[j-1][i].first + costo[i][j];
      if (costoAlternativo < distribucionOptima[j][j-1].first) {
	    distribucionOptima[j][j-1] = make_pair(costoAlternativo,i);
      }
    }
  }

  vector<int> tEnMaq1(0);
  int k = 0;
  int C = distribucionOptima[cantTrabajos][0].first;

  int elAnterior = cantTrabajos-1;
  int ultimoOtraMaquina = 0;
  for (int i=1; i<cantTrabajos; i++) {
      if (distribucionOptima[cantTrabajos][i].first < C) {
	C = distribucionOptima[cantTrabajos][i].first;
	elAnterior = distribucionOptima[cantTrabajos][i].second;
	ultimoOtraMaquina = i;
      }
  }
  
  //cout << "el anterior es " << elAnterior << endl;

  int trabajoEnMaquina = cantTrabajos;	
  tEnMaq1.push_back(trabajoEnMaquina);
  k++;

  while (elAnterior != 0){
    tEnMaq1.push_back(elAnterior);
    k++;
    trabajoEnMaquina = elAnterior;
    while (ultimoOtraMaquina > elAnterior) {
      ultimoOtraMaquina = 
	      distribucionOptima[ultimoOtraMaquina][elAnterior].second;
      }
    elAnterior = 
	distribucionOptima[trabajoEnMaquina][ultimoOtraMaquina].second;
}

  vector<int> e = vector<int>(k);
  for (int i=0; i<k; i++) {
	  e[i] = tEnMaq1[k-i-1];
  }

  //cout << "vector de trabajos" << endl;
  //mostrarVec(e);
  
  //cout << "matriz de cofiguracion Optima" << endl;
  //mostrarMatriz(distribucionOptima,cantTrabajos+1,cantTrabajos);
  //cout << endl;

  salida_ej1 salida(C, k, e);
  return salida;
}
\end{lstlisting}
\newpage
\subsection{Testing de Correctitud}

Los tests expuestos a continuación fueron diseñados con el fin de verificar
diferentes casos particulares que pudimos identificar. Para cada test vamos
a exponer la entrada, la salida y, en caso de que sea necesario, una
justificaci\'on de la correctitud de la soluci\'on.\\

\noindent\textbf{Test$\#$1}\\
\textbf{Caracterización:} Un solo trabajo.\\
\textbf{Input:}\\ \texttt{1\\5}\\
\textbf{Output:} \texttt{5 1 1}\\
\textbf{Status:} OK. Asigna el único trabajo a la máquina 1 y el costo total
es el costo de ubicar el trabajo 1 en la máquina 1.\\

\noindent\textbf{Test$\#$2}\\
\textbf{Caracterización:} Dos trabajos que conviene ubicar uno trás otro.\\
\textbf{Input:}\\ \texttt{2\\3\\8 3}\\
\textbf{Output:} \texttt{6 2 1 2}\\
\textbf{Status:} OK. Ubica ambos trabajos en la misma máquina y el costo
total es el costo de ubicar el trabajo 1 en una máquina vacía más el costo
de ubicar el trabajo 2 tras el trabajo 1.\\

\noindent\textbf{Test$\#$3}\\
\textbf{Caracterización:} Dos trabajos que conviene ubicar como primeros en
una máquina.\\
\textbf{Input:}\\ \texttt{2\\3\\3 8}\\
\textbf{Output:} \texttt{6 1 2}\\
\textbf{Status:} OK. Ubica un trabajo en cada máquina y el costo total es la
suma de los costos de ubicar cada trabajo como primer trabajo de una máquina.\\

\noindent\textbf{Test$\#$4}\\
\textbf{Caracterización:} Todos los costos son iguales.\\
\textbf{Input:}\\ \texttt{5\\1\\1 1\\1 1 1\\1 1 1 1\\1 1 1 1 1}\\
\textbf{Output:} \texttt{5 5 1 2 3 4 5}\\
\textbf{Status:} OK. Cualquier distribución de trabajos en máquinas tiene
costo igual a 5.\\

\noindent\textbf{Test$\#$5}\\
\textbf{Caracterización:} Todos los costos son distintos (incrementando).\\
\textbf{Input:}\\ \texttt{3\\1\\2 3\\4 5 6}\\
\textbf{Output:} \texttt{8 1 3}\\
\textbf{Status:} OK.\\

\noindent\textbf{Test$\#$6}\\
\textbf{Caracterización:} Todos los costos son distintos (decrementando).\\
\textbf{Input:}\\ \texttt{3\\6\\5 4\\3 2 1}\\
\textbf{Output:} \texttt{11 3 1 2 3}\\
\textbf{Status:} OK.\\

\noindent\textbf{Test$\#$7}\\
\textbf{Caracterización:} Varios trabajos que conviene ubicar uno tras otro.\\
\textbf{Input:}\\ \texttt{5\\1\\8 1\\8 8 1\\8 8 8 1\\8 8 8 8 1}\\
\textbf{Output:} \texttt{5 5 1 2 3 4 5}\\
\textbf{Status:} OK. Ubica todos los trabajos en la misma máquina.\\

\noindent\textbf{Test$\#$8}\\
\textbf{Caracterización:} Varios trabajos que conviene ubicar intercalados.\\
\textbf{Input:}\\ \texttt{5\\1\\1 8\\8 1 8\\8 8 1 8\\8 8 8 1 8}\\
\textbf{Output:} \texttt{5 3 1 3 5}\\
\textbf{Status:} OK. Ubica los trabajos impares en la misma máquina (por lo
tanto, ubica los pares en la otra).\\

\noindent\textbf{Test$\#$9}\\
\textbf{Caracterización:} La subsolución de la solución óptima no es solución
óptima para su subproblema asociado.\\
\textbf{Input:}\\ \texttt{3\\5\\3 5\\1 8 8}\\
\textbf{Output:} \texttt{11 1 3}\\
\textbf{Status:} OK. La solución óptima consiste en ubicar los trabajos 1 y 2 en
una máquina y el trabajo 3 en otra. Sin embargo, la subsolución de dos trabajos
que ubica a los trabajos 1 y 2 en la misma máquina no es solución óptima del
subproblema asociado (distribuír los trabajos 1 y 2 entre las 2 máquinas de
forma óptima.)\\

\newpage
\subsection{Testing de Performance}

\par{Realizamos un gráfico comparando la sucesión de tiempos obtenida con una función cuadrática, pues demostramos que la complejidad del algoritmo es cuadrática.
La función $f = n^2$, donde $C = \frac{1}{20500000}$. Se midió el tiempo con n desde 1 hasta 1000 con saltos de a 10 (con 100 mediciones por cada n).}


\begin{figure}[H]
\centering
\includegraphics{imgs/ej1_1000_10_100.pdf}
\caption{Test Performance: Tiempo(s) vs Cantidad de Trabajos.}
\end{figure}

\par{Hay que notar que las franjas para cada tamaño de entrada muestran el desvío estandar de todos los valores conseguidos para ese tamaño, esto nos pareci\'o mucho m\'as significativo que simplemente mostrar el máximo y el m\'inimo, ya que estos valores pueden variar mucho por otros procesos que pueda estar ejecutando la computadora a la vez.}


\newpage
\section{Ejercicio 2 - A}

\subsection{Interpretación del enunciado}

El siguiente problema trata sobre el almacenamiento y la transmisión de datos entre servidores. Se tiene un conjunto de $n$ servidores interconectados de a pares a través de $m$ enlaces. Un servidor $v_i$ puede transmitirle información a otro servidor $v_j$ de manera directa si y sólo si existe un enlace entre dichos servidores. Si no, la información deberá llegar a $v_j$ pasando por otros servidores intermedios.

El tiempo que tarda la información en transmitirse a través de un enlace es constante para todos los enlaces, aunque el uso de estos enlaces tiene un costo que depende de cada enlace en cuestión. Se cumple que para todo par de servidores en la red, el costo de enviar información en un sentido es el mismo que el de enviarla en el sentido contrario.

En esta red de servidores, se llama ``broadcast'' al proceso de replicar información entre todos los servidores, de manera que después de un determinado tiempo, todos los servidores de la red tengan la misma información. Para hacer esto, se designa a un servidor como $master$, que es el único que inicialmente tiene la información a replicar. Este servidor $master$ va a copiar la información a todos aquellos servidores que están conectados directamente a él.

Cada vez que un servidor recibe una actualización, envía la información recibida a todos los otros servidores conectados a él (exceptuando al servidor del cual recibió la actualización). Todas las transmiciones realizadas por un servidor se hacen simultáneamente. Esto se repite hasta que todos los servidores de la red tengan la misma información.

El objetivo de este ejercicio es el de desarrollar un algoritmo que determine un conjunto de enlaces en el que el costo de mantenerlos todos sea el mínimo y a su vez, la actualización pueda llegar desde el servidor $master$ a cada uno de los demás servidores.

\subsection{Resolución}

Se modelará la situación a través de un grafo ponderado no dirigido en el que cada nodo representará un servidor y cada arista entre un par $v1$, $v2$ de nodos representará un enlace entre los servidores correspondientes a $v1$ y $v2$. Se le asignará a cada arista el peso equivalente al costo de mantener su correspondiente enlace. Se asume que el grafo es conexo, ya que se parte de que todos los servidores están interconectados. 

Por ejemplo, el siguiente grafo:\\

\begin{center}
\begin{dot2tex}
graph graphname{
	rank=same;
	rankdir="LR";
	splines=line;
	{rank=same; 1 5}
	{rank=same; 2 6}
	{rank=same; 3 7}
	{rank=same; 4 8}
	1 -- 2 [label=5];
	2 -- 3 [label=17];
	3 -- 4 [label=3];
	1 -- 5 [label=24];
	5 -- 6 [label=8];
	6 -- 7 [label=16];
	7 -- 8 [label=19];
	4 -- 8 [label=7];
	2 -- 6 [label=51];
	3 -- 7 [label=21];
	5 -- 2 [label=29];
	2 -- 7 [label=3];
	7 -- 4 [label=65];
}
\end{dot2tex}
\end{center}

%revisar por que los nodos están a diferentes alturas
representa a una red con 8 servidores donde, por ejemplo, el costo de hacer una transmición entre los nodos 2 y 6 es de 51. 
%Si se asigna al
%~ servidor número 7 para ser el servidor $master$, los datos se
%~ transmitirían de la siguiente manera:
%~ 
%~ 
%~ \begin{center}
%~ \begin{dot2tex}
%~ graph graphname{
	%~ rankdir="LR";
	%~ ratio="compress";
	%~ nodesep=0.0005;
	%~ size="0.50,0.50";
	%~ 7 -- 2;
	%~ 7 -- 3;
	%~ 7 -- 4;
	%~ 7 -- 6;
	%~ 7 -- 8;
	%~ 2 -- 1;
	%~ 2 -- 5;
%~ }
%~ \end{dot2tex}
%~ \end{center}
%~ 
%~ Dado que se pide que los subproblemas de encontrar el conjunto de enlaces
%~ de costo mínimo y de seleccionar el servidor $master$ para minimizar el tiempo
%~ de ``broadcast'' se resuelvan por separado, se implementarán 2 algoritmos
%~ independientes para resolverlos.}

Para resolver este problema se debe, dado un conjunto de $n$ servidores y $m$ enlaces, encontrar un subconjunto de $m' < m$ enlaces tales que sea posible conectar todos los $n$ servidores entre si, pero utilizando los enlaces de menor costo posible. Esto se puede pensar como, dado un grafo conexo con pesos en sus aristas, encontrar un árbol generador mínimo del mismo. Vale aclarar que dado un grafo $G$, puede haber más de un AGM del mismo. En este ejercicio no hay requerimientos específicos sobre cuál AGM devolver. Para el grafo de la figura anterior, un AGM posible podría ser el siguiente:
 
%~ desde: 2 - hasta: 7 - peso: 3
%~ desde: 3 - hasta: 4 - peso: 3
%~ desde: 1 - hasta: 2 - peso: 5
%~ desde: 4 - hasta: 8 - peso: 7
%~ desde: 5 - hasta: 6 - peso: 8
%~ desde: 6 - hasta: 7 - peso: 16
%~ desde: 2 - hasta: 3 - peso: 17

\begin{center}
\begin{dot2tex}
graph graphname{
	rank=same;
	rankdir="LR";
	splines=line;
	{rank=same; 1 5}
	{rank=same; 2 6}
	{rank=same; 3 7}
	{rank=same; 4 8}
	1 -- 2 [label=5];
	2 -- 3 [label=17];
	3 -- 4 [label=3];
	1 -- 5 [label=24, style=dotted];
	5 -- 6 [label=8];
	6 -- 7 [label=16];
	7 -- 8 [label=19, style=dotted];
	4 -- 8 [label=7];
	2 -- 6 [label=51, style=dotted];
	3 -- 7 [label=21, style=dotted];
	5 -- 2 [label=29, style=dotted];
	2 -- 7 [label=3];
	7 -- 4 [label=65, style=dotted];
}
\end{dot2tex}
\end{center}

Optamos por implementar el Algoritmo de Kruskal para obtener el AGM. Ya que el mismo cumple con la cota de complejidad temporal requerida en este problema. Este algoritmo
itera sobre las aristas agregando al resultado la arista de menor peso que no
forme ciclos. Para esto, se ordenan las aristas según su peso (para así
obtener en tiempo constante la arista de menor peso) y luego se las agrega
hasta formar un árbol. A continuación se muestra el pseudocódigo del algortimo
de Kruskal:\\

\begin{algorithm}[H]
	\caption{Algoritmo de Kruskal}
	\KwData{\textbf{Grafo} $G$}
	$G.aristas \longleftarrow$ ordenar$\_$por$\_$peso$\_$asc($G.aristas$)\\
	\textbf{vector<arista>} $res \longleftarrow \emptyset$\\
	\ForEach{$e \in G.aristas$}{
		\If{!forma$\_$ciclo($e$, $res$)}{
		$res \longleftarrow res \cup e$\\
		}
	}
	\textbf{return} $res$\\
\end{algorithm}

\subsection{Demostración de Correctitud}

\textbf{Propiedad 1 -} \emph{Un árbol generador mínimo del grafo que representa
la red es una solución al problema.}\\

\par{Un AGM es un árbol y por lo tanto existe un único camino entre todo par
de vértices; entonces se cumple que cualquier servidor puede enviar
información y que todo otro servidor recibirá dicha información eventualmente.
El costo de mantener todos los enlaces es la suma de los costos de mantener
cada enlace, es decir es equivalente al peso del árbol generador; y, como de
todos los posibles árboles generadores, se toma el de mínimo peso, podemos
asegurar que estamos tomando el conjunto de enlaces de menor costo.}\\

\textbf{Propiedad 2 -} \emph{El algoritmo de Kruskal genera una árbol generador
mínimo a partir de un grafo conexo con pesos en los ejes.}\\

\par{Está demostrado que el algoritmo de kruskal es correcto (citar
demostración) y devuelve un AGM. La demostración de correctitud consiste en
probar que el resultado es un árbol ya que la cantidad de aristas es una menos
que la cantidad de nodos y no forman ciclos ya que sólo se agregan las aristas
que no forman ciclos con el conjunto de aristas ya agregado. La demostración
de optimalidad se basa en suponer que existe un árbol generador de menor
peso que el que devuelve el algoritmo de Kruskal y se llega a un absurdo,
ya que Kruskal elige siempre las aristas de menor peso.}

\subsection{Cota de Complejidad}

\par{La complejidad del algortimo es la complejidad de Kruskal.}

\subsection{Implementación}

Nuestra implementación del algoritmo de Kruskal es la siguiente:
\begin{lstlisting}
  
  grafo kruskal(){
    disjointSet ds(this->cantNodos);
    vector<arista> res;
    /*Ordeno la lista de aristas para Kruskal */
    sort(this->aristas.begin(), this->aristas.end()); 
        
    int i = 0;  /* Indice de arista que itero */
    int j = 0;  /* Cantidad de aristas que agrege */
    
    while ( j < this->cantNodos-1 ) {
      if(!ds.sameSet(this->aristas[i].nodo1, this->aristas[i].nodo2)){
        ds.union_by_rank(this->aristas[i].nodo1, this->aristas[i].nodo2);
        res.push_back(this->aristas[i]);
        j++;
      }
      i++;
    }
    
    /* Armo el grafo para devolver */
    grafo g_res(this->cantNodos);
    for(int i = 0; i < res.size(); ++i)
      g_res.asociar(res[i].nodo1+1, res[i].nodo2+1, res[i].peso);
    
    return g_res;
  }
\end{lstlisting}

Notar que fue necesario agregar a nuestro programa una parte que creara un grafo para poder devolver algo que la segunda parte del ejercicio 2 pudiera procesar sin tener que volver a leer la entrada.


\newpage
\section{Ejercicio 2 - B}
\subsection{Interpretación del enunciado}
Para este ejercicio se pide, dada una red de servidores, los cuales están todos interconectados entre sí con un conjunto de enlaces de costo mínimo; encontrar un servidor tal que, al designarlo $master$ de la red, se garantice que el broadcast se complete en el menor tiempo posible. Vale destacar que este problema toma como entrada la salida del anterior.

Veamos algunos ejemplos. Si consideramos la solución propuesta anteriormente, tenemos que la misma es el siguiente árbol:

\begin{center}
\begin{dot2tex}
graph graphname{
	rank=same;
	rankdir="LR";
	splines=line;
	{rank=same; 1 5}
	{rank=same; 2 6}
	{rank=same; 3 7}
	{rank=same; 4 8}
	1 -- 2 [label=5];
	2 -- 3 [label=17];
	3 -- 4 [label=3];
	4 -- 8 [label=7];
	5 -- 6 [label=8];
	6 -- 7 [label=16];
	2 -- 7 [label=3];
}
\end{dot2tex}
\end{center} 

Veamos entonces, cuanto tardaría en completarse un broadcast para cada servidor. El enunciado dice que el tiempo de transimisón de cada enlace es el mismo dados dos servidores cualesquiera que estén conectados. Por lo que podemos asumir que, si tenemos un árbol $A$ y un vertice $v$, el tiempo total del broadcast es la altura del árbol $A$ tomando a $v$ como raiz del mismo. Para el árbol del ejemplo, los tiempos totales de broadcast correspondientes para cada servidor son los siguientes:

\begin{center}
  \begin{tabular}{ c | c | c | c | c | c | c | c | c}
    servidor & 1 & 2 & 3 & 4 & 5 & 6 & 7 & 8 \\ \hline
    tiempo   & 4 & 3 & 4 & 5 & 6 & 5 & 4 & 6 \\
    \end{tabular}
\end{center}

Se observa que en este caso la única solución óptima es seleccionar como $master$ al servidor número 2. Podemos ver que obtuvimos los peores resultados cuando seleccionamos como $master$ a los servidores 5 y 8 que coincidentemente son hojas del árbol. 

\subsection{Resolución}

La pregunta que nos surge entonces es: Dado un árbol, ¿cuál nodo deberíamos seleccionar como raíz del mismo para garantizar que este árbol tenga altura mínima? Propondremos y demostraremos que el nodo óptimo es aquel que se encuentra a la mitad del camino más largo del árbol. Nuestra solución va a implementar un algoritmo para encontrarlo en
tiempo lineal. 

Nuestra resolución implementa el siguiente procedimiento. Primero se realiza un BFS partiendo desde el nodo raíz del árbol. Nos vamos a quedar con el nodo más lejano que podamos encontrar a la raíz. De ser más de uno, nos vamos a quedar con el último que hayamos recorrido. Llamaremos a este nodo $v_1$.

Luego, vamos a hacer otro BFS partiendo desde el nodo $v_1$, y nos vamos a quedar con el nodo más alejado a $v_1$ que encontremos, al cual llamaremos $v_2$. Luego, el camino más largo del árbol es el que está comprendido entre los nodos $v_1$ y $v_2$. Haciendo un DFS\footnote{Un DFS con algunas modificaciones para encontrar el camino y luego devolver el nodo medio de este.} desde $v_1$ hasta $v_2$ vamos a ir recorriendo este camino. El nodo que buscamos es aquel que se encuentra en la mitad de este camino.

\newpage
Un pseudocódigo de lo anteriormente descripto es el siguiente:\\

\begin{algorithm}[H]
	\caption{Busqueda del nodo medio del camino más largo de un árbol}
	\KwData{\textbf{Arbol} $A$}
	$v_1 \longleftarrow BFS(raiz(A))$\\
	$v_2 \longleftarrow BFS(v_1)$\\
	$v_{medio} \longleftarrow DFS(v_1,v_2)$\\
	\textbf{return} $v_{medio}$\\
\end{algorithm}

\subsection{Demostración de Correctitud}

Antes de demostrar nuestra solución, tenemos que enunciar dos propiedades sencillas:\\

\textbf{Propiedad 1 -}  \emph{Sea $A$ un árbol, cualquier nodo de $A$ puede ser raiz.}\\

\textbf{Justificación:} Para que un grafo $G$ sea un árbol, se tienen que cumplir dos propiedades: que sea conexo y que no tenga ciclos. Es fácil ver que esto se sigue cumpliendo independientemente de que nodo tomemos como raiz del mismo.\\

\textbf{Propiedad 2 -} \emph{Sea $P$ un camino compuesto por los nodos $v_0, v_1, ... , v_n$, siempre existe por lo menos un ``nodo medio'' $v_m$ del mismo tal que $dist(v_0, v_m) \geq dist(v_0, v_i)$ y $dist(v_m, v_n) \geq dist(v_i, v_n) \forall i \neq m$.}\\

\textbf{Justificación:} La idea intuitiva es que dado un camino $P$ siempre existe un nodo que se encuentra a la mitad del mismo. Si el camino $P$ tiene una cantidad impar de nodos entonces es fácil ver que este nodo es único. Si $P$ tiene una cantidad par de nodos, entonces hay 2 posibles candidatos que cumplen la propiedad.\\

\textbf{Propiedad 3 -}  \emph{Sean $A$ un árbol y $P = \{p_1, p_2, ... ,p_n\}$ un conjunto de todos sus posibles caminos. Si $p_{max}$ es camino máximo en $A$ entonces tomar el nodo medio $v_m$ que se encuentra a la mitad de $p_{max}$ como una raiz de $A$ nos garantiza que la altura de $A$ va a ser la mínima posible.}\\

\textbf{Justificación:} Vamos a probarlo por el absurdo, asumiendo que puedo tomar un nodo distinto a $v$ como raiz y aún así obtener un árbol de menor altura. Lo separamos en dos casos:\\

\textbf{Caso 1:} Asumo que puedo tomar un nodo que no está en $p_{max}$ como raiz y que puedo obtener un árbol de menor altura. De haber tomado a $v_m$ como nueva raiz del $A$, es fácil ver que este la altura de $A$ seria $length(p_{max})/2$. Asumir que existe un nodo $v'$ tal que $v' \in p_i$ y que tomar a $v'$ como raiz de $A$ va a generar un árbol de menor altura que el generado habiendo elegido a $v_m$, es equivalente a decir que $length(p_{max})/2 > length(p_i)/2$, es decir, que $length(p_{max}) > length(p_i)$, lo cual es un absurdo.\\

\textbf{Caso 2:} Asumo que puedo tomar como raiz de $A$ a un nodo de $p_{max}$ que no sea $v_m$ y aún así obtener un árbol de menor altura que tomando a $v_m$. Esto es absurdo por la propiedad 2 vista anteriormente.\\

%Asumo que puedo tomar como raiz de $A$ a un nodo de $p_{max}$ que no sea $v_m$. Si asumimos que $v$ está ubicado en la mitad del camino $p_{max}$, el árbol generado al tomar a $v$ como raiz tiene $length(p_{max})/2$ de altura para ambos lados. Ahora bien, si elijo como raiz a un nodo $v'$ que no está en el medio de $p_{max}$, incondicionalmente alguno de sus dos subcaminos (el de la derecha o el de la izquierda) va a ser mayor a $length(p_{max})/2$, probando el absurdo que queríamos.\\

Luego, por 1) y por 2) podemos ver que no es posible elegir otro nodo que no sea $v_m$ como raiz de $A$ y obtener un árbol de menor altura que tomando $v_m$.\\

\newpage
Con estas tres propiedades enunciadas, la demostración principal de nuestro algoritmo es la de la siguiente propiedad: \emph{Sea $A$ un árbol y sean $v$ y $w$ las hojas de $A$ que determinan el camino máximo $p_{max}$ dentro de $A$, vale que, dado un nodo $u$ cualquiera de $A$, el nodo a mayor distancia de $u$ es o $v$ o $w$.}\\

Vamos a dividir la demostración en dos casos. Primero vamos a ver el caso en el que $u$ se encuentra en $p_{max}$. Lo vamos a demostrar por el absurdo. Supongamos que existe un nodo $v'$ tal que $dist(u,v') > dist(u,v)$. Luego podemos crearnos un nuevo camino $p'$ que esté compuesto por el camino que va desde $v'$ hasta $u$ y luego desde $u$ hasta $w$. Viendo que la longitud de $p_{max}$ es $dist(u,v) + dist(u,w)$ y $dist(u,v') > dist(u,v)$ tenemos que la longitud de $p'$ es mayor a la de $p_{max}$, invalidando nuestra hipótesis de que $p_{max}$ es el camino máximo en $A$ y llegando a un absurdo.\\

En el segundo caso vamos a ver que pasa cuando $u$ no se encuentra en $p_{max}$. Si $u$ no se encuentra dentro de $p_{max}$ entonces tenemos que definir un nuevo nodo llamado $c$, el cual si es un nodo dentro de $p_{max}$ y, de hecho, es el nodo más cercano a $c$ que está dentro de $p_max$. Este nodo siempre es único ya que todo árbol carece de ciclos. Es fácil ver que, según nuestra propiedad, el nodo más alejado a $u$ debería ser la hoja más alejada a $c$ dentro de $p_{max}$. Podemos suponer sin perder generalidad que esta hoja es $v$.\\

De la misma manera que en el caso anterior, supongamos ahora otro nodo $v'$ tal que $dist(u,v') > dist(u,v)$. Esto implica que $dist(c,v') > dist(c,v)$. Luego vuelve a existir un camino $p'$ que está compuesto por el camino entre $v'$ y $c$ y entre $c$ y $q$. Luego como $dist(c,v') > dist(c,v)$, tenemos que la longitud de este camino es mayor a la de $p_{max}$ y nuevamente un absurdo para este caso. \Box

\subsection{Cota de Complejidad}

Para justificar la complejidad de nuestra solución tenemos que tener en cuenta que la cantidad de aristas de un árbol siempre va a ser igual a $n-1$, donde $n$ es la cantidad de nodos del mismo. Nuestra solución consiste en correr dos BFS y luego un DFS. La complejidad de peor caso de hacer esto es:

\begin{center}
$O(m) + O(m) + O(m) = 3 * O(m) = O(m)$
\end{center}

Pero si tenemos en cuenta que el grafo es un árbol, donde $m = n - 1$, la complejidad de nuestra solución termina quedando $O(n)$.

\subsection{Implementación}

La implementación de BFS que hicimos fue la estandar, con la particularidad de que no hace una búsqueda en sí, sino que devuelve el último nodo que visitó, que es uno de los más alejados al nodo que recibe como parámetro $n$.

\begin{lstlisting}
int bfs(int n, bool *visitadas) {
  /* Creo una cola y encolo a n */
  queue<int> q;
  q.enqueue(n);
  int nodoActual;
  
  while(!q.vacia()){
    /* Me guardo en una variable al nodo actual */
    nodoActual = q.first();
    q.dequeue();
    
    /* Encolo los vecinos de nodoActual */
    for(int i = 0; i < this->vecinos(nodoActual).size(); ++i){
      int u = this->vecinos(nodoActual)[i];
      if(!visitadas[u]){
        visitadas[u] = true;
        q.enqueue(u);
      }
    }
  }
  return nodoActual;
}
\end{lstlisting}

A diferencia de nuestra versión de BFS, la versión de DFS que implementamos si realiza un search. Nuestra implementación va guardando en memoria el camino que va recorriendo desde el nodo $n$ hasta el nodo $m$ en cada llamada recursiva. Al dar con el nodo buscado, devuelve el valor del nodo que se encuentra a mitad del camino que llevaba recorrido.

\begin{lstlisting}
int dfs(int n, int m){
  /* Valor del nodo master */ 
  int U = 0;
  
  /* Me creo un vector para irme guardando 
   * el camino que voy recorriendo 
   */
  vector<int> p_max;
  p_max.push_back(n);
  
  /* Seteo lista de nodos visitados y flag de encontrado */
  bool encontrada = false;
  bool visitadas[this->cantNodos];
  for(int i = 0; i < this->cantNodos; i++)
    visitadas[i] = false;
  visitadas[n] = true;
   
  for(int j = 0; j < (*this->lista_global[n]).size(); ++j){
    if(visitadas[(*this->lista_global[n])[j]] == false){
      return dfs_recur((*this->lista_global[n])[j], m, 
                        visitadas, encontrada, p_max, U);
    }
    if (encontrada) break;
  }
}

int dfs_recur(int n, int m, bool* visitadas, 
              bool &encontrada, vector<int> p, int U){
  p.push_back(n);
  visitadas[n] = true;
  
  if (n == m){
    /* Encontre el nodo que buscaba */
    encontrada = true;
    /* Devuelvo el valor de la mitad del vector que tengo hasta ahora */
    U = p[p.size()/2]+1;
    return U;
  }
  for(int j = 0; j < (*this->lista_global[n]).size(); ++j){
    if(visitadas[(*this->lista_global[n])[j]] == false)
      U = dfs_recur((*this->lista_global[n])[j], m, 
                     visitadas, encontrada, p, U);
  }
  return U;
}
\end{lstlisting}



\newpage
\subsection{Testing de Correctitud}

Los tests expuestos a continuación fueron diseñados con el fin de verificar diferentes casos particulares que pudimos identificar. Para cada test vamos a exponer la entrada, la salida y, en caso de que sea necesario, una justificaci\'on de la correctitud de la soluci\'on. Tener en cuenta que este ejercicio toma como entrada la salida del ejercicio anterior, es decir, un árbol generador mínimo. No consideramos los pesos de las aristas ya que no son relevantes respecto a lo que pide este ejercicio.\\

\begin{figure}[H]
\centering
\def\svgwidth{340 pt}
\input{imgs/tests_ej2.pdf_tex}
\end{figure}

Para todos los tests los resultados fueron correctos.

\subsection{Testing de Performance}

Para realizar el test, generamos árboles aleatorios, con pesos aleatorios (entre 0 y 100). Decimos que los árboles son aleatorios, porque nuestro algoritmo va agrandando una componente conexa uniendo vertices
que están en la frontera de la componente (adyacentes a vertices que estan en la componente conexa) con cualquier (usando random mod $i-1$ ) nodo que ya está en ella.
Justificaremos que los grafos que generamos son árboles, es decir son conexos y tienen n-1 ejes.
Son conexos porque conectas cada nodo al resto de la componenete conexa que vas construyendo y tiene $n-1$ ejes
porque es la cantidad de aristas que se agregan (se cicla en la cantidad de nodos empezando por el segundo).

A continuacion exponemos el código que genera las aristas de dicho árbol.

\begin{lstlisting}
void generar_aristas_aleatorias() {

int n = this->cantNodos;
//cout << "Generando un grafo random:" << endl;
//cout << "Cantidad de nodos: " << n << endl;

srand(time(NULL));

for(int i = 2; i <= n; ++i){
  int nodo = rand()%(i-1)+1;
  //cout << "Voy a asociar el nodo " << i << " con el nodo " << nodo << endl;
  int peso = rand() % PESO_MAX;
  asociar(i, nodo,  peso);
}

}
\end{lstlisting}

A continuacion mostrmaos el gráfico resultante de la ejecución del programa al incrementar la cantidad de nodos del
árbol:

\begin{figure}[H]
\centering
\includegraphics{imgs/ej2_1000_10_100.pdf}
\caption{Test Performance: Tiempo(s) vs Cantidad de Servidores}
\end{figure}

Notar que los tiempos de ejecución se incrementan linealmente en función de la cantidad de servidores.
 
\subsection{Preguntas Adicionales}

\textbf{1 - }  \emph{Mostrar con un contraejemplo que es posible resolver las dos partes por separado de manera óptima pero que aun así haya una solución en la que la replicación termine en menos tiempo. Comentar posibles soluciones al problema.}\\

Podemos ver el siguiente ejemplo, supongamos que tenemos el grafo: 

\begin{figure}[H]
\centering
\def\svgwidth{140 pt}
\input{imgs/ejemploEj2_b.pdf_tex}
\end{figure}

Se puede ver que tenemos dos árboles generadores mínimos. 

\begin{figure}[H]
\centering
\def\svgwidth{200 pt}
\input{imgs/ejemploEj2_b2.pdf_tex}
\end{figure}

Nuestro procedimiento devuelve el árbol de la izquierda. No obstante, el de la derecha tiene un tiempo de broadcast mejor más chico.  

\textbf{2 - }  \emph{¿Cómo se debe modificar la solución si en lugar de transmitir por broadcast se lo hace por multicast, es decir, se debe mandar un paquete a cada destino, sin hacer copias?}\\

Habría que modificar el ejercicio 2-B únicamente. En la versión del problema que funciona por broadcast nos importa buscar el nodo que garantice que el tiempo de broadcast total sea el menor posible. Si lo hacemos por multicast entonces tenemos que enviar el dato a todos los nodos, una vez a cada uno. Esto significa recorrer todos los caminos del árbol posibles para llegar hasta cada nodo. Deberíamos modificar la solución del ejercicio para que esta recorra todos los caminos hasta todos los nodos posibles, lo cual podemos hacer con una versión modificada de DFS, que guarde todos los caminos que vamos recorriendo para llegar a cada nodo.


\newpage
\section{Ejercicio 3}

\subsection{Interpretación del enunciado}
\par{En el piso de un museo se desean instalar sensores l\'aser de seguridad. Cada sensor cubre una determinada cantidad de metros cuadrados del piso. Existen dos tipos de sensores, los sensores bidireccionales de \$4.000 que emiten dos l\'aseres en direcciones opuestas y los sensores cuatridireccionales de \$6.000 que emiten cuatro l\'aseres formando un \'angulo recto entre cada par consecutivo de l\'aseres, es decir, forman una cruz. Cada sensor cubre la posici\'on (x,y) del piso sobre
la que est\'a situado adem\'as de todas las posiciones sobre las que incidan sus l\'aseres. Los l\'aseres bidireccionales pueden ser orientados horizontal o verticalmente. La emisi\'on del l\'aser solo se detiene al alcanzar una pared. Adem\'as de las paredes que delimitan el piso, existen otras paredes dentro del mismo. Los pisos son rectangulares con un ancho y alto determinados. En cada posici\'on (x,y) del piso ($x<ancho$ y $y<alto$) puede haber una pared, un lugar vac\'io o un sensor. No puede haber m\'as de un sensor en la misma posci\'on. Tampoco se puede ubicar un sensor de forma que sea alcanzado por un l\'aser de otro sensor.}
\medskip
\par{Se quiere ubicar alguna cantidad de sensores de forma que toda posici\'on del piso sea vigilada por al menos un sensor. Una posici\'on se considera vigilada si es alcanzada por el l\'aser de alg\'un sensor. Existen adem\'as ciertas posiciones importantes, las cuales deben ser vigiladas por dos sensores distintos. Dado lo costosos que son los sensores, se pide tambi\'en encontrar la forma m\'as barata de vigilar todo el piso. El objetivo de este ejercicio es desarrollar un algoritmo que, dado un piso del museo, determine la forma menos costosa de vigilar cada posici\'on con al menos un sensor (dos las importantes) sin superponer sensores ni que estos se vigilen entre s\'i. Si bien no hay restricci\'on a la complejidad, se pide que el algoritmo desarrollado utilice la t\'ecnica de $backtracking$, y que se implementen podas para reducir el tiempo de ejecuci\'on.}

\subsection{Resolución}
\par{El piso viene representado como una matriz de enteros de dimensiones conocidas. Cada casillero de la matriz se corresponde a una posici\'on del piso y contiene un 0 si hay una pared en dicha posici\'on del piso, un 1 si es un espacio simple (o vac\'io) o un 2 si el espacio es importante. El algoritmo desarrollado realiza backtracking iterativo sobre los casilleros en los cuales se pueden ubicar los sensores, es decir los casilleros en principio vac\'ios. Dado que un casillero importante debe ser vigilado por dos sensores distintos, no tiene sentido ubicar sensores en tales posiciones, ya que tendr\'ia que ubicarse otro sensor en otra posici\'on que vigile esa casilla, pero esto provocar\'ia que un sensor vigile a otro, lo cual est\'a prohibido. Ignorar los casilleros importantes como posibles ubicaciones para los sensores es una primera poda.}
\medskip
\par{Para modelar el problema se tomar\'an los sensores bidireccionales como dos tipos de sensores distintos (aunque con el mismo precio) si se encuentran orientados horizontamente que si se encuentran orientados verticalmente. Entonces se definen tres nuevos posibles valores para cada casillero (uno por cada tipo de sensor). Las posibles soluciones del \'arbol de soluciones del algoritmo consisten en, para cada casillero en principio vac\'io, asignarle un valor correspondiente a alguna de las cuatro posibilidades: que siga vac\'io, que se ubique un sensor horizontal, que se ubique un sensor vertical o que se ubique un sensor cuatridireccional. El total de soluciones es 4^{n}, $con $n$ la cantidad de casilleros en principio vac\'ios. Esta $n$ est\'a acotada por el tama\~no de entrada (el producto entre el ancho y el alto del piso), aunque se considerar\'a la complejidad en funci\'on de dicha $n$. La justificaci\'on para tal decisi\'on es que, como recorrer todas las posibilidades tiene complejidad O(4$^n) $(y backtracking b\'asicamente hace eso) un piso muy grande cubierto de paredes salvo por unos pocos casilleros ser\'a mucho m\'as f\'acil de resolver que uno m\'as peque\~no pero con mayor proporci\'on de casilleros vac\'ios sobre paredes.$}
\par{A continuaci\'on se muestra el pseudoc\'odigo correspondiente al backtracking iterativo sobre los casilleros en principio vac\'ios.}

\begin{algorithm}[H]
	\caption{Resolución basada en Backtracking Ejercicio 3}
	\begin{algorithmic}
		\KwData{Piso $piso$}\\
		Piso $mejor$ \longleftarrow piso\\
		costo($mejor$) \longleftarrow costoMaximo + 1\\
		\While{loop}{
			\If {pisoValido(piso) y costo(piso) < costo(mejor)}{
				$mejor$ \longleftarrow piso
			}
			Casilla $c$ \longleftarrow $primeraCasillaVacia$\\
			$overflow$ \longleftarrow $cambiarValor$(c)\\
			\While{$overflow$} {
				\eIf {$ultimaCasilla$(c)} {
					$loop$ = false\\
					break\\
				} {
					$c$ \longleftarrow $casillaSiguienteVacia$(c)\\
					$overflow$ \longleftarrow $cambiarValor$(c)\\
				}
			}
		}
		\eIf {costo(mejor) > costoMaximo} {
			\textbf{return} No hay solución
		} {
			\textbf{return} $mejor$
		}
	\end{algorithmic}
\end{algorithm}

\par{El piso $mejor$ es el mejor piso encontrado hasta ahora, es decir, la distribuci\'on de sensores m\'as barata. El costo de este piso (la cantidad de dinero necesaria para implementarlo) se inicializa con $costoMaximo$ + 1, para que la primera soluci\'on v\'alida que encuentre, cualqueira sea su costo, lo sobreescriba. Si al finalizar no se encuentra ninguna soluci\'on v\'alida, el costo de la mejor soluci\'on seguir\'a siendo mayor a $costoMaximo$, lo que se utiliza para saber si se debe retornar una soluci\'on, o si no se encontr\'o ninguna (l\'ineas 15 a 18). Al empezar cada iteraci\'on del ciclo principal (l\'inea 3), $piso$ contiene una de las 4^{n} $ soluciones. Lo primero que se hace es verificar si dicha soluci\'on es v\'alida y mejor (m\'as barata) que la \'optima conseguida hasta el momento (l\'ineas 4 y 5). La funci\'on $pisoValido$ recorre todos los casilleros del piso y verifica lo siguiente.$}

\begin{itemize}
	\item Si el casillero est\'a vac\'io, verifica que al menos un sensor vigile su posici\'on.
	\item Si el casillero contiene un sensor, verifica que ning\'un otro sensor vigile su posici\'on.
	\item Si el casillero es importante, verifica que 2 sensores vigiles su posici\'on.
\end{itemize}

\par{La verificaci\'on de cada casillero consiste en recorrer, en el peor caso, toda la fila y toda la columna del casillero, dando una complejidad de O($w$+$h$) con $w$ el ancho y $h$ el alto del piso. Luego la complejidad de la funci\'on $pisoValido$ tiene una complejidad de O($w$*$h$*($w$+$h$)). El resto del ciclo (l\'ineas 6 a 14) cambian el valor de al menos un casillero, obteniendo otra soluci\'on para la siguiente iteraci\'on. La funci\'on $cambiarValor$ toma una casilla y le cambia el valor. El valor puede ser VACIO (no contiene ning\'un sensor), SENSORH (contiene un sensor horizontal), SENSROV (contiene un sensor vertical) o SENSOR4 (contiene un sensor cuatridireccional) y los recorre en ese orden. Cuando se cambia el valor de SENSOR4 a VACIO orta vez, la funci\'on devuelve $true$ para notificar que ya se han probado todas las combinaciones de la casilla, y ahora se debe cambiar el valor de otra casilla. Cuando $cambiarValor$ devuelve $true$ con la \'ultima casilla, significa que ya se han probado todas las combinaciones y el algoritmo debe terminar (l\'ineas 9 a 11). Tanto $primeraCasillaVacia$ como $casillaSiguienteVacia$ recorren la lista de las casillas que en principio est\'an vac\'ias, es decir, incluyendo a las que en ese momento contengan sensores.}
\medskip
\par{Si bien pareciera que el algoritmo funciona a fuerza bruta (probando todas las soluciones posibles), recorre el \'arbol de soluciones al igual que el m\'etodo de backtracking recursivo. Si se considera cada posible soluci\'on como el conjunto de los casilleros vac\'ios donde cada uno puede tomar uno de cuatro valores, entonces cada soluci\'on se puede representar con un n\'umero de $n$ d\'igitos en base 4. Si a su vez se representan en el \'arbol de soluciones de backtracking, se puede establecer una relaci\'on uno a uno entre cada hoja del \'arbol y cada n\'umero en base 4. De esta forma se justifica que el algorimto desarrollado recorre las mismas soluciones que el m\'etodo de backtracking recursivo.}

\subsection*{Podas}
\par{Sin podas, este algoritmo recorre las 4^{n} $ posibilidades. Una primera poda que se implement\'o, adem\'as de la poda trivial de no poner sensores en los lugares importantes, es evaluar, antes de comenzar el ciclo, si alg\'un lugar importante est\'a posicionado de forma tal que no pueda haber 2 sensores vigil\'ando su posici\'on. Como ya se mencion\'o, cada lugar importante debe ser vigilado por un sensor en su misma fila y otro en su misma columna. Entonces, si un lugar importante se encuntra de forma tal que no puede haber ning\'un sensor en su misma fila o en su misma columna, la instancia no tiene soluci\'on posible. La implementaci\'on de esta poda consiste en recorrer cada casillero importante y, de forma similar a la que se verificaba que un casillero sea valido en $pisoValiso$, recorrer su fila y su columna. Si no se encuentra un casillero vac\'io antes de toparse con una pared, en alguna direcci\'on, resulta que la instancia no tiene posible soluci\'on.$}
\medskip
\par{Otra poda implementada consiste en determinar la validez de cada casillero en el momento en que se le asigna determinado valor para reducir la complejidad de $pisoValido$. Por ejemplo, cuando $cambiarValor$ le asigna a una casilla el valor SENSORH (coloca en su posici\'on un sensor horizontal) se asegura que este sensor no est\'e vigilando a otro sensor recorriendo la fila de la casilla. Si es as\'i, intenta asignarle otro valor, en este caso SENSORV, hasta encontrar alguno v\'alido. Esto incrementa la complejidad de $cambiarValor$ ya que debe verificar que el valor que se est\'a asignando es v\'alido. Pero al asegurar que solo instancias v\'alidas (sin sensores apunt\'andose entre s\'i) llegar\'an al principio del ciclo, $pisoValido$, solo debe evaluar que los casilleros sin sensores sean vigilados por alg\'un sensor y que los casilleros importantes sean vigilados por 2 sensores.}
\medskip
\par{Una poda similar consiste en que al entrar en la funci\'on $cambiarValor$, si la casilla es vigilada por otro sensor, no vale la pena probar ubicar sensores en ella. En estos casos se retorna $true$ como notificaci\'on de que ya se han evaluado las cuatro posibilidades, aunque no sea as\'i. Finalmente se implement\'o otra poda para que, si la soluci\'on parcial que se est\'a construyendo es m\'as costosa que la soluci\'on m\'as barata encontrada hasta el momento, se ``retroceda''. Es decir, se dejen vac\'ios todos los casilleros hasta volver a alcanzar un costo temporal menor al de la soluci\'on m\'as barata hasta el momento. Es correcto realizar esta poda debido a que el costo de una soluci\'on parcial (que necesita m\'as sensores para ser una soluci\'on v\'alida) no puede reducirse cuando se alcance una soluci\'on definitiva (cuando termine de agregar los sensores necesarios) ya que agregar sensores s\'olo puede incrementar el costo final.}
\medskip
\par{Las podas implementadas reducen el espacio de soluciones de la misma forma en este algoritmo iterativo como en un backtracking recursivo. Por ejemplo, sea la soluci\'on $A$ = $a_{1}$, $a_{2}$, ... $a_{i-1}$, $a_i$, ... $a_n$, con $a_{j}$ = VACIO $\forall$ j $\in$ (i..n]. Si $a_i$ es VACIO y se va a evaluar asignarle SENSORH (ubicar un sensor horizontal en su casilla), primero se determina mediante la segunda poda enunciada, si tiene sentido hacerlo. Si no es as\'i (si un sensor horizontal en esa posici\'on vigilar\'ia a otro sensor en la misma fila) se deben podar todas las soluciones con el prefijo $A_{1..i}$ = $a_1$, $a_2$, ... $a_{i-1}$, SENSORH. La t\'ecnica de backtracking recursivo realizar\'ia esta poda al recorrer el nodo en el nivel $i$, antecesor de la hoja correspondiente a la soluci\'on A. Dicho nodo es antecesor de todas las hojas cuyas soluciones asociadas tienen el prefijo $A_{1..i}$, entonces al dejar de recorrer esa ramificaci\'on, se evitan evaluar todas esas soluciones. El algoritmo desarrollado con backtracking iterativo realiza la misma poda cuando itera sobre la soluci\'on A. Al comprobar que no tiene sentido asignarle SENSORH a $a_i$, le asigna en su lugar SENSORV, ignorando as\'i todas las soluciones con el prefijo $A_{1..i}$.}

\newpage
\subsection{Complejidad}
\par{A\'un aplicando las podas, la complejidad del algoritmo sigue siendo $O(4^{n})$ ya que en el peor caso, ninguna de las cotas reducir\'ia el conjunto de soluciones y todas ellas deber\'ian evaluarse. Sin embargo, esperamos que en la mayor\'ia de los casos se vea reducido el tiempo total de ejecuci\'on gracias a las podas (es muy com\'un que en un piso haya 2 casilleros vac\'ios adyacentes, caso en el que no se evaluari\'a poner sensores cuatridireccionaoles en ambos). La complejidad de $pisoValido$ sigue siendo O($w$*$h$*($w$+$h$)), ya que debe recorrer todos los casilleros y, para los vac\'ios y los importantes (en la primera iteraci\'on son todos menos las paredes), recorrer una fila y una columna. La funci\'on $cambiarValor$ tambi\'en debe recorrer una fila y una columna pero s\'olo para un casillero (aunque podr\'ia tener que hacerlo 4 veces), por lo que su complejidad es O(4*$w$*$h$). La complejidad final del algoritmo resulta entonces:

\begin{equation}
T(n) = O(4^n*(w*h*(w+h)+4*w*h) = O(4^n*w*h*(w+h+4)) 
\end{equation}

Si acotamos $n$ por $w*h$ y llamamos $s$ a $w*h$, nos queda:

\begin{equation}
T(s) = O(4^s*(ws+hs+4s) = O(4^s*(2s^2+4s)) = O(4^s*s^2)
\end{equation}

Con s la cantidad de casilleros del piso.

\subsection{Implementaci\'on}
\par{Cada piso se representa con una matriz de enteros, donde cada entero representa el contenido de la posici\'on en el piso. Se guardan dos pisos, el $pisoMejor$ (almacena el mejor encontrado hasta el momento) y el $pisoActual$ (el cual se modifica en cada iteraci\'on). La funci\'on $cambiarValor$ toma en un arreglo los casilleros vac\'ios y suma 1 al n\'umero en base 4 que representa a la soluci\'on de $pisoActual$. Ese cambio se ve reflejado en el piso. La funci\'on del pseudoc\'odigo $pisoValido$ se reemplaz\'o en el c\'odigo por $evaluarPiso$ la cual recorre el $pisoActual$ y, si es mejor que el $pisoMejor$, pasa a ser el nuevo $pisoMejor$. Gracias a las podas se sabe que el $pisoActual$ que recorre $evaluarPiso$ es v\'alido en el sentido que no hay sensores apunt\'andose entre s\'i, por lo que s\'olo se asegura que todos los casilleros est\'en siendo vigilados por al menos un sensor y que los casilleros importantes est\'en siendo vigilados por 2 sensores.}
\medskip
\par{La primera poda se ejecuta antes de comenzar el ciclo principal. Para cada casillero importante, se recorren su fila y su columna en busca de alguna posici\'on vac\'ia. Si para alguno no se encuentran, la funci\'on principal retorna sin entrar al ciclo. Las otras podas se implementan en $cambiarValor$ ya que act\'uan al momento de asignar valores a los casilleros. Para determinar la validez de los sensores en los casilleros se implementaron las funciones $vigila(Posicion$ $p)$ que devuelve si un sensor en la posici\'on $p$ est\'a vigilando a otro sensor, $vigilada(Posicion$ $p)$ que devuelve si alg\'un sensor est\'a vigilando la posici\'on $p$ y $doble\_vigilada(Posicion$ $p)$ que devuelve si 2 sensores est\'an vigilando la posici\'on $p$ (utilizada por $evaluarPiso$ para determinar si todas las posiciones importantes est\'an siendo vigiladas por 2 sensores). Estas tres funciones recorren la fila y la columna de la posici\'on $p$ en busca de sensores.}

\newpage
\subsection{Demostraci\'on de Correctitud}
\par{
Para demostrar que es correcto el algoritmo, verificaremos que se cumplen las siguientes propiedades:

\begin{itemize}

	\item El algoritmo sin podas, se fija exhaustivamente (con fuerza bruta) todas las posiblidades, es decir
en las n casillas libres cubriendo todas las posibilidades: que siga vacio, o haya cualquier tipo de sensor. Luego, si hay solución la encuetra y si no, devuelve que no hay solución.
 
	\item Todas las podas son válidas, es decir se saltean casos que no llegan a la solución, o en caso de ser una posible solución 
(que todos los casilleros esten vigilados) seguro no es optima (no es la solución más barata).

\end{itemize}

\Par{Se descarta la posibilidad de instalar sensores en los lugares importantes, ya que si hay un sensor en un casillero importante necesita que lo apunten dos sensores y estos sensores estarían apuntando al sensor instalado, cosa que prohibe el enunciado.
Luego solo es posible agregar sensores en los casilleros vacíos.}

\Par{Podemos pensar a la solución como un número con dígitos $a_0 a_1 ... a_i ... a_n$, en los cuales $(\forall 1 \le i\le n) \quad 0 \le a_i \le 3$.
Donde 0 representa a VACIO , 1 a SENSORH, 2 SENSORV y 3 a SENSOR4.
Cada dígito indica que hay en el casillero i que (en el inicio) estaba vacío.

Veamos que el algoritmo lo que esta haciendo es asignar de izquierda a derecha los números siempre en orden creciente (desde el 0 hasta el 3).
Luego, si no hay ninguna poda, verifica todos los candidatos a solución ($4^n$ candidatos).

Demostración por Inducción (con el algortitmo sin podas):


Hipotesis inductiva: el algoritmo recorre todas las posibles soluciones con n casilleros vacios.
Luego, queremos ver que si tenemos un museo con n+1 casilleros vacios, el algoritmo recorre todas las posibles soluciones.

Caso Base:
El algoritmo cambia el valor de la última y única casilla con las 4 posiblidades (vacío o los tres tipos de sensores). Es decir, tras probar las 4 posibilidades,
 cambiar$\_$valor asigna a la variable $overflow = true $ y entra al ciclo (de $while (overflow)$) y como es la última casilla, asigna loop = false e interrumpe el ciclo. 

Paso inductivo:
\begin{equation}	
	$el algoritmo recorre todas las$  $4^n$ posibles soluciones con n casilleros vacios  $\Rightarrow$ \\
	el algoritmo recorre todas las posibles soluciones con n+1 casilleros vacios. 
\end{equation}

Sea el piso que tiene n+1 casilleros vacíos. Sea una posible solucion el número $A$ = $a_{1}$, $a_{2}$, ... , $a_i$, ... $a_n$ $a_{n+1}$, con $a_i$ un dígito que indica
los 4 posibles estados del casillero. Por la hipótesis inductiva, sabemos que el algoritmo simula los $4^n$ prefijos. Para cada uno de los $4^n$ prefijos el algoritmo 
prueba las 4 alternativas para la última casilla. Por lo tanto prueba las $4^{n+1}$ posibles soluciones.

Esto es porque donde hubiese terminado el ciclo con n casillas, con n+1, $ultimaCasilla$ devuelve false, y se obtiene una casilla más (siendo esta la última de las n+1)
a la que se le cambia una vez el valor y se vuelven a evaluar las $4^n$ posibles soluciones del prefijo que se tenía.
Esto se repite para las 4 opciones que puede tomar la última casilla. 


}

\medskip
\\Ahora demostraremos que las podas son válidas:

\begin{itemize}
\item \par{Alguna más barata : si ya hay una solución con un costo menor al costoMaximo, y se tiene unsa solución parcial con un mayor costo, esta
solución se descarta al igual que toda solución con ese prefijo. 


Demostración: Si $costo(soluciónOptimaDelMomento) \le costoActual(soluciónParcial)$, la solución que tenga como prefijo a la solución parcial, va a tener más o
igual cantidad de sensores, logrando que cualquiera solución con ese prefijo va a ser más cara que la solucionOptimaDelMomento. }

\medskip

\item \par{Si la casilla i está siendo vigilada, ningún número con un sensor(de cualquier tipo) en esa casilla puede ser solución:


Demostración: Cualquier solución que tenga a una casilla i que este siendo vigilada y se le agrega un sensor, este sensor va a tener una señal que impacte al dispositivo agregado, escenario que no puede ser posible por la prohibición del enunciado.
}
\medskip
\item \par{Si una casilla importante se encuntra de forma tal que no puede haber ningún sensor en su misma fila o en su misma
columna, la instancia no tiene solución posible:


Demostración: Sea la casilla i, una casilla importante donde los casilleros que tiene en su misma fila y columnas(hasta llegar a una pared)
son PARED o IMPORTANTE. En este escenario no es posible colocar sensores en las paredes ya que solo se los puede colocar en casilleros libres.
Tampoco se puede colocar en un casillero IMPORTANTE debido a que se apuntarían mutuamente.
}


}

\newpage
\subsection{Testing}
\textbf{Correctitud}\\

Para la realización de los tests de correctitud no haremos muchas variaciones en el tamaño del piso del museo. Consideramos que este aspecto no es muy relevante a la hora de buscar casos bordes.\\

\noindent\textbf{Test$\#$1:} Piso del museo compuesto enteramente por casilleros libres.
\begin{figure}[H]
\centering
\def\svgwidth{140 pt}
\input{imgs/ej3_0.pdf_tex}
\end{figure}
\noindent\textbf{Status:} OK. Se cubren todos los casilleros.\\

\noindent\textbf{Test$\#$2:} Piso del museo compuesto enteramente por paredes.
\begin{figure}[H]
\centering
\def\svgwidth{140 pt}
\input{imgs/ej3_1.pdf_tex}
\end{figure}
\noindent\textbf{Status:} OK. No es necesario ni posible colocar ningun sensor.\\

\noindent\textbf{Test$\#$3:} Todos los casilleros son importantes y necesitan ser cubiertos con 2 sensores.
\begin{figure}[H]
\centering
\def\svgwidth{140 pt}
\input{imgs/ej3_2.pdf_tex}
\end{figure}
\noindent\textbf{Status:} OK. No existe solución\\

\noindent\textbf{Test$\#$4:} Piso cuadriculado.
\begin{figure}[H]
\centering
\def\svgwidth{140 pt}
\input{imgs/ej3_3.pdf_tex}
\end{figure}
\noindent\textbf{Status:} OK. No hay otra posibilidad más que colocar un sensor, vertical u horizontal en cada espacio libre.\\

\newpage
\noindent\textbf{Test$\#$5:} Casilleros importantes en esquinas.
\begin{figure}[H]
\centering
\def\svgwidth{140 pt}
\input{imgs/ej3_4.pdf_tex}
\end{figure}
\noindent\textbf{Status:} OK.\\

\noindent\textbf{Test$\#$6:} Casilleros importantes en el centro.\\
\begin{figure}[H]
\centering
\def\svgwidth{140 pt}
\input{imgs/ej3_5.pdf_tex}
\end{figure}
\noindent\textbf{Status:} OK.\\

\noindent\textbf{Test$\#$7:} Tablero de una sola fila.\\
\begin{figure}[H]
\centering
\def\svgwidth{140 pt}
\input{imgs/ej3_6.pdf_tex}
\end{figure}
\noindent\textbf{Status:} OK.\\

\noindent\textbf{Test$\#$8:} Tablero de una sola columna.\\
\begin{figure}[H]
\centering
\def\svgwidth{140 pt}
\input{imgs/ej3_7.pdf_tex}
\end{figure}
\noindent\textbf{Status:} OK.\\


\newpage
\textbf{Performance}\\
\par{Para este ejercicio, los casos evaluados se realizaron en un conjunto m\'as reducido de instancias, dada la elevada complejidad de la soluci\'on propuesta. Otra particularidad del problema es que la complejidad del algoritmo no crece en funci\'on del tama\~no de entrada (la cantidad de casilleros del piso), aunque s\'i en funci\'on de algo que puede ser acotado por \'el (la cantidad de casilleros en principio vac\'ios). Por ello se decidi\'o generar dos conjuntos de instancias. Primero se generaron instancias en las que cada una se corresponde con una cantidad de casilleros en principio vac\'ios. Luego se generaron instancias en las que cada una se corresponde con el tama\~no del piso. Para el primer conjunto, dada una cantidad $n$ de casilleros vac\'ios se genera un piso de $h$ filas y $w$ columnas con $h = \sqrt{n}$ + 1} y $w = n / \sqrt{n} +1$. Luego se determinaron al azar paredes y posiciones importantes para dejar exactamente $n$ casilleros vac\'ios. Para el segundo conjunto, dado un tama\~no $n$ se genera un piso de $h$ filas y $w$ columnas tal que $h*w = n$. Una vez generados los conjuntos, se ejecutaron ambos y se graficaron los resultados a continuaci\'on.}
\par{El siguiente gr\'afico muestra los tiempos de ejecuci\'on de las instancias generadas para el primer conjunto. Lo que nos importa ver es c\'omo se comporta el algoritmo en funci\'on de la cantidad de casilleros vac\'ios por lo que se grafic\'o en funci\'on de tal. Las series verdes corresponden a las mediciones del algoritmo sin podas, que recorre todas las posibles soluciones, mientras que las series rojas corresponden a la versi\'on del algoritmo que implementa podas. Para cada $n$ cantidad de casilleros vac\'ios, se ejecutaron 50 instancias distintas. Los cuadrados rojos y los c\'irculos verdes son los promedios de las 50 ejecuciones para el algoritmo con y sin podas respectivamente. Cada uno de estos viene acompa\~nado con intervalos verticales que denotan la varianza de cada medici\'on. Tambi\'en se graficaron con cruces rojas y verdes las mediciones m\'aximas y m\'inimas para cada $n$. La curva azul es una funci\'on del tipo $4^n$. Notar que la escala del eje vertical no es lineal sino logar\'itmica.}
\begin{figure}[H]
\centering
\def\svgwidth{140 pt}
\includegraphics{../codigo/ej3/tests/ej3a.pdf}
\caption{Resultados de las mediciones obtenidas por el script $ej3\_test$ con los par\'ametros 35 1 50 para instancias generadas en funci\'on de la cantidad de casilleros vac\'ios.}
\end{figure}
\par{Lo primero que se puede apreciar en este gr\'afico, es que la complejidad del algoritmo sin podas se ajusta bastante bien a la cota $4^n$ con $n$ la cantidad de casilleros. Si bien esa funci\'on tambi\'en acota los tiempos de ejecuci\'on del algoritmo con podas, estos son considerablemente menores y la diferencia se agranda a medida que aumenta el $n$. La complejidad del algoritmo sin podas parece estabilizarse a medida que se incrementa el tama\~no de entrada (recordar que el tama\~no de entrada fue generado para ser directamente proporcional a la cantidad de casilleros vac\'ios). Esto se deduce al ver como su varianza tiende a achicarse. El algoritmo con podas, por otro lado, parece tener intervalos en los que su complejidad no aumenta demasiado pero luego tiene saltos abruptos. Estos saltos se han identificado en las posiciones correspondientes a valores de $n = k^2$, para $k$ entero (En principio se hab\'ia ejecutado tambi\'en para $n = 36$ pero debido a su excesivo tiempo de procesamiento, se detuvo la ejecuci\'on). Estos valores tienen la particularidad de generar matrices cuadradas (de $\sqrt{n}+1$ x $\sqrt{n}+1$ casilleros). Es tambi\'en en estos saltos en los que se aprecia una mayor varianza, la cual parece disminu\'ir hasta el siguiente salto. Para la mayor\'ia de los valores de $n$, el valor m\'inimo no se ve en el gr\'afico debido a que su tiempo de ejecuci\'on fue tan bajo que se redonde\'o a $0$. Probablemente sean casos en los que, por la primera poda (evaluar si existe una posici\'on importante no vigilable) no se ejecut\'o ninguna iteraci\'on del ciclo principal.}
\par{El siguiente gr\'afico muestra los tiempos de ejecuci\'on de las instancias generadas para el segundo conjunto, es decir en funci\'on del tama\~no de entrada. En este caso, cada una de las $n$ posiciones se determin\'o al azar, aunque favoreciendo la designaci\'on de casilleros vac\'ios para que sean estos los que predominen y no las paredes. Una vez m\'as, las series verdes corresponden a las mediciones del algoritmo sin podas y las rojas al algoritmo con podas. La curva azul es exactamente la misma que la graficada en la figura anterior. Servir\'a para comparar la complejidad del algoritmo en funci\'on del tama\~no de entrada con la complejidad del algoritmo en funci\'on de la cantidad de casilleros. La varianza, los m\'aximos y los m\'inimos fueron graficados de la misma forma que la figura anterior. Tambi\'en con el objetivo de comparar ambas opciones, se mantuvieron las escalas en los ejes.}
\begin{figure}[H]
\centering
\def\svgwidth{140 pt}
\includegraphics{../codigo/ej3/tests/ej3b.pdf}
\caption{Resultados de las mediciones obtenidas por el script $ej3\_test$ con los par\'ametros 35 1 50 para instancias generadas en funci\'on de la cantidad de casilleros totales.}
\end{figure}
\par{En principio se observa que ambas series est\'an acotadas por la funci\'on que se ajustaba a la complejidad del algoritmo sin podas en funci\'on de la cantidad de casilleros vac\'ios. Incluso el algoritmo sin podas parece crecer con menor complejidad, aunque podr\'ia deberse a que el tama\~no de sus instancias incrementa en 1 para cada valor, mientras que en el caso anterior, incrementar en 1 el valor de $n$ pod\'ia incrementar en m\'as de 1 el tama\'no del piso. Es notable la gran varianza de ambos algoritmos, as\'i como el hecho que que tampoco se graficaron los valores m\'inimos. Todo esto puede ser atribu\'ido a que, dado que no se fijaron la cantidad de casilleros vac\'ios, estos podr\'ian haber sido muy pocos, haciendo muy peque\~no el conjunto de soluciones posibles.}


\newpage
\section{Conclusiones}

 
\end{enumerate}
 En primera instancia, la realización del primer ejercicio nos permitió familiarizarnos con el proceso de desarrollo de una solución utilizando la técnica de programación dinámica:
\begin{enumerate}
 \item Caracterizamos la estructura de una solución óptima.
 \item Definimos el valor de la solución óptima recursivamente.
 \item Computamos el valor de la solución óptima, con una estrategia bottom-up .
 \item Construimos la solución óptima con la información computada.
\end{enumerate}
 
Por otro lado, los ejercicios 2 y 3 nos permitieron acercarnos de manera introductoria al modelado de problemas utilizando conceptos de teoría de grafos. El proceso de desarrollo nos impuso la necesidad de contemplar diferentes modelos para cada problema y analizar cuales de estos representaban el problema de manera correcta. 

Luego, utilizando la bibliografía sugerída por la materia y las clases teóricas de la misma, pudimos contemplar diferentes algoritmos y terminamos decidiendo un curso de implementación que en principio pareciera satisfactorio. Durante el proceso de desarrollo los algoritmos implementados fueron cambiados en varios puntos para hacerlos más compatibles con el problema en si. 

A su vez, tuvimos que investigar tanto diferentes opciones de estructuras de datos a la hora de implementar grafos, como diferentes estructuras auxiliares para los algoritmos. En algunos casos incluso realizamos una serie de tweaks a las estructuras empleadas para adaptarlas a nuestros algoritmos y que estos cumplieran bien las cotas de complejidad impuestas por el enunciado.

Por el costado formal, las justificaciones de nuestras soluciones nos sirvieron para complementar las ejercitaciones de la práctica de la materia; contemplando diferentes técnicas de demostración para verificar la correctitud de nuestras implementaciones.

En resumen, creemos que el desarrollo de este trabajo práctico cumplió un rol fundamental dentro del programa de la materia y fue un buen complemento para las clases teóricas y prácticas de la misma.

\newpage
\section{Informe de modificaciones para la reentrega}

\subsection{Pautas de Implementación}
\textbf{Clase GrafoEx}
\par{A pesar de no haberlo mencionado en la primera entrega, se desarrollaron
en realidad dos clases basadas en grafos. La primera, definida en $grafo.h$
es un grafo estándar utilizado para el ejercicio 2. La segunda, definida en
$grafoEx.h$ representa un grafo con nodos diferenciables, utilizado para el
ejercico 3. La mayor parte de las funciones son idénticas en ambas clases,
aunque la función para generar una AGM a partir del algoritmo de $Kruskal$,
en el caso de grafoEx, en lugar de generar un árbol, genera un bosque con
un nodo diferenciable en cada componente conexa. Para representar la provincia
del ejercicio 3, representamos las fábricas con los nodos diferenciables y
los clientes con los nodos no diferenciables. Además, $grafoEx$ tiene una
variante del algoritmo de $Kruskal$, llamada $Kruskal\_no\_res$ la cual
hace lo mismo que la función de $Kruskal$ pero no devuelve el resultado.}

\textbf{Disjoint Set}
\par{Otra clase desarrollada que tampoco fue mencionada en la primera entrega
es la implementada en $disjointSet.h$. Esta clase se basa en...}

\textbf{Tests corregidos}
\par{En la primera entrega los tiempos medidos con $test.cpp$ fueron muy poco
precisos debido a que, al medir el tiempo de ejecución de cada $resolver\_x$
estábamos incluyendo el código correspondiente a inicializar las estructuras de
los resultados y, en algunos casos copiarlas. Para resolverlo, modificamos
el archivo $tests.cpp$ para que cada función $testear\_ejx$, en lugar de hacer
una llamada a $resolver\_ejx$, ejecute las mismas funciones que esta última
pero sin guardarse ni devolver los resultados obtenidos. En el ejercicio 3
se utilizó la función $Kruskal\_no\_res$ en lugar de $Kruskal$ con este
objetivo. Como se editó el archivo $tests.cpp$, se rehicieron todos los
gráficos.}
\par{Para mayor claridad, la función $generar\_instancia\_ej2$ fue reemplazada
por $generar\_instancia\_ej2b$, ya que en realidad, esta función generaba una
instancia de entrada para la parte $b$ del ejercicio, es decir un árbol.
Asimismo, la función $generar\_aristas\_aleatorias$ de $grafo.h$ genera en
realidad un árbol para ser utilizado como entrada del ejercicio. En el caso del
$grafoEx$, $generar\_aristas\_aleatorias$ genera un grafo no necesariamente
conexo, aunque sí asegurando que todo cliente está en una componente conexa
en la que hay al menos una fábrica. Luego, para cada arista posible, elige
al azar (con probabilidad 0.5) si agregarla o no al grafo; Por lo que la
cantidad de aristas de entrada para el ejercico 3 es, en promedio $n.(n-1)$ / 4.}

\textbf{Código}
\par{Además de algunos cambios menores en el código, se quitaron todas las
utilizaciones de $new$ y ahora todos los parámetros se pasan por referencia,
en lugar de hacerse por copia. También cambiamos el $Makefile$ quitándo el
flag -O2 para obtener mediciones más estables.}
\subsection{Ejercicio 1}
\textbf{Testing de performance}
\par{Se rehizo el gráfico en función de los cambios realizados en $tests.cpp$.
También ampliamos el análisis del mismo.}
\subsection{Ejercicio 2 - A}
\textbf{Resolución}
\par{Se agregó a la explicación de la resolución del ejercicio una idea del
funcionamiento del algoritmo de $Kruskal$}

\textbf{Demostración de correctitud}
\par{Se agregó la sección faltante en la primera entrega en la que demostramos
la correctitud del ejercico a partir de la demostración de correctitud del
algoritmo de $Kruskal$}

\textbf{Cota de Complejidad}
\par{Se agregó la sección faltante en la primera entrega en la que demostramos
la cota de complejidad del ejercico a partir de la complejidad del
algoritmo de $Kruskal$}

\textbf{Implementación}
\par{Se reescribió esta sección profundizando en la implementación del
algoritmo de $Kruskal$ y de la clase $disjointSet$.}

\subsection{Ejercicio 2 - B}
\textbf{Resolución}
\par{Se agregó una idea del funcionamiento de los algoritmos de recorrido de
grafos, $BFS$ y $DFS$ utilizados para la resolución del problema.}

\textbf{Implementación}
\par{Se reescribió esta sección profundizando en la implementación de los
algoritmos de $BFS$ y $DFS$ utilizados.}

\textbf{Testing de performance}
\par{Se rehizo el gráfico en función de los cambios realizados en $tests.cpp$.
También ampliamos el análisis del mismo.}
\subsection{Ejercicio 3}
\textbf{Cota de Complejidad}
\par{Se agregó la sección faltante en la primera entrega en la que demostramos
la cota de complejidad del ejercico a partir de la complejidad del
algoritmo de $Kruskal$}

\textbf{Testing de correctitud}
\par{Se rehicieron los gráficos que muestran las instancias pero los tests
son los mismos.}
\end{document}

\textbf{Testing de performance}
\par{Se rehizo el gráfico en función de los cambios realizados en $tests.cpp$.
También ampliamos el análisis del mismo.}
\end{document}
