\section{Introducción}
\par{El presente informe apunta a documentar el proceso de desarrollo del Trabajo
Práctico número 3 de la materia Algoritmos y Estructuras de Datos III, cursada
correspondiente al segundo cuatrimestre del año 2013.}\\

\par{Este trabajo práctico consiste en el análisis de diferentes heurísticas
para tratar el \emph{Problema de la Clique de Máxima Frontera} (\emph{CMF} de
ahora en más). Dado un grafo $G=(V,E)$, una clique de $G$ es un subconjunto $K$
de los nodos de $V$ tal que para todo par de nodos de $K$, existe una arista
en $E$ que los une. Es decir,}
\[
sea\ G = (V,E),\ K \subseteq V\ es\ clique\ de\ G \Leftrightarrow (v,w)
\in E\ \forall\ v,w \in K
\]

\par{En otras palabras, una clique de $G$ es el conjunto de nodos de un
subgrafo completo de $G$. Dado un conjunto de nodos $C$, se define la frontera
de $C$ en $G$ ($\delta_G$($C$)) como la cantidad de aristas que tienen un
extremo en $C$ y el otro en $V\backslash C$. Es decir,}
\[
sean\ G = (V,E),\ y\ C \subseteq V\ un\ subconjunto\ de\ nodos,\ \delta_G(C)
= |{(v,w) \in E / v \in C \land w \notin C}|
\]

\par{El problema de $CMF$ consiste en, dado un grafo $G$, encontrar
una clique $K$ de $G$ tal que la frontera de $K$ sea mayor a la frontera de
cualquier otra clique de $K$, es decir,}
\[
sean\ G = (V,E),\ y\ K \subseteq V\ una\ clique\ de\ G,\ K\ es\ CMF
\Leftrightarrow \delta_G(K) \geq \delta_G(K')\ \forall K'\ clique\ de\ G
\]

%\par{Exhibimos un ejemplo del problema en la siguiente figura:}

%\textbf{GRAFICO$\_$COPADO.PNG}

%\par{Podemos ver como los vértices coloreados con negro corresponden a la
%clique que tiene la frontera mas grande del grafo, mientras que los coloreados
%con gris corresponden a su frontera.}\\

\par{Para resolver este problema se pide implementar:}
\begin{itemize}
\item Un algoritmo exacto.
\item Una heurística constructiva golosa.
\item Una heurística de búsqueda local.
\item Una metaheurística de búsqueda tabú.
\end{itemize} 

\par{Luego de haber terminado con la programación de nuestras soluciones,
vamos a realizar una experimentación con el objetivo de verificar y comparar
tanto los tiempos de ejecución como los resultados de los programas extrayendo
así conclusiones sobre la performance y la optimalidad de los diferentes
métodos propuestos en este trabajo práctico.}\\

\par{Esperamos, al finalizar este proyecto, tener un entendimiento teórico
sobre el análisis heurístico de problemas difíciles de tratar junto las
herramientas prácticas para poder abordar este tipo de problemas en el futuro.}
