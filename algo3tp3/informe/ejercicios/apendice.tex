\section{Apéndice}

\subsection {La clase Grafo}
\begin{lstlisting}
struct arista{
	int nodo1;
	int nodo2;
	
	arista(int n, int m){
		this->nodo1 = n;
		this->nodo2 = m;
	}

};

class grafo
public:
 int cantNodos;
 vector< vector<int> > lista_global;
 vector< vector<int> > matriz_ady;
 vector<arista> aristas;

 //Constructor: Construye un grafo vacío
 grafo(){
  grafo(0);
 }

 //Constructor: Construye un grafo con n nodos sin aristas
 grafo(int n){
  constructor(n);
 }

 //Construye el grafo a partir de la entrada estandar
  // (llamar despues del constructor vacío)
 void inicializar() {
  int n, m, a, b, c;
  cin >> n;
  constructor(n);
  if(n != 0) {
   cin >> m;
   for(int i = 0; i < m; ++i){
    cin >> a;
    cin >> b;
    //Paso de 1..n a 0..n-1::
    a--; b--;
    asociar(a,b);
   }
  }
 }

 //Construye un grafo con aristas aleatorias
  // (llamar despues del constructor basico)
 void generar_aristas_aleatorias() {
  srand(time(NULL));
  for(int i = 0; i < cantNodos; ++i)
  for(int j = i; j < cantNodos; ++j)
   if(rand() \%2 == 0) asociar(i, j);
 }

 int gradoMax(){
  unsigned int gradoMax = 0;
  for (int i = 0; i < cantNodos; ++i ){
   if (lista_global[i].size() > gradoMax) {
    gradoMax = lista_global[i].size();
   }
  }
  return gradoMax;
 }

 int grado(int i) {
  return lista_global[i].size();
 }

 void asociar(int nodo1, int nodo2){

  if(nodo1 != nodo2){

   /* Agrego la arista faltante  */
   this->aristas.push_back(arista(nodo1,nodo2));
   
    //agrego cada nodo en la lista del otro nodo incidente
     
    (this->lista_global[nodo1]).push_back(nodo2);
    (this->lista_global[nodo2]).push_back(nodo1);
  
    (this->matriz_ady[nodo1][nodo2]) = 1;
    (this->matriz_ady[nodo2][nodo1]) = 1;
  }
 }
 
 bool hayArista(int nodo1, int nodo2) const{
  return (matriz_ady[nodo2][nodo1] == 1);
 }

 // Devuelve si el nodo 'nodo' se conecta con la clique 'clique'
 bool se_conecta_con_clique(int nodo, vector<bool> &clique){
  for (int i = 0; i < cantNodos; ++i) {
   if ( (clique[i]) && (matriz_ady[i][nodo]==0) ){
    return false;
   }
  }
  return true;
 }

 // Devuelve si el conjunto 'nodos' representa una clique
 bool es_clique(vector<bool> nodos) {
  vector<int> clique = vector<int>();
  for (int i=0; i<cantNodos; i++) {
   if (nodos[i]) {
    for (int j=0; j<clique.size(); j++) {
     if(matriz_ady[i][clique[j]]==0) return false;
    }
    clique.push_back(i);
   }
  }
  return true;
 }

 // Determina cuanto varía la frontera de la clique 'clique'
  // al agregar el nodo 'nodo'
 int cuanto_aporta(int nodo, Solucion &s){
  int frontera_inicial = s.second;
  alternar(s, nodo);
  int frontera_final = s.second;
  alternar(s, nodo);
  return frontera_final - frontera_inicial;
 }

 // Devuelve el v-esimo vecino de la solucion s
 Solucion obtener_vecino(Solucion &s, Solucion &vAnterior, int v) {
  Solucion vecino;
  int nodos_distintos = 1;//Me dice cuantos nodos debo agregar/sacar
  int vecinos = combinatorio(cantNodos,1);
  while (v >= vecinos) {
   nodos_distintos++;
   v -= vecinos;
   vecinos = combinatorio(cantNodos, nodos_distintos);
  }
  if (v==0) {
   vecino = make_pair(s.first, s.second);
   for (int i=0; i<nodos_distintos; i++) {
    alternar(vecino, i);
   }
  } else {
   vecino = make_pair(vAnterior.first, vAnterior.second);
   if (vecino.second < 0) vecino.second*=-1;
   int ov = 0;
   for (int i=cantNodos-1; i>=0; i--) {
    if (s.first[i]!=vAnterior.first[i]) {
     if (i==cantNodos-1-ov) {
      ov ++;
      alternar(vecino, i);
     } else {
      if (ov>0) {
       alternar(vecino, i);
       alternar(vecino, i+1);
       int j = i+2;
       while(ov>0 && j<cantNodos && vecino.first[j]==s.first[j]) {
        alternar(vecino, j);
        j++; ov--;
       }
       if (ov==0) break;								
      } else {
       alternar(vecino, i);
       alternar(vecino, i+1);
       break;
      }
     }
    }
   }
  }
  if (!es_clique(vecino.first)) vecino.second *= -1;
  return vecino;
 }

 void alternar(Solucion &v, int i) {
  v.first[i] = !v.first[i];
  //Recalcular vecino.second (frontera) en funcion del nodo
  // sacado/agregado (no importa si es o no una clique, todavía)
  if (v.first[i]) {
   for (int j=0; j<lista_global[i].size(); j++) {
    if (v.first[lista_global[i][j]]) {
     v.second--;
    } else v.second++;
   }
  } else {
   for (int j=0; j<lista_global[i].size(); j++) {
    if (v.first[lista_global[i][j]]) {
     v.second++;
    } else v.second--;
   }			
  }
 }
 
private:
 void constructor(int n) {
  //cargo la cantidad de nodos
  this->cantNodos = n;

  vector <int> vec;

  //inicializo todas las listas vacias
  for(int i = 0; i < n; i++){
   (this->lista_global).push_back(vector <int>()) ;
  }

  for(int i = 0; i < n; i++){
   (this->matriz_ady).push_back(vector <int>(n)) ;
  }
 }
\end{lstlisting}
\newpage
\subsection{Algoritmo Exacto}

\begin{lstlisting}
Solucion exacto(grafo &g) {
Solucion optima = make_pair(vector<bool>(g.cantNodos,false), 0);
Solucion actual = make_pair(vector<bool>(g.cantNodos,false), 0);
exacto_recur(g, 0, optima, actual);
return optima;
}

Solucion exacto_sin_poda(grafo &g) {
Solucion optima = make_pair(vector<bool>(g.cantNodos,false), 0);
Solucion actual = make_pair(vector<bool>(g.cantNodos,false), 0);
exacto_recur_sp(g, 0, optima, actual);
return optima;
}

void exacto_recur(grafo &g, int i, Solucion &optima, Solucion &actual) {
if (actual.second > optima.second) {
 optima = actual;
}
if (i==g.cantNodos) {
 return;
}

// si se conecta_con_clique sigo la recursion con y sin ese nodo
if ( g.se_conecta_con_clique(i, actual.first) ) {
 g.alternar(actual, i);
 exacto_recur(g, i+1, optima, actual);
 g.alternar(actual, i);
 exacto_recur(g, i+1, optima, actual);
// si no se conecta con mas nodos tampoco va a ser una clique,
// luego hago la recursion sin ese nodo
}else{
 exacto_recur(g, i+1, optima, actual);
}
}

void exacto_recur_sp(grafo &g, int i, Solucion &optima, Solucion &actual) {
if (actual.second > optima.second) {
 optima = actual;
}
if (i==g.cantNodos) {
 return;
}
g.alternar(actual, i);
exacto_recur_sp(g, i+1, optima, actual);
g.alternar(actual, i);
exacto_recur_sp(g, i+1, optima, actual);
}
\end{lstlisting}

\newpage
\subsection{Heur\'istica Golosa}
\begin{lstlisting}
Solucion goloso(grafo g){
Solucion s = make_pair(vector<bool>(g.cantNodos,false), 0);

while (true){
 int candidato = -1;
 int maximo_aporte = 0;

 // veo cual es el nodo que mas aporta y si aporta algo
  // positivo, lo agrego
 for (int i = 0 ; i < g.cantNodos; ++i){				
  if (!s.first[i]){
   if ( g.se_conecta_con_clique(i, s.first) ){
    int aporta = g.cuanto_aporta(i,s);
    if (aporta > maximo_aporte){
     candidato = i;
     maximo_aporte = aporta;
    }
   }
  }
 }

 if (candidato == -1){
  break;
 } else {
  g.alternar(s, candidato);
 }
}

return s;
}

\end{lstlisting}
\newpage
\subsection{Heur\'istica de búsqueda Local}
\begin{lstlisting}
// Parametros:
// * cant_iter: maxima cantidad de iteraciones sin mejorar
// (se reinicia cada vez que mejora)
// * vecinos_size: tamaño de la vecindad
// (cantidad de nodos distintos (como maximo) de cada vecino)
// La cantidad de vecinos en una vecindad de tamaño V es
 // Sum{i=1..V} (n i)
Solucion local(grafo &g, Solucion &sol_inicial, int vecinos_size) {

Solucion sol_actual = make_pair(sol_inicial.first, sol_inicial.second);
bool maximo_local = false;
int vecindad = cant_vecinos(g.cantNodos, vecinos_size);
while (!maximo_local) {
 Solucion sol_mejor = make_pair(vector<bool>(0), -1);//solucion vacia
 //Recorrer vecindad
 Solucion temporal = make_pair(vector<bool>(0), -1);
 for (int i=0; i<vecindad; i++) {
  temporal = g.obtener_vecino(sol_actual, temporal, i);
  if (temporal.second > sol_mejor.second) {
    sol_mejor = temporal;
  }
 }
 if (sol_mejor.second > sol_actual.second) {
  sol_actual = sol_mejor;
 } else {
  maximo_local = true;
 }
}
return sol_actual;
}

\end{lstlisting}
\newpage
\subsection{Metaheur\'istica de búsqeuda Tab\'u}
\begin{lstlisting}
// Parametros:
// * cant_iter: maxima cantidad de iteraciones sin mejorar
// (se reinicia cada vez que mejora)
// * tabu_size: tamaño de la lista tabu
// (cantidad de soluciones que puede almacenar)
// * vecinos_size: tamaño de la vecindad
// (cantidad de nodos distintos (como maximo) de cada vecino)
// La cantidad de vecinos en una vecindad de tamaño V es
// Sum{i=1..V} (n i)
Solucion tabu(grafo &g, Solucion &sol_inicial, int cant_iter, int tabu_size, int vecinos_size) {

Solucion sol_optima = make_pair(sol_inicial.first, sol_inicial.second);
Solucion sol_actual = make_pair(sol_inicial.first, sol_inicial.second);
Tabu_list lista_tabu(tabu_size);
int iter = 0;
int vecindad = cant_vecinos(g.cantNodos, vecinos_size);
while (iter < cant_iter) {
 //solucion vacia
 Solucion sol_mejor = make_pair(vector<bool>(0), -pow(g.cantNodos,2));
 //solucion vacia
 Solucion sol_mejor_tabu = make_pair(vector<bool>(0), -pow(g.cantNodos,2));
 //Recorrer vecindad
 Solucion temporal = make_pair(vector<bool>(0), -1);
 for (int i=0; i<vecindad; i++) {
  temporal = g.obtener_vecino(sol_actual, temporal, i);
  if (temporal.second > sol_optima.second) {
   sol_optima = temporal;
   if (sol_mejor.second > sol_actual.second) iter = -1;
  }
  if (temporal.second > sol_mejor_tabu.second) {
   sol_mejor_tabu = temporal;
  }
  if ((temporal.second > sol_mejor.second)
   &&(!lista_tabu.es_tabu(temporal.first))) {
    sol_mejor = temporal;
  }
 }
 iter++;
 lista_tabu.add_tabu(sol_actual.first);
 sol_actual = sol_mejor;
 // Si no pude tomar ningun vecino valido, tomo una solucion tabu
 if (sol_actual.second==-pow(g.cantNodos,2)) {
  sol_actual = sol_mejor_tabu;
 }
 // Si tampoco tengo una solucion tabu, me quede estancado
 if (sol_actual.second==-pow(g.cantNodos,2)) {
  iter = cant_iter;
 }
}

	return sol_optima;
}

\end{lstlisting}
