\section{Algoritmo Exacto}

{\color{dGray}\subsection{Implementación del Algoritmo}

\par{Para realizar el algoritmo exacto, utilizamos la técnica de programación de
backtracking. Para ver cuál es el clique con una frontera de mayor
cardinalidad, calculamos la frontera de cada clique que hay en el grafo.
Partimos de una solución en la que el conjunto de nodos es vacío\footnote{{
\color{dGray}Es debatible si esta solución es válida o no, pero al tener
frontera igual a cero, cualquier solución válida con al menos uno de frontera
será mejor. En todo caso, si el grafo no tiene aristas, se devolverá esta
solución vacía.}}. Luego utlizamos recursión para agregar cada posible nodo al
conjunto. La función recursiva se llama a sí misma dos veces: una agregando el
i-ésimo nodo a la solución actual y otra sin agregarlo.}\\

\par{A la función recursiva se le pasan como parámetros la solución actual y
la solución óptima hallada hasta el momento (ambas inicializadas con la
solución vacía). Cada vez que se entra a la función recursiva se evalúa el
tamaño de la frontera de la solución actual y si es mayor a la frontera de
la solución óptima, la solución actual reemplaza a la óptima. Una vez
finalizado el algoritmo se retorna la solución almacenada como óptima. La
solución vacía almacenada como óptima será reemplazada en cuanto se evalúe
cualquier solución con tamaño de frontera mayor a cero. Como cada nodo es una
clique, si el algoritmo retorna la solución vacía significa que el grafo no
tiene aristas.}\\

\par{Con el objetivo de reducir la complejidad del algoritmo exacto se ha
implementado una importante poda. Cuando la función va a llamarse
recursivamente, primero se evalúa si tiene sentido agregar el nodo a la
solución actual, es decir si con el nodo que se agrega, la solución actual
sigue representando una clique. Si con ese nodo no se
forma un clique, luego ese subconjunto de nodos sumado a otros nodos, tampoco
va a ser clique. Por esta razón, se poda toda solución que agregue el nodo
a la solución actual, es decir todo subconjunto de nodos, que no forman una
clique.}}\\

\par{\textbf{Como esta primera poda resultó ser muy poco inteligente, se
implementó una segunda poda basada en la cantidad de nodos de la clique.
Proponemos que ninguna clique máxima puede tener más de $n$/2 nodos o, en otras
palabras, que dada cualquier clique de $k$ nodos ($k$ $>$ $n$/2) existe una
clique con menos nodos y mayor frontera. La demostración de tal enunciado
surge de ver cómo se altera la frontera de una clique al quitar un nodo $v$. Las
aristas incidentes a $v$ pueden ser de dos tipos. Pueden conectar a $v$
con un nodo perteneciente a la clique o pueden conectar a $v$ con un
nodo que no pertenezca a la clique.}}\\

\par{\textbf{Las aristas del primer tipo son aristas que no pertenecían a la
frontera de la clique (ya que unían dos nodos pertenecientes a la clique)
pero al quitar un nodo $v$ de la clique, estas pasan a
pertencer a la frontera de la clique, ya que conectan un nodo fuera de la
clique (el nodo $v$ que se acaba de quitar) con uno que sí está en ella (el
otro nodo que sigue perteneciendo a la clique). La cantidad de aristas de este
tipo es exactamente la cantidad de nodos de la clique menos uno,
es decir $k$-1, porque para que sea una clique el nodo $v$ que fue quitado debe
haber estado conectado con todos los otros $k$-1 nodos de la clique. Entonces,
al quitar un nodo cualquiera la cantidad de aristas de frontera de la clique
aumenta en exactamente $k$-1 aristas.}}\\

\par{\textbf{Las aristas del segundo tipo son aristas que son parte de la
frontera de la clique pero al quitar un nodo $v$ dejan de serlo, ya que
conectan dos nodos de los cuales ninguno pertenece a la clique ($v$ deja de
pertenecer y el otro nodo no pertenecía). La cantidad de aristas de este tipo
es la cantidad de aristas incidentes a $v$ menos las correspondientes a unir
a $v$ con el resto de los nodos de la clique, es decir d($v$) - ($k$-1).
Entonces, al quitar un nodo cualquiera la cantidad de aristas de frontera de la
clique disminuye en el grado del nodo menos $k$-1, una cantidad que no puede
ser mayor a $n$-1-($k$-1).}}\\

\par{\textbf{En definitiva, al quitar un nodo de una clique la frontera de la
clique aumenta en $k$-1 y disminuye en $n$-1-$k$+1 = $n$-$k$. Si la cantidad de
nodos de la clique es $k$ $>$ $n$/2, entonces la frontera aumenta en al menos
$n$/2 y disminuye en a los sumo $n$/2. En conclusión, para toda clique con $k$
$>$ $n$/2 nodos, existe una clique de mayor frontera que se obtiene quitándole
nodos a la clique. Habiendo demostrado esto, podemos asegurar que la clique
máxima de un grafo no puede tener más de $n$/2 nodos. En no evaluar dichas
soluciones consiste la segunda poda implementada.} {\color{dGray}A continuación
se muestra un pseudocódigo del algoritmo exacto basado en backtracking
recursivo.}}\\

\begin{algorithm}[H]
	{\color{dGray}
	\caption{Pseudocódigo del algoritmo exacto}
	\KwData{\textbf{Grafo} $G(V,E)$}
	\textbf{Solucion} $solucion\_optima$ $\leftarrow$ $solucion\_vacia$\\
	\textbf{Solucion} $solucion\_actual$ $\leftarrow$ $solucion\_vacia$\\
	Recursion(0, $solucion\_optima$, $solucion\_actual$, \textbf{{\color{black}0}})\\
	\textbf{return} $solucion\_optima$
	}
\end{algorithm}

\begin{algorithm}[H]
	{\color{dGray}
	\caption{Llamada recursiva}
	\KwData{\textbf{int} $i$, \textbf{Solucion} $solucion\_optima$, \textbf{Solucion} $solucion\_actual$, {\color{black}\textbf{int size\_solucion}}}
	$solucion\_optima$ $\leftarrow$ Max($solucion\_optima$, $solucion\_actual$)\\
	\If{$i$ = $cantNodos$}{
		Terminar recursion
	}}
	\textbf{\If{size\_solucion + 1 $>$ cantNodos/2}{
		Terminar recursion
	}}
	{\color{dGray}
	\eIf{se\_conecta\_con\_clique($i$, $solucion\_actual$)}{
		Agregar $i$ a $solucion\_actual$\\
		Recursion($i+1$, $solucion\_optima$, $solucion\_actual$, \textbf{{\color{black}size\_solucion + 1}})\\
		Quitar $i$ de $solucion\_actual$\\
		Recursion($i+1$, $solucion\_optima$, $solucion\_actual$, \textbf{{\color{black}size\_solucion}})
	}{
		Recursion($i+1$, $solucion\_optima$, $solucion\_actual$, \textbf{{\color{black}size\_solucion}})
	}
	}
\end{algorithm}

\par{\textbf{Como se mencionó en la sección \textit{Pautas de Implementación},
Una solución se representa con un vector de booleanos y un entero. El vector
de booleanos tiene un elemento por cada nodo del grafo. El valor del
\textit{i}-ésimo booleano indica si el \textit{i}-ésimo nodo del grafo
pertenece al conjunto de nodos representado por la solución. El entero
representa el tamaño de frontera del conjunto representado. Si es negativo,
El conjunto no es una clique.}}\\

\par{\textbf{Como el conjunto de nodos en la solución se representa con un
vector de booleanos, agregar y quitar nodos (líneas 7 y 9 del pseudocódigo)
debería consistir únicamente en alternar el correspondiente elemento del
vector. Sin embargo, esas funciones también se encargan de recalcular la
frontera en función del nodo agregado o quitado, para lo cual deben recorrer los
nodos adyacentes al él.}}\\

\par{{\color{dGray}La función $Max$ compara dos soluciones y retorna la de mayor
frontera.} \textbf{Como lo único que hace esta función es comparar dos enteros
(el segundo elemento de la dupla de cada solución) no es necesario mostrar un
pseudocódigo para ella.}
{\color{dGray} $cantNodos$ es la cantidad de nodos del grafo. Cuando el índice
del nodo que se va a agregar $i$ es igual a dicha cantidad, significa que ya no
hay más nodos para agregar y la recursión debe terminar. En otras palabras, se
llegó a una hoja del árbol de backtracking.
La función $se\_conecta\_con\_clique$ toma un
nodo $i$ y una solución $S$ y retorna si la solución $S$ seguiría representando
una clique tras agregarle el nodo $i$. Para ello evalúa si existe una arista
entre el nodo $i$ y cada nodo de la clique representada por $S$.} \textbf{A
continuación se muestra el pseudocódigo de la función
\textit{se\_conecta\_con\_clique}.}}

\begin{algorithm}[H]
	\caption{\textbf{se\_conecta\_con\_clique}}
	\KwData{\textbf{int} $nodo$, \textbf{Solucion} $solucion$}
	\For{\textbf{int} $i$ $\in$ $[$ $0..cantNodos$-$1$ $]$ }{
		\If{$i$ $\in$ $solucion$ $\land$ $\neg \exists$ $(i,$ $nodo)$ $\in$ $E_G$}{
			\textbf{return FALSE}
		}
	}
	\textbf{return TRUE}
\end{algorithm}

\par{\textbf{Como la solución se representa con un arreglo de booleanos, ver si
el nodo \textit{i} pertenece a la solución (línea 2: }$i$ $\in$ $solucion$
\textbf{) consiste en obtener el i-ésimo booleano del arreglo, lo cual se
realiza en tiempo constante. Como el grafo se representa con matrices de
adyacencia, saber si determinada arista ($i$, $nodo$) existe (línea 2: }$\neg
\exists$ ($i$, $nodo$) $\in$ $E_G$\textbf{) consiste en obtener el elemento
($i$, $nodo$) de dicha matriz, lo cual también se realiza en tiempo constante.
Como esta función pertenece a la clase \textit{grafo}, el grafo en el cual se
verifica la existencia de las aristas no se pasa como parámetro de la
función.}}\\

\par{\textbf{La implementación de la segunda poda consiste simplemente en
evaluar la cantidad de nodos de la solución que se recorre (líneas 4 y 5 del
pseudocódigo). Si esta es mayor a la mitad de la cantidad de nodos se retorna
en lugar de seguir agregando nodos a la solución. Como la solución es un vector
de booleanos, la cantidad de nodos de la solución se pasa como un parámetro más
del algoritmo. Cuando se agrega un nodo a la solución (línea 8) se incrementa
en uno el valor de dicho parámetro.}}

{\color{dGray}\subsection{Orden de complejidad}}

\par{{\color{dGray}El algoritmo exacto recorre todas las posibles soluciones.
Si bien hemos implementado una poda, en el peor caso (un grafo completo $K_n$)
todas las soluciones serán válidas y todas ellas deberán ser evaluadas. Dado que
en una solución cada nodo puede estar o no estar, la cantidad de soluciones
posibles (la cantidad de subconjuntos del conjunto de nodos) es $2^n$ con
$n$ la cantidad de nodos. Las operaciones realizadas para cada solución son
polinomiales y, por lo tanto no aportan} \textbf{significativamente}
{\color{dGray}a la complejidad exponencial.}}\\

\par{\textbf{Como se mencionó en la
sección \textit{Pautas de Implementación}, el grafo se representa con listas
de adyacencia y con matriz de adyacencia simultáneamente.} {\color{dGray}La
función $se\_conecta\_con\_clique$} \textbf{recorre los $n$ nodos del grafo.
Como ya se vió en el pseudocódigo de esta función, todas las operaciones sobre
cada nodo son constantes} \textbf{(Como el grafo está representado con matriz de
adyacencia, saber si hay una arista entre un par de nodos es
constante)}{\color{dGray}, por lo que la complejidad de esta función es la
cantidad de nodos $n$ del grafo. Cuando se agregan o se quitan nodos de una
solución, la frontera de dicha solución debe recalcularse.}}\\

\par{\textbf{Como ya se mencionó, no se cuenta con una función que
calcule, dado un conjunto de nodos, la frontera de dicho conjunto en
el grafo. En lugar de eso, la frontera de una solución se calcula cuando
a una solución anterior se le agrega o quita un nodo, ya que todas
las soluciones que utilizamos se construyen agregándole
o quitándole un nodo a una solución ya usada.} {\color{dGray}La complejidad de
agr}\textbf{e}{\color{dGray}gar o quitar un nodo $i$ es $d_G(i)$ (el grado de
dicho nodo, que también puede acotarse superiormente por $n$), ya que deben
evaluarse todas sus aristas. En definitiva, la complejidad del algoritmo exacto
es $2^n * ( n + n + n )$} \textbf{$\in$ O(n.2$^n$).}}

{\color{dGray}\subsection{Experimentación}

\subsubsection{Performance}

\par{El archivo $test\_exacto.cpp$ en la carpeta $codigo/exacto$ contiene la
implementación del código que mide los tiempos de ejecución del algoritmo
exacto. Se ejecutó este programa con $N$ = {\color{black}\textbf{30}},\\$s$ = 1
y $k$ = 10. Es decir, para cada $n$ entero entre 1 y {\color{black}\textbf{30}},
se generaron 10 grafos aleatorios (con la función
$generar\_aristas\_aleatorias$). Para cada instancia generada se
corrieron {\color{black}\textbf{3}} verisones del algoritmo exacto. Uno con la
{\color{black}\textbf{primera}} poda{\color{black}\textbf{, otro con ambas
podas}} y otro sin ella{\color{black}\textbf{s}}.
En el siguiente gráfico se muestran los resultados obtenidos.}

\begin{center}
{\color{black}\textbf{Gráfico de performance del algoritmo exacto en\\función de la cantidad
de nodos del grafo}
\includegraphics[scale=1.3]{imgs/exacto_30_1_10.pdf}}
\end{center}

\par{Cada punto $\bullet$ en el gráfico representa el promedio de los tiempos
medidos para cada una de las 10 ejecuciones de una determinada cantidad de nodos
del grafo. El tamaño del segmento vertical sobre cada punto $\bullet$ representa
su varianza asociada. Además, para cada cantidad de nodos $n$ se graficaron la
máxima medición con $\blacktriangle$ y la mínima medición con
$\blacktriangledown$.}\\

\par{La función graficada con una curva sin puntos es {\color{black}\textbf{la}}
función {\color{black}\textbf{2.10$^{-7}$.x.2$^x$}}. Al estar utilizando escala
logarítmica en el eje de ordenadas, esta curva
se ve como una recta. Como se puede observar, la curva definida por la versión
sin poda del algoritmo (la curva resultante de unir los puntos $\bullet$) se
asemeja mucho a tal curva. Mientras que la versión con
{\color{black}\textbf{una}} poda tiene menores tiempos promedio de ejecución,
si bien tiene la misma complejidad.}\\

\par{{\color{black}\textbf{La versión con dos podas no se ve mejor a la versión
con una poda. Esto podría deberse a la forma en que los grafos aleatorios
son generados. Por cada posible arista, se tiene 0.5 probabilidad de
agregarla al grafo. En promedio, la cantidad de aristas de un grafo
aleatorio generado de esta forma debería estar al rededor de ($n$.$n$-1)/4 (la
mitad de la máxima cantidad de aristas posibles. Es probable entonces que
se generen pocas cliques grandes y, por lo tanto la segunda poda no elimine
más soluciones que la primera. Lo que sí resulta raro es que en determinados
casos, la versión con dos podas tenga mayores tiempos de ejecución que la
versión con una.}}}\\

\par{En la mayo{\color{black}\textbf{r}}ía de los casos, la varianza de
{\color{black}\textbf{todas las}} versiones es tan pequeña
que apenas puede verse en el gráfico. Esto nos dice que
{\color{black}\textbf{las tres}} versiones del
algoritmo son estables{\color{black}\textbf{, en cuanto a que, sin importar la
estructura específica de cada instancia, el tiempo de ejecución para instancias
del mismo tamaño, es muy similar.}} Era de esperarse de la versión sin poda, ya
que dado cualquier grafo, la cantidad de soluciones que deben recorrerse es
siempre la misma. Sin embargo, de la{\color{black}\textbf{s versiones con
podas}} se esperaba una varianza bastante
superior debido a que la cantidad de soluciones que est{\color{black}\textbf{os
deben}} recorrer depende
en gran medida de {\color{black}\textbf{la estructura específica del}} grafo
entrada.}}\\

\par{\textbf{Como se mencionó en la sección \textit{Implementación del
Algoritmo}, la primera poda no sirve en grafos completos, ya que todo
conjunto de nodos es una clique y, por lo tanto, ninguna solución se poda.
Para solucionar esto, se implementó la segunda poda que, sin importar la
estructura del grafo (sin importar la densidad de aristas) la cantidad de
soluciones que evalúa es exactmente la mitad que la versión sin podas.}}\\

\par{\textbf{El siguiente experimento busca mostrar tanto la
similitud entre las versiones con y sin poda, como la mejora de la segunda
poda respecto a estas dos, en instancias con grafos completos. Se ejecutó
el archivo \textit{test\_peor\_caso.cpp} en la carpeta \textit{codigo/exacto}.
Funciona igual que el test anterior pero en lugar de generar \textit{k}
grafos aleatorios genera un único grafo completo para cada valor de $n$. Se
repitieron los parámetros \\\textit{N} = 30 y \textit{s} = 1. En el siguiente
gráfico se muestran los resultados obtenidos.}}

\begin{center}
{\color{black}\textbf{Gráfico de performance del algoritmo exacto\\
con grafos completos}
\includegraphics[scale=1.3]{imgs/peor_30_1_10.pdf}}
\end{center}

\par{\textbf{Cada punto en el gráfico representa la medición obtenida para
un determinado tamaño de entrada. Como se realizó una medición por tamaño, no
hay varianza, mínimos ni máximos. El grafo entrada en todos los casos es un
grafo completo generado con la función \textit{generar\_completo} de la clase
grafo. Lo primero que salta a la vista es que los tiempos de ejecución de
la versión con una poda son superiores a los de la versión sin podas.
Debido a la escala logarítmica, la versión con dos podas no parece
tener tiempos de ejecución mucho mejores que las otras dos versiones,
sin embargo son de al rededor de la mitad, tal como se esperaba.}}
