\nonstopmode
\documentclass[10pt,a4paper]{article}
\usepackage[utf8]{inputenc} % para poder usar tildes en archivos UTF-8
\usepackage[spanish]{babel} % para que comandos como \today den el resultado en castellano
\usepackage{a4wide} % márgenes un poco más anchos que lo usual
\usepackage{color}
\usepackage{gnuplottex}
\usepackage{amssymb}
%\usepackage{ccfonts,eulervm}
\usepackage{dot2texi}
\usepackage{tikz}
\usetikzlibrary{shapes,arrows}
\usepackage[T1]{fontenc}
\usepackage{listings}
\usepackage{xcolor}
\usepackage{amsmath}
\lstset { %
    language=C++,
    %backgroundcolor=\color{black!5}, % set backgroundcolor
                   basicstyle=\ttfamily,
                keywordstyle=\color{blue}\ttfamily,
                stringstyle=\color{red}\ttfamily,
                commentstyle=\color{green}\ttfamily,
                morecomment=[l][\color{magenta}]{\#}
}
\usepackage{float}
\usepackage{fancyhdr}
\pagestyle{fancy}
\thispagestyle{fancy}
\addtolength{\headheight}{1pt}
\lhead{AED3}
\rhead{RTP3}
\usepackage[ruled,vlined,linesnumbered]{algorithm2e}
\usepackage[conEntregas]{caratula}
\renewcommand*{\algorithmcfname}{Algoritmo}

\begin{document}

\section*{Informe de modificaciones}

\par{Se reentrega únicamente la sección referente al algoritmo exacto. Las
modificaciones en el informe están marcadas en negrita, mientras que todo lo
que se mantuvo igual está en color gris.}

\subsection*{Implementación del Algoritmo}

\par{Se implementó una nueva poda y se agregó toda la explicación
correspondiente, incluyendo la justificación de por qué la poda es correcta.
Se agregó al pseudocódigo la utilización de dicha poda y se comentó esta
modificación en el pseudocódigo.
Se agregó una explicación de por qué no se incluyó el pseudocódigo de la
función $Max$. Se agregó el pseudocódigo de la función
$se\_conecta\_con\_clique$ y un breve análisis de su complejidad. También
se amplió la explicación sobre el pseudocódigo haciendo énfasis en la
estructura utilizada para representar una solución y la forma en que eso
afecta la complejidad de modificar una solución.}

\subsection*{Orden de complejidad}

\par{Se corrigió la complejidad del algoritmo. También se modificó la oración
que aseguraba que la complejidad lineal de cada iteración no modificaba la
complejidad de la cantidad de iteraciones (de lo cual habíamos deducido
erróneamente que $n.2^n$ $\in$ O($2^n$)). Se agregaron también las referencia
a la complejidad de la función $se\_conecta\_con\_clique$ y a la
complejidad de modificar una solución, ambas basándose en las estructuras
utilizadas para representar un grafo.}

\subsection*{Experimentación}

\par{Se rehizo el gráfico de performance para comparar las dos versiones
anteriores (con poda y sin poda) con la nueva versión del algoritmo (2 podas).
En la primera entrega se ejecutó la versión con poda hasta $n$ = 50. Para poder
comparar todas las versiones en los mismos tamaños de entrada, esta vez se
recortó a 30 el tamaño máximo de entrada.}\\

\par{Se realizó un segundo test para comparar las tres versiones en el peor
caso, es decir, con grafos completos. En este segundo test, todas las
versiones tienen altos tiempos de ejecución, por lo que no se pudo
ejecutar con más de 30 nodos. Con el fín de comparar ambos test, se decidió
que 30 sea el límite para ambos (otra razón para reducir el máximo tamaño
de entrada en el primer test). Para este segundo test se implementó una
nueva función de la clase grafo para generar grafos aleatorios. Se graficó
y analizó el resultado del segundo test.}

\end{document}
