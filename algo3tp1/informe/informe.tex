\nonstopmode
\documentclass[10pt,a4paper]{article}
\usepackage[utf8]{inputenc} % para poder usar tildes en archivos UTF-8
\usepackage[spanish]{babel} % para que comandos como \today den el resultado en castellano
\usepackage{a4wide} % márgenes un poco más anchos que lo usual
\usepackage{color}
\usepackage{gnuplottex}
%\usepackage{ccfonts,eulervm}
\usepackage[T1]{fontenc}
\usepackage{float}
\usepackage{fancyhdr}
\pagestyle{fancy}
\thispagestyle{fancy}
\addtolength{\headheight}{1pt}
\lhead{AED3}
\rhead{TP1}
\usepackage[spanish,ruled,vlined,linesnumbered]{algorithm2e}
\usepackage[conEntregas]{caratula}
\renewcommand*{\algorithmcfname}{Algoritmo}

\begin{document}

\titulo{Trabajo Práctico I}
%\subtitulo{Subtítulo del tp}

\fecha{\today}

\materia{Algoritmos y Estructuras de Datos III}

\integrante{Amil, Diego Alejandro}{68/09}{amildie@gmail.com}
\integrante{Barabas, Ariel}{775/11}{ariel.baras@gmail.com}
\integrante{Aleman, Damian Eliel}{377/10}{damianealeman@gmail.com}
\integrante{Fern\'andez Gonzalo Pablo}{836/10}{ralo4155@hotmail.com}

\maketitle

\tableofcontents
\newpage
\section{Introducción}
El presente informe apunta a documentar el desarrollo del Trabajo Práctico número 1 de la materia Algoritmos y Estructuras de Datos III, cursada correspondiente al segundo cuatrimestre del año 2013. Este trabajo pr\'actico consiste en la realización de un análisis teórico-experimental de un conjunto de problemas propuestos por la cátedra. Se requiere, para cada uno de los tres problemas, la implementación de un algoritmo que satisfaga criterios tanto de correctitud como de complejidad temporal.

Vamos a exponer, para cada uno de los problemas, los siguientes apartados:

\begin{itemize}
\item Una interpretación del enunciado, deallando ejemplos y/o casos particulares.
\item Una solución propuesta.
\item Un pseudocódigo que implemente dicha solución, junto con una explicación de su correctitud y una justificación de su complejidad.
\item Un apartado de testing, tanto de correctitud como de performance.
\end{itemize}


\section{Pautas de Implementación}
El lenguaje elegido para la implementación de los algoritmos es \texttt{C++}. De ser necesario vamos a utilizar la librería standard del mismo y aclarar los costos de las operaciones en cuestión. En el caso de tener que implementar una clase propia para simplificar el código o proveer de cierto encapsulamiento, los costos de los métodos de la misma serán verificados y justificados. La estructura de directorios que utilizaremos para la implementación será la siguiente para todos los ejercicios:

\begin{verbatim}
\codigo
	 timer.h
     \ej1
          ej1.cpp
          ej1.h
          Makefile
          \test
               ej1_test.cpp
               ej1_test.sh
\end{verbatim}

El archivo \texttt{timer.h} contiene las funciones necesarias pera medir el tiempo de ejecución de nuestros programas. Vamos a usar la función \texttt{clock$\_$gettime} de la librería \texttt{time.h}. Estas funciones son idénticas para las mediciones en todos los ejercicios. Las funciones pedidas por la cátedra se encuentran en el archivo \texttt{ej1.h}, y se incluyen en \texttt{ej1.cpp}. Este archivo trabaja con entrada y salida standard de manera que, para ejecutar un programa con su respectivo conjunto de casos, será suficiente con direccionarlo por consola escribiendo \texttt{./ej1<ej.in}. Obviaremos mencionar detalles referentes a la carga de datos en las implementaciones.

El directorio \texttt{/test} contiene los archivos necesarios para efectuar tests de performance de nuestros programas. Para hacer esto, sólo es necesario correr el programa \texttt{ej1$\_$test.cpp} pasandole 3 parámetros, que son: el \texttt{n} máximo hasta el cual testear, el salto entre \texttt{n} y la cantidad de tests para cada \texttt{n}. De esta manera ejecutar, por ejemplo \texttt{./ej1$\_$test 10000 500 20} va a generar y correr tests con $500 \leq n \leq 10000$, con un incremento de 500 entre cada \texttt{n} y 20 tests distintos para cada \texttt{n}. La salida de la ejecuci\'on consiste en los tiempos de ejecuci\'on medidos para cada \texttt{n}.

El script \texttt{ej1$\_$test.sh} encapsula la funcionalidad del programa anterior. Llamarlo con los mismos parámetros va a automáticamente compilar todo lo necesario, llamar a al programa para generar y correr los tests y graficar todo, generando un \texttt{.pdf} con el gráfico en ese mismo directorio. Este es el mismo proceso que empleamos para generar los gráficos de este informe.

Para los tests de performance, realizamos la medida de tiempos con la funcion clock\_gettime de la librería estandar de C++.
Ya que los tiempos dependen de la carga del sistema que lo está corriendo cuando medimos los tiempos, medimos la media de los datos, bajo la
siguiente ecuacion:

\begin{equation}
 \bar{x} = \frac{1}{n} \sum_{i=1}^{n}x_{i}
\end{equation}
De esta manera, el valor medio hecho con muchas mediciones es un valor más confiable que el una unica medición.
También, para tener una noción de como varían los datos, calculamos el desvío estandar de la muestra de tiempos bajo la ecuación:

\begin{equation} 
\sigma = \sqrt \frac{\sum\limits_{i=1}^{n}
  \left(x_{i} - \bar{x}\right)^{2}}
  {n-1}
\end{equation}

\newpage

\section{Ejercicio 1}
\subsection{Interpretación del enunciado}
\par{Se tiene una imprenta en la que cada día deben realizarse determinados
trabajos. Los trabajos son realizados por las máquinas de la imprenta. Esta
imprenta cuenta con dos máquinas idénticas. Los trabajos son distintos, pero
pueden ser realizados por cualquiera de la dos máquinas siempre que se repete
un orden dado. Las máquinas deben prepararse para realizar determinados
trabajos y estas preparaciones tienen un costo, el cuál depende tanto del
trabajo a realizar como del estado de la máquina. Teniendo los trabajos
ordenados y los costos de preparación de las máquinas para cada trabajo, se
debe tomar cada trabajo, elegir la máquina en la que se realizará ese trabajo,
prepararla, y realizar el trabajo. El objetivo del ejercicio es desarrollar un
algoritmo que determine la mejor distribución de los trabajos entre las
máquinas que reduzca el costo total de la operación, es decir la suma de los
costos de cada preparación.}

\subsection{Resolucion}
\par{Sea $p(j)$ el problema de obtener la distribución óptima de $j$ trabajos
entre $2$ máquinas, definimos el problema derivado $p'(j,i)$ con $0<i<j$, el
cual consiste en obtener la distribución óptima de $j$ trabajos entre $2$
máquinas con la restricción de que los últimos trabajos realizados en cada
máquina sean $t_j$ y $t_i$. Notar que es indiferente en qué máquina se ejecuta
cada uno de estos. Definimos entonces la $distribucionOptima(j,i$)
como la distribución de costo mínimo con $j$ trabajos y el
trabajo $t_i$ como último trabajo de una de las máquinas. Notar que cualquier
distribución de $j$ trabajos siempre va a tener a $t_j$ como último trabajo en
una de las máquinas. Luego, la solución para el problema $p(j)$ es la solución
de costo mínimo de $p(j,i)$ para todo $i$, es decir:}

$$p(j) = \min_{\forall i} p(j,i) $$

El problema $p'$ se resolvió con programación dinámica mediante $distribucionOptima(j,i)$, que se calcula de la siguiente manera:
$$distribucionOptima(j,i) = \left\{
\begin{array}{c l}
 costo(0,1) & j = 1\\
 distribucionOptima(j-1,0) + costo(j-1,j) & i = 0 \land j > 1\\
 distribucionOptima(j-1,i) + costo(j-1,j) & 1 \le i < j-1 \land j > 1\\
 \displaystyle \min_{\forall k} distribucionOptima(j-1,k) + costo(k,j) & i = j-1 \land j > 1\\
\end{array}
\right$$

\par{Notar que siempre $i < j$. La entrada para ambos problemas es la cantidad
de trabajos y los costos de preparación de las máquinas para cada trabajo
definidos como:}
\begin{equation*}
$costo($i$,$j$) = Costo de preparar una máquina para realizar el trabajo $t_j$ luego
de haber realizado el trabajo $t_i$ (o no haber realizado ningún trabajo si
$i=0$). $0 \leq i < j
\end{equation*}\\
\par{A continuación se muestra el pseudocódigo de la solución para $p(j)$,
resolviendo $p'(j,i)$.}\\
\begin{algorithm}[H]
	\caption{Algoritmo de Ejercicio 1}
	\KwData{\textbf{int} $cantTrabajos$, $costos$}
	$distribucionOptima(1,0) \longleftarrow costo(0,1)$\\
	\For{$j \in \{2..cantTrabajos\}$}{
		
		$distribucionOptima(j,0) \longleftarrow distribucionOptima(j-1,0) + costo(j-1,j)$\\
		$distribucionOptima(j,j-1) \longleftarrow distribucionOptima(j-1,0) + costo(0,j)$
		
		\For{$i \in \{1..j-2\}$}{
	
			$distribucionOptima(j,i) \longleftarrow distribucionOptima(j-1,i) + costo(j-1,j)$\\
			$distribucionOptima(j,j-1) \longleftarrow min(distribucionOptima(j,j-1), distribucionOptima(j-1,i) + costo(i,j))$\\
		}
	}
	\textbf{return} $ \displaystyle \min_{\substack{\forall i}} distribucionOptima(cantTrabajos,i) $\\
\end{algorithm}

\subsection{Demostración de Correctitud}

\textbf{Propiedad 1 -}  \emph{ Existe un $i$ tal que la solución del problema $p'(j,i)$
es solución de $p(j)$. }\\

\textbf{Demostración:} Es inmediato ver que alguna solución del problema $p(j)$ pertenece al
conjunto de las soluciones de $p(j,i)$ para algún $i$. Dado que $p(j)$ no tiene
la restricción del trabajo realizado como último en la otra máquina, la solución al
problema $p(j)$ es la solución de $p(j,i)$ con menor costo:
$$p(j) = \min_{\forall i} p(j,i)$$

\textbf{Propiedad 2 -} \emph{$ (\forall i < j) \quad distribucionOptima(j,i) $ devuelve una de las distribuciones con menor costo dados
j trabajos e i como último trabajo.}\\

\textbf{Demostración:}
Vamos a demostrar por inducción en j.


\textbf{Caso base. Si $j = 1:$}
Queremos ver que $ (\forall i < j) \quad distribucionOptima(1,i) $ devuelve una de las distribuciones con menor costo dados
$1$ trabajo e i como ultimo trabajo. Notar que siempre $i < j$ , por lo tanto $i$ debe ser $0$. 
Hay sólo una forma de configurar un trabajo en las dos máquinas, que este trabajo sea el único de una máquina
y la otra máquina este libre de trabajos, es decir que el costo de la distribución optima es igual a el costo de preparar
el trabajo $1$ luego del trabajo $0$ que es los mismo que \\
$distribucionOptima(1,0) = costo(0,1)$.
Luego, queda demostrado el caso base.\\


\textbf{Paso Inductivo:} Se dividirá la demostración en dos casos :\\

\textbf{Caso $i <  j-1$:} 

Definimos a $ distribucionOptima(j,i) $ = 
$distribucionOptima($j$-1,$i$) + costo($j$-1,$j$) & $\\
Supongamos que no es la distribución optima y lleguemos a un absurdo. Decir que es la distribución es óptima es lo mismo que afirmar que
que no hay ninguna distribución de $j$ trabajos con $i$ como último trabajo que cueste menos que $ distribucionOptima(j,i) $.
Sea $d_2$ la distribución que cuesta menos que $ distribucionOptima(j,i) $ y tiene j trabajos e i como último trabajo.
Luego por como se 
define la preparación de los trabajos,en $d_2$ el trabajo $j$ está al final de una máquina (y como $(i < j-1) \Rightarrow i \neq j $ )
y el trabajo $i$ está en la otra máquina. De esta manera, si quitamos el trabajo $j$, obtenemos una configuración de $j-1$ trabajos con $i$
como último trabajo, llamemosla $d_{2'}$.

\begin{align*}
\text{Por lo tanto}
\qquad costo (d_2) &= costo (d_{2'}) + costo(j-1,j) \\
\text{y como }
 \qquad costo(d_2) &<  distribucionOptima(j,i) \\ 
\text{entonces}
 \qquad costo (d_{2'}) &< distribucionOptima(j-1,i). Absurdo. 
\end{align*}

\textbf{Caso $i = j-1:$}

Definimos a $ distribucionOptima(j,i) $ =  
\displaystyle \min_{\substack{\forall i \in \{0..j-1\}}} \ $distribucionOptima($j$-1,$i$) + costo($i$,$j$)$
 & \\
 
Supongamos que no es la distribución optima y lleguemos a un absurdo.
Sea $d_2$ la distribución que cuesta menos que $ distribucionOptima(j,i) $ y tiene $j$ trabajos e $i$ como último trabajo.
Luego por como se 
define la preparación de los trabajos,en $d_2$ el trabajo $j$ está al final de una máquina (y como $(i = j-1) \Rightarrow i \neq j $ )
y el trabajo $i$ está en la otra máquina.
De esta manera, si quitamos el trabajo j, obtenemos una configuración de $j-1$ trabajos con $j-1$
y algún k (con $ 0 \le k < j-1$) como últimos trabajos en las dos máquinas, llamemosla $d_{2'}$.

\begin{align*}
\text{Por lo tanto}
\qquad costo (d_2) &= costo (d_{2'}) + costo(k,j)  \text{para algun k (con }  0 \le k < j-1) \\
\text{y como }
 \qquad costo(d_2) &<  distribucionOptima(j,i) \\ 
\text{entonces}
 \qquad costo (d_{2'}) &< distribucionOptima(j-1,k). 
\end{align*}

Absurdo pues por la definición
\displaystyle \min_{\substack{\forall k \in \{0..j-1\}}} \ $distribucionOptima($j$-1,$k$) + costo($k$,$j$)$. \\

\textbf{Corolario de Propiedad 2 -} \emph{$ distribucionOptima(cantTrabajos,i) $ devuelve una de las distribuciones con menor costo dados
cantTrabajos trabajos e i como ultimo trabajo.)}\\

Ahora veremos que el algoritmo retorna la distribución óptima según como la definimos.\\
\textbf{Caso $j = 1:$} En la línea $1$ del pseudocódigo se asigna la distribución óptima según como la definimos:
$$costo(1,0)$$
\textbf{Para $j > 1$:}
\textbf{Caso $i = 0$:} En la línea $3$ del pseudocódigo se asigna la distribución óptima según como la definimos:
$$distribucionOptima(j-1,0) + costo(j-1,j)$$
\textbf{Caso $1 \le i < j-1:$} En la línea $6$ del pseudocódigo se asigna la distribución óptima según como la definimos:
$$distribucionOptima(j-1,i) + costo(j-1,j)$$
\textbf{Caso $i = j-1:$} Sea $g(i) = distribucionOptima(j-1,i) + costo(i,j).$
En este caso se inicializa $distribucionOptima(j,j-1)$ en la linea $4$ con $g(0)$.
Luego en el ciclo interno se compara el valor actual de $distribucionOptima(j,j-1)$ con los valores
$g(i)$ para todo $ 1 \le i \le j-2 $. 


\subsection{Cota de Complejidad}
Para demostrar la complejidad del algoritmo, vemos que la incialización de la matriz de costos
tiene una complejidad de $O(n^2)$, con $n$ la cantidad de trabajos, ya que completa una matriz de $n^2$ elementos con dos ciclos anidados.

Luego, el algoritmo completa la matriz de $distribucionOptima$, con dos ciclos anidados. En cada uno de estos ciclos,
las variables de iteración se van incrementando de a uno y cada una debe iterar
en un numero menor o igual a la cantidad de trabajos.
\textbf{Cada asignación tiene una complejidad de $O(1)$, ya que al momento de la asignación (en la iteración $j$)
$distribucionOptima(j-1,i)$ ya está calculada y guardada en la matriz.}\\

\begin{algorithm}[H]
	\caption{Algoritmo de Ejercicio 1}
	\KwData{\textbf{int} $cantTrabajos$, $costos$}
	$distribucionOptima(1,0) \longleftarrow costo(0)(1)$	         \tcc*[r]{$O(1)$}
	\For{$j \in \{2..cantTrabajos\}$ \tcc*[r]{$O(n)$}}{	
		
		$distribucionOptima(j,0) \longleftarrow distribucionOptima(j-1)(0) + costo(j-1)(j)$\\ 	
		$distribucionOptima(j,j-1) \longleftarrow distribucionOptima(j-1)(0) + costo(0)(j)$		
		
		\For{$i \in \{1..j-2\}$ \tcc*[r]{$O(n)$} }{	
	
			$distribucionOptima(j,i) \longleftarrow distribucionOptima(j-1,i) + costo(j-1,j)$\\	${O(1)}$
			$distribucionOptima(j,j-1) \longleftarrow min(distribucionOptima(j,j-1), distribucionOptima(j-1)(i) + costo(i)(j))$\\
		}
	}
	\textbf{return} $ \displaystyle \min_{\substack{\forall i}} distribucionOptima(cantTrabajos,i) $	\tcc*[r]{$O(n)$}
\end{algorithm}

Luego la complejidad del algoritmos es $ O(n^2) + n * O(n) + O(n) = O(n^2).$

\subsection{Implementación}

\begin{lstlisting}
salida_ej1 resolver_ej1(entrada_ej1 entrada) {
  int cantTrabajos = entrada.n;
  vector<vector<int> > costo = entrada.matriz;

  //En la posicion i j está el costo de la configuración optima para
  //los primeros i trabajos con el trabajo j como ultimo de la otra máquina.
  vector < vector<par> > 
	distribucionOptima(cantTrabajos+1,vector<par> (cantTrabajos) );
  distribucionOptima[1][0] = make_pair(costo[0][1],0);
  
  for ( int j = 2; j <= cantTrabajos; ++j ) {
    distribucionOptima[j][0] = make_pair (distribucionOptima[j-1][0].first 
					+ costo[j-1][j],j-1);
    distribucionOptima[j][j-1] = make_pair (distribucionOptima[j-1][0].first
					+ costo[0][j],0);
    for ( int i = 1; i < j-1 ; ++i ) {
      distribucionOptima[j][i] = make_pair(distribucionOptima[j-1][i].first
						    + costo[j-1][j],j-1);
      int costoAlternativo = distribucionOptima[j-1][i].first + costo[i][j];
      if (costoAlternativo < distribucionOptima[j][j-1].first) {
	    distribucionOptima[j][j-1] = make_pair(costoAlternativo,i);
      }
    }
  }

  vector<int> tEnMaq1(0);
  int k = 0;
  int C = distribucionOptima[cantTrabajos][0].first;

  int elAnterior = cantTrabajos-1;
  int ultimoOtraMaquina = 0;
  for (int i=1; i<cantTrabajos; i++) {
      if (distribucionOptima[cantTrabajos][i].first < C) {
	C = distribucionOptima[cantTrabajos][i].first;
	elAnterior = distribucionOptima[cantTrabajos][i].second;
	ultimoOtraMaquina = i;
      }
  }
  
  //cout << "el anterior es " << elAnterior << endl;

  int trabajoEnMaquina = cantTrabajos;	
  tEnMaq1.push_back(trabajoEnMaquina);
  k++;

  while (elAnterior != 0){
    tEnMaq1.push_back(elAnterior);
    k++;
    trabajoEnMaquina = elAnterior;
    while (ultimoOtraMaquina > elAnterior) {
      ultimoOtraMaquina = 
	      distribucionOptima[ultimoOtraMaquina][elAnterior].second;
      }
    elAnterior = 
	distribucionOptima[trabajoEnMaquina][ultimoOtraMaquina].second;
}

  vector<int> e = vector<int>(k);
  for (int i=0; i<k; i++) {
	  e[i] = tEnMaq1[k-i-1];
  }

  //cout << "vector de trabajos" << endl;
  //mostrarVec(e);
  
  //cout << "matriz de cofiguracion Optima" << endl;
  //mostrarMatriz(distribucionOptima,cantTrabajos+1,cantTrabajos);
  //cout << endl;

  salida_ej1 salida(C, k, e);
  return salida;
}
\end{lstlisting}
\newpage
\subsection{Testing de Correctitud}

Los tests expuestos a continuación fueron diseñados con el fin de verificar
diferentes casos particulares que pudimos identificar. Para cada test vamos
a exponer la entrada, la salida y, en caso de que sea necesario, una
justificaci\'on de la correctitud de la soluci\'on.\\

\noindent\textbf{Test$\#$1}\\
\textbf{Caracterización:} Un solo trabajo.\\
\textbf{Input:}\\ \texttt{1\\5}\\
\textbf{Output:} \texttt{5 1 1}\\
\textbf{Status:} OK. Asigna el único trabajo a la máquina 1 y el costo total
es el costo de ubicar el trabajo 1 en la máquina 1.\\

\noindent\textbf{Test$\#$2}\\
\textbf{Caracterización:} Dos trabajos que conviene ubicar uno trás otro.\\
\textbf{Input:}\\ \texttt{2\\3\\8 3}\\
\textbf{Output:} \texttt{6 2 1 2}\\
\textbf{Status:} OK. Ubica ambos trabajos en la misma máquina y el costo
total es el costo de ubicar el trabajo 1 en una máquina vacía más el costo
de ubicar el trabajo 2 tras el trabajo 1.\\

\noindent\textbf{Test$\#$3}\\
\textbf{Caracterización:} Dos trabajos que conviene ubicar como primeros en
una máquina.\\
\textbf{Input:}\\ \texttt{2\\3\\3 8}\\
\textbf{Output:} \texttt{6 1 2}\\
\textbf{Status:} OK. Ubica un trabajo en cada máquina y el costo total es la
suma de los costos de ubicar cada trabajo como primer trabajo de una máquina.\\

\noindent\textbf{Test$\#$4}\\
\textbf{Caracterización:} Todos los costos son iguales.\\
\textbf{Input:}\\ \texttt{5\\1\\1 1\\1 1 1\\1 1 1 1\\1 1 1 1 1}\\
\textbf{Output:} \texttt{5 5 1 2 3 4 5}\\
\textbf{Status:} OK. Cualquier distribución de trabajos en máquinas tiene
costo igual a 5.\\

\noindent\textbf{Test$\#$5}\\
\textbf{Caracterización:} Todos los costos son distintos (incrementando).\\
\textbf{Input:}\\ \texttt{3\\1\\2 3\\4 5 6}\\
\textbf{Output:} \texttt{8 1 3}\\
\textbf{Status:} OK.\\

\noindent\textbf{Test$\#$6}\\
\textbf{Caracterización:} Todos los costos son distintos (decrementando).\\
\textbf{Input:}\\ \texttt{3\\6\\5 4\\3 2 1}\\
\textbf{Output:} \texttt{11 3 1 2 3}\\
\textbf{Status:} OK.\\

\noindent\textbf{Test$\#$7}\\
\textbf{Caracterización:} Varios trabajos que conviene ubicar uno tras otro.\\
\textbf{Input:}\\ \texttt{5\\1\\8 1\\8 8 1\\8 8 8 1\\8 8 8 8 1}\\
\textbf{Output:} \texttt{5 5 1 2 3 4 5}\\
\textbf{Status:} OK. Ubica todos los trabajos en la misma máquina.\\

\noindent\textbf{Test$\#$8}\\
\textbf{Caracterización:} Varios trabajos que conviene ubicar intercalados.\\
\textbf{Input:}\\ \texttt{5\\1\\1 8\\8 1 8\\8 8 1 8\\8 8 8 1 8}\\
\textbf{Output:} \texttt{5 3 1 3 5}\\
\textbf{Status:} OK. Ubica los trabajos impares en la misma máquina (por lo
tanto, ubica los pares en la otra).\\

\noindent\textbf{Test$\#$9}\\
\textbf{Caracterización:} La subsolución de la solución óptima no es solución
óptima para su subproblema asociado.\\
\textbf{Input:}\\ \texttt{3\\5\\3 5\\1 8 8}\\
\textbf{Output:} \texttt{11 1 3}\\
\textbf{Status:} OK. La solución óptima consiste en ubicar los trabajos 1 y 2 en
una máquina y el trabajo 3 en otra. Sin embargo, la subsolución de dos trabajos
que ubica a los trabajos 1 y 2 en la misma máquina no es solución óptima del
subproblema asociado (distribuír los trabajos 1 y 2 entre las 2 máquinas de
forma óptima.)\\

\newpage
\subsection{Testing de Performance}

\par{Realizamos un gráfico comparando la sucesión de tiempos obtenida con una función cuadrática, pues demostramos que la complejidad del algoritmo es cuadrática.
La función $f = n^2$, donde $C = \frac{1}{20500000}$. Se midió el tiempo con n desde 1 hasta 1000 con saltos de a 10 (con 100 mediciones por cada n).}


\begin{figure}[H]
\centering
\includegraphics{imgs/ej1_1000_10_100.pdf}
\caption{Test Performance: Tiempo(s) vs Cantidad de Trabajos.}
\end{figure}

\par{Hay que notar que las franjas para cada tamaño de entrada muestran el desvío estandar de todos los valores conseguidos para ese tamaño, esto nos pareci\'o mucho m\'as significativo que simplemente mostrar el máximo y el m\'inimo, ya que estos valores pueden variar mucho por otros procesos que pueda estar ejecutando la computadora a la vez.}

\newpage

%\usepackage{amsmath} %<-- ajaja quien puso esto aca??
\section{Ejercicio 2}

\subsection{Interpretación del enunciado}
\par{Se tienen una cantidad $n$ de cursos con sus respectivas fechas de inicio y finalizaci\'on representadas con n\'umeros enteros positivos. El objetivo de este ejercicio es desarrollar un algoritmo que, dadas las fechas de inicio y fin de cada curso, encuentre un subconjunto de cursos que no se solapen entre s\'i, es decir, sean compatibles. Se pide que la complejidad del algoritmo sea estrictamente menor a O($n$^{2})$ siendo $n$ la cantidad de cursos$.}
\par{Para este ejercicio, asumiremos que los cursos no se interrumpen. Es decir, por ejemplo, que el curso que arranca el día 5 y termina el día 8 va a dictarse también los días 6 y 7. Entonces lo que nos queda es un problema de optimización: ¿Cómo elegir la mayor cantidad de cursos posibles de manera que ningún par de cursos dado tenga días en común?. En particular para que se puedan tomar dos cursos, la fecha de inicio de uno de los dos tiene que ser mayor (estricto) que la fecha de finalizacon del otro curso.}
\par{Cabe destacar que lo que este problema busca es maximizar la cantidad de cursos disponibles para ser cursados por un único alumno, y no buscar que haya más días cubiertos por un curso. También hay que mencionar que los cursos no tienen ningún valor numérico asignado.}
%	PONER EJEMPLOS

\subsection{Resolución}
\par{Para resolver este ejercicio se desarroll\'o un algoritmo goloso. Se busca devolver un conjunto de cursos. El algoritmo itera agregando, en caso de ser posible, un curso en cada iteraci\'on. Mientras se pueda seguir agregando cursos (no se puede si todos los restantes se solapan con alguno seleccionado), se agrega de los que quedan, el de menor fecha de finalizaci\'on que no se solape con alguno seleccionado. Luego se devuelven los cursos seleccionados.}
\par{A continuación expondremos un pseudocódigo de la solución implementada. Suponemos los datos levantados directamente desde la entrada y almacenados en un vector de cursos llamado $cronograma$.}\

%~ \begin{algorithm}[H]
%~ \SetLine
%~ \hspace{-13pt}\textbf{cronograma} resolver(\textbf{cronograma} $C$)\\
%~ C $\longleftarrow$ ordenar$\_$finalización($C$)\\
%~ $ultimaFecha \longleftarrow -1$\\
%~ $cronograma\_aceptado \longleftarrow \emptyset$\\
%~ \ForEach{c $\in$ C}{
  %~ \If{$fechaInicio(c) > ultimaFecha$}{
        %~ $cronograma\_aceptado \longleftarrow cronograma\_aceptado \cup c$\\
        %~ $ultimaFecha \longleftarrow fechaFin(c)$\\
  %~ }
%~ }
%~ \textbf{return} $cronograma\_aceptado$\\
%~ \caption{Resolución Golosa Ejercicio 2}
%~ \end{algorithm}
%~ 
\begin{algorithm}[H]
	\caption{Resolución Golosa Ejercicio 2}
	\begin{algorithmic}
		\KwData{vector(Curso) $entrada$}\\
		vector(Curso) $cronograma\_aceptado \longleftarrow \emptyset$\\
		\While{$puedoAgregarCursos$}{ 
			$cronograma\_aceptado \longleftarrow cronograma\_aceptado$ \cup\\
			$c \in entrada$ / $sonCompatibles(c, cronograma\_aceptado)$ \land \\
							$fechaFin(c) \geq fechaFin(c')$ \forall c'$ \in$ $entrada$
		}
		\textbf{return} $cronograma\_aceptado$\\
	\end{algorithmic}
\end{algorithm}
\par{El vector de cursos $entrada$ contiene todos los cursos de la instancia, mientras que el vector $cronograma\_aceptado$ contiene los que se van agregando a la soluci\'on. La funci\'on $sonCompatibles$ devuelve si el curso $c$ no se solapa con alg\'un curso del vector de aceptados.}

%~ \subsection{Complejidad}
%~ En lo que a complejidad se refiere, los fragmentos más costosos de nuestro algoritmo son la función ordenar$\_$finalización (l.2) y el ciclo que recorre linealmente todos los cursos $c$ dentro del cronograma $C$ (l.5). La función ordenar$\_$finalización está implementada mediante la función \textfff{sort} dentro de la STL de \textfff{C++}. La misma tiene una complejidad de $O(n log n)$ \footnote{http://www.cplusplus.com/reference/algorithm/sort/}. El ciclo de que recorre todos los cursos del cronograma $C$ tiene una complejidad de $O(n)$, ya que implementamos el cronograma sobre un vector. Luego, la complejidad de nuestra función resolver termina siendo de $T(n) = O(n log n) + O(n) = O(n log n)$.

\subsection{Complejidad}
\par{Para obtener en cada iteraci\'on del ciclo principal (l\'inea 2) el curso con menor fecha de finalizaci\'on, se pueden ordenar, antes de entrar al ciclo, los cursos seg\'un su fecha de finalizaci\'on. Ordenar los cursos puede hacerse en $O(n log n)$ con $n$ la cantidad de elementos a ordenar, en este caso, la cantidad de cursos de la entrada. Luego, con los cursos ordenados, el ciclo debe iterar sobre los $n$ cursos, evaluando si cada uno es compatible con el resto del cronograma aceptado. Esto puede hacerse determinando si se solapa con el anterior (como se agregan en orden, si se solapa con alguno, tiene que ser con el \'ultimo agregado), lo cual tiene complejidad O(1), ya que es una comparaci\'on de enteros (la fecha de finalizaci\'on del curso que se itera y la del \'ultimo curso agregado). Entonces, El ciclo que recorre los $n$ cursos del cronograma de entrada tiene una complejidad de $O(n)$. Luego, la complejidad de la función $resolver$ termina siendo de $T(n) = O(n log n) + O(n) = O(n log n)$.}

\subsection{Implementaci\'on}
\par{La implementaci\'on de este algoritmo en C++ primero ordena los cursos seg\'un su fecha de finalizaci\'on. Luego recorre todos los cursos en orden y para cada curso, eval\'ua si puede ser agregado o no. La función de ordenamiento está implementada mediante la función \textfff{sort} dentro de la STL de \textfff{C++}. La misma tiene una complejidad de $O(n log n)$ \footnote{http://www.cplusplus.com/reference/algorithm/sort/}. En cada iteraci\'on del ciclo que recorre los cursos, no hace m\'as que comparar la fecha de finalizaci\'on del curso que se itera con la del \'ultimo curso agregado. Si es mayor estricta, agrega el curso a los cursos aceptados.}

\subsection{Demostraci\'on de Correctitud}
\par{En esta secci\'on vamos a demostrar que el algoritmo resuelve el problema de encontrar la m\'axima cantidad de cursos que no se superponen.
Para eso demostraremos las siguientes propiedades:}

\begin{itemize}
\item \textbf{Prop 1: }\emph{Sea B un conjunto de cursos que maximiza la cantidad de cursos compatibles, y sea A el conjunto que siempre elige la de menor fecha de finalizaci\'on que sea compatible, \#A = \#B.}
\item \textbf{Prop 2: }\emph{Nuestro algoritmo siempre elige la de menor fecha de finalizaci\'on que sea compatible.}\\
\end{itemize}

%\textbf{1. Sea B un conjunto de cursos que maximiza la cantidad de cursos compatibles,
%   y sea A el conjunto que siempre elige la de menor fecha de finalizaci\'on que sea compatible. Luego \#A = \#B.}\\
%Para esto veremos que una solucion al problema reside en tomar el curso de menor fecha de finalizacion que sea compatible con
%los cursos tomados hasta el momento de la seleccion y encontrar la maxima cantidad de cursos que sean compatibles con el curso que acabamos de elegir.

\textbf{Prop 1)} Para demostrar esto, emplearemos la siguiente notaci\'on: Sea A = $\{A_1,A_2,A_3...A_k\}$ el conjunto de cursos que da nuestro algoritmo, donde se cumple que:
\begin{equation}
fechaFin(A_i)<fechaFin(A_j) \quad \forall i\textlessj ,1\le i<j\le k\\
\end{equation}
con $k$ la cantidad de cursos de nuestra soluci\'on; y definiendo a B = $\{B_1,B_2,B_3..B_m\}$, donde:\\
\begin{equation}
fechaFin(Ai)\textless fechaFin(A_j) \quad \forall i<j, 1\le i<j\le m\\ 
\end{equation}
con $m$ la cantidad de cursos de la soluci\'on \'optima, como B es el conjunto de cursos de la soluci\'on \'optima vale que $k\le m$ .\\

\textbf{Lema:} $FechaFin(A_i) \le FechaFin(B_i)$ $\forall i: 1\le i \le k$. Probaremos esto por inducción. Queremos ver que:
\begin{equation}
	fechaFin(A_i)\leq fechaFin(B_i) \quad \forall i: 1\le i \le k 
\end{equation}

Tomamos como caso base $fechaFin(A_1)\leq fechaFin(B_1)$. Esto es cierto ya que al inicio del algoritmo, cualquier curso es compatible (ya que el conjunto de cursos seleccionados hasta ese momento es vac\'io) y adem\'as el algoritmo siempre selecciona el curso compatible con menor fecha de finalizaci\'on. Luego queda probado el caso base.\\

Nuestro paso inductivo requiere probar que:
\begin{equation}
\forall c : 1 \le c < k \quad fechaFin(A_c)\leq fechaFin(B_c) \quad \Rightarrow fechaFin(A_{c+1})\leq fechaFin(B_{c+1})
\end{equation}

Como vale la hip\'otesis inductiva sabemos que $fechaFin(A_c)\leq fechaFin(B_c)$. Tambi\'en sabemos que la $fechaFin(B_{c+1}) > fechaInicio(B_{c+1})$ y, como B es un conjunto que tiene a cursos compatibles tenemos que $fechaInicio(B_{c+1}) > fechaFin(B_c)$. Luego con la afirmaci\'on anterior sumada a la hip\'otesis inductiva podr\'iamos seleccionar el curso $B_{c+1}$:

	  \begin{equation}
		fechaFin(A_c) \stackrel{\rm{por HI}} \le  fechaFin(B_c) \stackrel{\rm{B conj compatible}} < fechaInicio(B_{c+1})   \\ 
	\end{equation}
	  \\ Ya que nuestro algoritmo selecciona el curso compatible con menor fecha de finalización, se deduce
	  que $fechaFin(A_{c+1})\le fechaFin(B_{c+1})$ \Box \\
	  
	
Con el lema podremos probar que vale que $k\ge m$, y como sab\'iamos que $k\le m$ tendremos que el conjunto
de soluciones dadas por el algoritmo es \'optimo en cuanto a maximizar la cantidad de cursos que sean compatibles. Demostraremos, a continuación, que $K\ge m$, por el absurdo.\\ 

Sea $B$ el conjunto que maximiza la cantidad de cursos con un cardinal de $m$, y el conjunto que devuelve nuestro algoritmo un cardinal de $k$. Supongamos que $k<m$ y lleguemos a un absurdo.
Como $B$ es un conjunto que tiene cursos compatibles, el curso $B_{k+1}$ es compatible con el conjunto $\{B_1,B_2,B_3..B_k\}$. Del lema sabemos que $FechaFin(A_k) \le fechaFin(B_k)$, por lo tanto $B_{k+1}$ tambi\'en ser\'ia compatible con el conjunto $\{A_1,A_2,A_3..A_k\}$. Luego, ese curso entraria en el cronograma de cursos acepatados de nuestro algoritmo. Esto es un absurdo, ya que supusimos que $\#A = k$ \Box\\


\textbf{Prop 2)} Es preciso notar que nuestro algoritmo ordena en forma creciente de acuerdo a la fecha de finalizaci\'on y luego recorre desde el de menor fecha, hasta el de mayor fecha de finalizaci\'on. Por otro lado siempre elige tomar un curso m\'as, si este es compatible ya que de la forma que recorremos vale que:
\begin{equation}
fechaFin(curso_{iteracionj}) \geq fechaFin(curso_{iteracioni}) \quad \forall j>i
\end{equation}


Adem\'as el algoritmo en cada iteraci\'on comprueba que: 
\begin{equation}
fechaInicio(curso_{iteracionj}) > ultimaFechaFin 
\end{equation}
para que no haya solapamientos entre el curso a elegir con los ya elegidos anteriormente.\\
Esto es v\'alido debido a que la $fechaInicio(curso_{iteracionj}) > ultimaFechaFin $ y\\
$fechaFin(curso_{iteracionj}) \geq fechaFin(curso_{iteracioni}) \quad \forall j>i$\\

\newpage
\subsection{Testing}
\textbf{Correctitud}\\
Los tests expuestos a continuación fueron diseñados con el fin de verificar diferentes casos particulares que pudimos identificar. Para cada test vamos a exponer la entrada, la salida y, en caso de que sea necesario, una justificaci\'on de la correctitud de la soluci\'on.\\

\noindent\textbf{Test$\#$1}\\
\textbf{Caracterización:} Varios cursos que no coinciden en ningún dia.\\
\textbf{Input:} \textfff{4 1 2 3 4 5 6 7 8}\\
\textbf{Output:} \textfff{1 2 3 4}\\
\textbf{Status:} OK. Los 4 cursos se pueden tomar.\\

\noindent\textbf{Test$\#$2}\\
\textbf{Caracterización:} Varios cursos, todos incompatibles entre si.\\
\textbf{Input:} \textfff{3 1 5 2 6 3 7}\\
\textbf{Output:} \textfff{1}\\
\textbf{Status:} OK. Si bien elegir cualquiera de los cursos es solución, nuestro algorimo elije el primero analizado.\\

\noindent\textbf{Test$\#$3}\\
\textbf{Caracterización:} Prioriza obtener la mayor cantidad de cursos, por sobre la mayor cantidad de horas cursadas.\\
\textbf{Input:} \textfff{4 9 15 3 7 21 27 1 31}\\
\textbf{Output:} \textfff{2 1 3}\\
\textbf{Status:} OK. Se descarta el curso que dura desde el 1ro hasta el 31 del mes, para que se puedan dictar los 3 cursos que no se solapan entre si.\\

\noindent\textbf{Test$\#$4}\\
\textbf{Caracterización:} Cursos encadenados.\\
\textbf{Input:} \textfff{11 1 2 2 3 3 4 4 5 5 6 6 7 7 8 8 9 9 10 10 11 11 12}\\
\textbf{Output:} \textfff{1 3 5 7 9 11}\\
\textbf{Status:} OK. Se preserva la mayor cantidad de cursos, eliminando los necesarios para ``desconectar'' toda la serie. En estos casos, hay que eliminar un curso de por medio.\\

\newpage
\textbf{Performance}\\
\par{Para realizar los tests de performance escribimos un programa (\textfff{testGen.cpp}) que genera de manera aleatoria varias instancias de prueba para cada $n$ entre 1 y 1000000 con un salto de 10000(20 mediciones por cada n). Para poder visualizar de la mejor manera posible la curva de performance de nuestro programa, programamos nuestro generador para que produzca tests en el peor caso posible\footnote{Las mediciones de tiempos de las instancias se encuentran en codigo/ej2/tests/test.out}. Para ese ejercicio, y viendo que la complejidad de nuestro algoritmo viene dada por la operación de ordenamiento, decidimos generar los tests de modo tal que las fechas de inicializacion y finalizacion de cada curso sea aleatoria. En el gráfico agregamos la función $f = C * n log (n)$ donde C = $\frac{1}{37500000}$ } que acota por arriba a la media de las mediciones.

\begin{figure}[H]
\centering
\includegraphics{imgs/ej2.pdf}
\caption{Test de Performance: Tiempo(s) vs Cantidad de Cursos}
\end{figure}


\newpage

\section{Ejercicio 3}

\subsection{Interpretación del enunciado}
\par{En el piso de un museo se desean instalar sensores l\'aser de seguridad. Cada sensor cubre una determinada cantidad de metros cuadrados del piso. Existen dos tipos de sensores, los sensores bidireccionales de \$4.000 que emiten dos l\'aseres en direcciones opuestas y los sensores cuatridireccionales de \$6.000 que emiten cuatro l\'aseres formando un \'angulo recto entre cada par consecutivo de l\'aseres, es decir, forman una cruz. Cada sensor cubre la posici\'on (x,y) del piso sobre
la que est\'a situado adem\'as de todas las posiciones sobre las que incidan sus l\'aseres. Los l\'aseres bidireccionales pueden ser orientados horizontal o verticalmente. La emisi\'on del l\'aser solo se detiene al alcanzar una pared. Adem\'as de las paredes que delimitan el piso, existen otras paredes dentro del mismo. Los pisos son rectangulares con un ancho y alto determinados. En cada posici\'on (x,y) del piso ($x<ancho$ y $y<alto$) puede haber una pared, un lugar vac\'io o un sensor. No puede haber m\'as de un sensor en la misma posci\'on. Tampoco se puede ubicar un sensor de forma que sea alcanzado por un l\'aser de otro sensor.}
\medskip
\par{Se quiere ubicar alguna cantidad de sensores de forma que toda posici\'on del piso sea vigilada por al menos un sensor. Una posici\'on se considera vigilada si es alcanzada por el l\'aser de alg\'un sensor. Existen adem\'as ciertas posiciones importantes, las cuales deben ser vigiladas por dos sensores distintos. Dado lo costosos que son los sensores, se pide tambi\'en encontrar la forma m\'as barata de vigilar todo el piso. El objetivo de este ejercicio es desarrollar un algoritmo que, dado un piso del museo, determine la forma menos costosa de vigilar cada posici\'on con al menos un sensor (dos las importantes) sin superponer sensores ni que estos se vigilen entre s\'i. Si bien no hay restricci\'on a la complejidad, se pide que el algoritmo desarrollado utilice la t\'ecnica de $backtracking$, y que se implementen podas para reducir el tiempo de ejecuci\'on.}

\subsection{Resolución}
\par{El piso viene representado como una matriz de enteros de dimensiones conocidas. Cada casillero de la matriz se corresponde a una posici\'on del piso y contiene un 0 si hay una pared en dicha posici\'on del piso, un 1 si es un espacio simple (o vac\'io) o un 2 si el espacio es importante. El algoritmo desarrollado realiza backtracking iterativo sobre los casilleros en los cuales se pueden ubicar los sensores, es decir los casilleros en principio vac\'ios. Dado que un casillero importante debe ser vigilado por dos sensores distintos, no tiene sentido ubicar sensores en tales posiciones, ya que tendr\'ia que ubicarse otro sensor en otra posici\'on que vigile esa casilla, pero esto provocar\'ia que un sensor vigile a otro, lo cual est\'a prohibido. Ignorar los casilleros importantes como posibles ubicaciones para los sensores es una primera poda.}
\medskip
\par{Para modelar el problema se tomar\'an los sensores bidireccionales como dos tipos de sensores distintos (aunque con el mismo precio) si se encuentran orientados horizontamente que si se encuentran orientados verticalmente. Entonces se definen tres nuevos posibles valores para cada casillero (uno por cada tipo de sensor). Las posibles soluciones del \'arbol de soluciones del algoritmo consisten en, para cada casillero en principio vac\'io, asignarle un valor correspondiente a alguna de las cuatro posibilidades: que siga vac\'io, que se ubique un sensor horizontal, que se ubique un sensor vertical o que se ubique un sensor cuatridireccional. El total de soluciones es 4^{n}, $con $n$ la cantidad de casilleros en principio vac\'ios. Esta $n$ est\'a acotada por el tama\~no de entrada (el producto entre el ancho y el alto del piso), aunque se considerar\'a la complejidad en funci\'on de dicha $n$. La justificaci\'on para tal decisi\'on es que, como recorrer todas las posibilidades tiene complejidad O(4$^n) $(y backtracking b\'asicamente hace eso) un piso muy grande cubierto de paredes salvo por unos pocos casilleros ser\'a mucho m\'as f\'acil de resolver que uno m\'as peque\~no pero con mayor proporci\'on de casilleros vac\'ios sobre paredes.$}
\par{A continuaci\'on se muestra el pseudoc\'odigo correspondiente al backtracking iterativo sobre los casilleros en principio vac\'ios.}

\begin{algorithm}[H]
	\caption{Resolución basada en Backtracking Ejercicio 3}
	\begin{algorithmic}
		\KwData{Piso $piso$}\\
		Piso $mejor$ \longleftarrow piso\\
		costo($mejor$) \longleftarrow costoMaximo + 1\\
		\While{loop}{
			\If {pisoValido(piso) y costo(piso) < costo(mejor)}{
				$mejor$ \longleftarrow piso
			}
			Casilla $c$ \longleftarrow $primeraCasillaVacia$\\
			$overflow$ \longleftarrow $cambiarValor$(c)\\
			\While{$overflow$} {
				\eIf {$ultimaCasilla$(c)} {
					$loop$ = false\\
					break\\
				} {
					$c$ \longleftarrow $casillaSiguienteVacia$(c)\\
					$overflow$ \longleftarrow $cambiarValor$(c)\\
				}
			}
		}
		\eIf {costo(mejor) > costoMaximo} {
			\textbf{return} No hay solución
		} {
			\textbf{return} $mejor$
		}
	\end{algorithmic}
\end{algorithm}

\par{El piso $mejor$ es el mejor piso encontrado hasta ahora, es decir, la distribuci\'on de sensores m\'as barata. El costo de este piso (la cantidad de dinero necesaria para implementarlo) se inicializa con $costoMaximo$ + 1, para que la primera soluci\'on v\'alida que encuentre, cualqueira sea su costo, lo sobreescriba. Si al finalizar no se encuentra ninguna soluci\'on v\'alida, el costo de la mejor soluci\'on seguir\'a siendo mayor a $costoMaximo$, lo que se utiliza para saber si se debe retornar una soluci\'on, o si no se encontr\'o ninguna (l\'ineas 15 a 18). Al empezar cada iteraci\'on del ciclo principal (l\'inea 3), $piso$ contiene una de las 4^{n} $ soluciones. Lo primero que se hace es verificar si dicha soluci\'on es v\'alida y mejor (m\'as barata) que la \'optima conseguida hasta el momento (l\'ineas 4 y 5). La funci\'on $pisoValido$ recorre todos los casilleros del piso y verifica lo siguiente.$}

\begin{itemize}
	\item Si el casillero est\'a vac\'io, verifica que al menos un sensor vigile su posici\'on.
	\item Si el casillero contiene un sensor, verifica que ning\'un otro sensor vigile su posici\'on.
	\item Si el casillero es importante, verifica que 2 sensores vigiles su posici\'on.
\end{itemize}

\par{La verificaci\'on de cada casillero consiste en recorrer, en el peor caso, toda la fila y toda la columna del casillero, dando una complejidad de O($w$+$h$) con $w$ el ancho y $h$ el alto del piso. Luego la complejidad de la funci\'on $pisoValido$ tiene una complejidad de O($w$*$h$*($w$+$h$)). El resto del ciclo (l\'ineas 6 a 14) cambian el valor de al menos un casillero, obteniendo otra soluci\'on para la siguiente iteraci\'on. La funci\'on $cambiarValor$ toma una casilla y le cambia el valor. El valor puede ser VACIO (no contiene ning\'un sensor), SENSORH (contiene un sensor horizontal), SENSROV (contiene un sensor vertical) o SENSOR4 (contiene un sensor cuatridireccional) y los recorre en ese orden. Cuando se cambia el valor de SENSOR4 a VACIO orta vez, la funci\'on devuelve $true$ para notificar que ya se han probado todas las combinaciones de la casilla, y ahora se debe cambiar el valor de otra casilla. Cuando $cambiarValor$ devuelve $true$ con la \'ultima casilla, significa que ya se han probado todas las combinaciones y el algoritmo debe terminar (l\'ineas 9 a 11). Tanto $primeraCasillaVacia$ como $casillaSiguienteVacia$ recorren la lista de las casillas que en principio est\'an vac\'ias, es decir, incluyendo a las que en ese momento contengan sensores.}
\medskip
\par{Si bien pareciera que el algoritmo funciona a fuerza bruta (probando todas las soluciones posibles), recorre el \'arbol de soluciones al igual que el m\'etodo de backtracking recursivo. Si se considera cada posible soluci\'on como el conjunto de los casilleros vac\'ios donde cada uno puede tomar uno de cuatro valores, entonces cada soluci\'on se puede representar con un n\'umero de $n$ d\'igitos en base 4. Si a su vez se representan en el \'arbol de soluciones de backtracking, se puede establecer una relaci\'on uno a uno entre cada hoja del \'arbol y cada n\'umero en base 4. De esta forma se justifica que el algorimto desarrollado recorre las mismas soluciones que el m\'etodo de backtracking recursivo.}

\subsection*{Podas}
\par{Sin podas, este algoritmo recorre las 4^{n} $ posibilidades. Una primera poda que se implement\'o, adem\'as de la poda trivial de no poner sensores en los lugares importantes, es evaluar, antes de comenzar el ciclo, si alg\'un lugar importante est\'a posicionado de forma tal que no pueda haber 2 sensores vigil\'ando su posici\'on. Como ya se mencion\'o, cada lugar importante debe ser vigilado por un sensor en su misma fila y otro en su misma columna. Entonces, si un lugar importante se encuntra de forma tal que no puede haber ning\'un sensor en su misma fila o en su misma columna, la instancia no tiene soluci\'on posible. La implementaci\'on de esta poda consiste en recorrer cada casillero importante y, de forma similar a la que se verificaba que un casillero sea valido en $pisoValiso$, recorrer su fila y su columna. Si no se encuentra un casillero vac\'io antes de toparse con una pared, en alguna direcci\'on, resulta que la instancia no tiene posible soluci\'on.$}
\medskip
\par{Otra poda implementada consiste en determinar la validez de cada casillero en el momento en que se le asigna determinado valor para reducir la complejidad de $pisoValido$. Por ejemplo, cuando $cambiarValor$ le asigna a una casilla el valor SENSORH (coloca en su posici\'on un sensor horizontal) se asegura que este sensor no est\'e vigilando a otro sensor recorriendo la fila de la casilla. Si es as\'i, intenta asignarle otro valor, en este caso SENSORV, hasta encontrar alguno v\'alido. Esto incrementa la complejidad de $cambiarValor$ ya que debe verificar que el valor que se est\'a asignando es v\'alido. Pero al asegurar que solo instancias v\'alidas (sin sensores apunt\'andose entre s\'i) llegar\'an al principio del ciclo, $pisoValido$, solo debe evaluar que los casilleros sin sensores sean vigilados por alg\'un sensor y que los casilleros importantes sean vigilados por 2 sensores.}
\medskip
\par{Una poda similar consiste en que al entrar en la funci\'on $cambiarValor$, si la casilla es vigilada por otro sensor, no vale la pena probar ubicar sensores en ella. En estos casos se retorna $true$ como notificaci\'on de que ya se han evaluado las cuatro posibilidades, aunque no sea as\'i. Finalmente se implement\'o otra poda para que, si la soluci\'on parcial que se est\'a construyendo es m\'as costosa que la soluci\'on m\'as barata encontrada hasta el momento, se ``retroceda''. Es decir, se dejen vac\'ios todos los casilleros hasta volver a alcanzar un costo temporal menor al de la soluci\'on m\'as barata hasta el momento. Es correcto realizar esta poda debido a que el costo de una soluci\'on parcial (que necesita m\'as sensores para ser una soluci\'on v\'alida) no puede reducirse cuando se alcance una soluci\'on definitiva (cuando termine de agregar los sensores necesarios) ya que agregar sensores s\'olo puede incrementar el costo final.}
\medskip
\par{Las podas implementadas reducen el espacio de soluciones de la misma forma en este algoritmo iterativo como en un backtracking recursivo. Por ejemplo, sea la soluci\'on $A$ = $a_{1}$, $a_{2}$, ... $a_{i-1}$, $a_i$, ... $a_n$, con $a_{j}$ = VACIO $\forall$ j $\in$ (i..n]. Si $a_i$ es VACIO y se va a evaluar asignarle SENSORH (ubicar un sensor horizontal en su casilla), primero se determina mediante la segunda poda enunciada, si tiene sentido hacerlo. Si no es as\'i (si un sensor horizontal en esa posici\'on vigilar\'ia a otro sensor en la misma fila) se deben podar todas las soluciones con el prefijo $A_{1..i}$ = $a_1$, $a_2$, ... $a_{i-1}$, SENSORH. La t\'ecnica de backtracking recursivo realizar\'ia esta poda al recorrer el nodo en el nivel $i$, antecesor de la hoja correspondiente a la soluci\'on A. Dicho nodo es antecesor de todas las hojas cuyas soluciones asociadas tienen el prefijo $A_{1..i}$, entonces al dejar de recorrer esa ramificaci\'on, se evitan evaluar todas esas soluciones. El algoritmo desarrollado con backtracking iterativo realiza la misma poda cuando itera sobre la soluci\'on A. Al comprobar que no tiene sentido asignarle SENSORH a $a_i$, le asigna en su lugar SENSORV, ignorando as\'i todas las soluciones con el prefijo $A_{1..i}$.}

\newpage
\subsection{Complejidad}
\par{A\'un aplicando las podas, la complejidad del algoritmo sigue siendo $O(4^{n})$ ya que en el peor caso, ninguna de las cotas reducir\'ia el conjunto de soluciones y todas ellas deber\'ian evaluarse. Sin embargo, esperamos que en la mayor\'ia de los casos se vea reducido el tiempo total de ejecuci\'on gracias a las podas (es muy com\'un que en un piso haya 2 casilleros vac\'ios adyacentes, caso en el que no se evaluari\'a poner sensores cuatridireccionaoles en ambos). La complejidad de $pisoValido$ sigue siendo O($w$*$h$*($w$+$h$)), ya que debe recorrer todos los casilleros y, para los vac\'ios y los importantes (en la primera iteraci\'on son todos menos las paredes), recorrer una fila y una columna. La funci\'on $cambiarValor$ tambi\'en debe recorrer una fila y una columna pero s\'olo para un casillero (aunque podr\'ia tener que hacerlo 4 veces), por lo que su complejidad es O(4*$w$*$h$). La complejidad final del algoritmo resulta entonces:

\begin{equation}
T(n) = O(4^n*(w*h*(w+h)+4*w*h) = O(4^n*w*h*(w+h+4)) 
\end{equation}

Si acotamos $n$ por $w*h$ y llamamos $s$ a $w*h$, nos queda:

\begin{equation}
T(s) = O(4^s*(ws+hs+4s) = O(4^s*(2s^2+4s)) = O(4^s*s^2)
\end{equation}

Con s la cantidad de casilleros del piso.

\subsection{Implementaci\'on}
\par{Cada piso se representa con una matriz de enteros, donde cada entero representa el contenido de la posici\'on en el piso. Se guardan dos pisos, el $pisoMejor$ (almacena el mejor encontrado hasta el momento) y el $pisoActual$ (el cual se modifica en cada iteraci\'on). La funci\'on $cambiarValor$ toma en un arreglo los casilleros vac\'ios y suma 1 al n\'umero en base 4 que representa a la soluci\'on de $pisoActual$. Ese cambio se ve reflejado en el piso. La funci\'on del pseudoc\'odigo $pisoValido$ se reemplaz\'o en el c\'odigo por $evaluarPiso$ la cual recorre el $pisoActual$ y, si es mejor que el $pisoMejor$, pasa a ser el nuevo $pisoMejor$. Gracias a las podas se sabe que el $pisoActual$ que recorre $evaluarPiso$ es v\'alido en el sentido que no hay sensores apunt\'andose entre s\'i, por lo que s\'olo se asegura que todos los casilleros est\'en siendo vigilados por al menos un sensor y que los casilleros importantes est\'en siendo vigilados por 2 sensores.}
\medskip
\par{La primera poda se ejecuta antes de comenzar el ciclo principal. Para cada casillero importante, se recorren su fila y su columna en busca de alguna posici\'on vac\'ia. Si para alguno no se encuentran, la funci\'on principal retorna sin entrar al ciclo. Las otras podas se implementan en $cambiarValor$ ya que act\'uan al momento de asignar valores a los casilleros. Para determinar la validez de los sensores en los casilleros se implementaron las funciones $vigila(Posicion$ $p)$ que devuelve si un sensor en la posici\'on $p$ est\'a vigilando a otro sensor, $vigilada(Posicion$ $p)$ que devuelve si alg\'un sensor est\'a vigilando la posici\'on $p$ y $doble\_vigilada(Posicion$ $p)$ que devuelve si 2 sensores est\'an vigilando la posici\'on $p$ (utilizada por $evaluarPiso$ para determinar si todas las posiciones importantes est\'an siendo vigiladas por 2 sensores). Estas tres funciones recorren la fila y la columna de la posici\'on $p$ en busca de sensores.}

\newpage
\subsection{Demostraci\'on de Correctitud}
\par{
Para demostrar que es correcto el algoritmo, verificaremos que se cumplen las siguientes propiedades:

\begin{itemize}

	\item El algoritmo sin podas, se fija exhaustivamente (con fuerza bruta) todas las posiblidades, es decir
en las n casillas libres cubriendo todas las posibilidades: que siga vacio, o haya cualquier tipo de sensor. Luego, si hay solución la encuetra y si no, devuelve que no hay solución.
 
	\item Todas las podas son válidas, es decir se saltean casos que no llegan a la solución, o en caso de ser una posible solución 
(que todos los casilleros esten vigilados) seguro no es optima (no es la solución más barata).

\end{itemize}

\Par{Se descarta la posibilidad de instalar sensores en los lugares importantes, ya que si hay un sensor en un casillero importante necesita que lo apunten dos sensores y estos sensores estarían apuntando al sensor instalado, cosa que prohibe el enunciado.
Luego solo es posible agregar sensores en los casilleros vacíos.}

\Par{Podemos pensar a la solución como un número con dígitos $a_0 a_1 ... a_i ... a_n$, en los cuales $(\forall 1 \le i\le n) \quad 0 \le a_i \le 3$.
Donde 0 representa a VACIO , 1 a SENSORH, 2 SENSORV y 3 a SENSOR4.
Cada dígito indica que hay en el casillero i que (en el inicio) estaba vacío.

Veamos que el algoritmo lo que esta haciendo es asignar de izquierda a derecha los números siempre en orden creciente (desde el 0 hasta el 3).
Luego, si no hay ninguna poda, verifica todos los candidatos a solución ($4^n$ candidatos).

Demostración por Inducción (con el algortitmo sin podas):


Hipotesis inductiva: el algoritmo recorre todas las posibles soluciones con n casilleros vacios.
Luego, queremos ver que si tenemos un museo con n+1 casilleros vacios, el algoritmo recorre todas las posibles soluciones.

Caso Base:
El algoritmo cambia el valor de la última y única casilla con las 4 posiblidades (vacío o los tres tipos de sensores). Es decir, tras probar las 4 posibilidades,
 cambiar$\_$valor asigna a la variable $overflow = true $ y entra al ciclo (de $while (overflow)$) y como es la última casilla, asigna loop = false e interrumpe el ciclo. 

Paso inductivo:
\begin{equation}	
	$el algoritmo recorre todas las$  $4^n$ posibles soluciones con n casilleros vacios  $\Rightarrow$ \\
	el algoritmo recorre todas las posibles soluciones con n+1 casilleros vacios. 
\end{equation}

Sea el piso que tiene n+1 casilleros vacíos. Sea una posible solucion el número $A$ = $a_{1}$, $a_{2}$, ... , $a_i$, ... $a_n$ $a_{n+1}$, con $a_i$ un dígito que indica
los 4 posibles estados del casillero. Por la hipótesis inductiva, sabemos que el algoritmo simula los $4^n$ prefijos. Para cada uno de los $4^n$ prefijos el algoritmo 
prueba las 4 alternativas para la última casilla. Por lo tanto prueba las $4^{n+1}$ posibles soluciones.

Esto es porque donde hubiese terminado el ciclo con n casillas, con n+1, $ultimaCasilla$ devuelve false, y se obtiene una casilla más (siendo esta la última de las n+1)
a la que se le cambia una vez el valor y se vuelven a evaluar las $4^n$ posibles soluciones del prefijo que se tenía.
Esto se repite para las 4 opciones que puede tomar la última casilla. 


}

\medskip
\\Ahora demostraremos que las podas son válidas:

\begin{itemize}
\item \par{Alguna más barata : si ya hay una solución con un costo menor al costoMaximo, y se tiene unsa solución parcial con un mayor costo, esta
solución se descarta al igual que toda solución con ese prefijo. 


Demostración: Si $costo(soluciónOptimaDelMomento) \le costoActual(soluciónParcial)$, la solución que tenga como prefijo a la solución parcial, va a tener más o
igual cantidad de sensores, logrando que cualquiera solución con ese prefijo va a ser más cara que la solucionOptimaDelMomento. }

\medskip

\item \par{Si la casilla i está siendo vigilada, ningún número con un sensor(de cualquier tipo) en esa casilla puede ser solución:


Demostración: Cualquier solución que tenga a una casilla i que este siendo vigilada y se le agrega un sensor, este sensor va a tener una señal que impacte al dispositivo agregado, escenario que no puede ser posible por la prohibición del enunciado.
}
\medskip
\item \par{Si una casilla importante se encuntra de forma tal que no puede haber ningún sensor en su misma fila o en su misma
columna, la instancia no tiene solución posible:


Demostración: Sea la casilla i, una casilla importante donde los casilleros que tiene en su misma fila y columnas(hasta llegar a una pared)
son PARED o IMPORTANTE. En este escenario no es posible colocar sensores en las paredes ya que solo se los puede colocar en casilleros libres.
Tampoco se puede colocar en un casillero IMPORTANTE debido a que se apuntarían mutuamente.
}


}

\newpage
\subsection{Testing}
\textbf{Correctitud}\\

Para la realización de los tests de correctitud no haremos muchas variaciones en el tamaño del piso del museo. Consideramos que este aspecto no es muy relevante a la hora de buscar casos bordes.\\

\noindent\textbf{Test$\#$1:} Piso del museo compuesto enteramente por casilleros libres.
\begin{figure}[H]
\centering
\def\svgwidth{140 pt}
\input{imgs/ej3_0.pdf_tex}
\end{figure}
\noindent\textbf{Status:} OK. Se cubren todos los casilleros.\\

\noindent\textbf{Test$\#$2:} Piso del museo compuesto enteramente por paredes.
\begin{figure}[H]
\centering
\def\svgwidth{140 pt}
\input{imgs/ej3_1.pdf_tex}
\end{figure}
\noindent\textbf{Status:} OK. No es necesario ni posible colocar ningun sensor.\\

\noindent\textbf{Test$\#$3:} Todos los casilleros son importantes y necesitan ser cubiertos con 2 sensores.
\begin{figure}[H]
\centering
\def\svgwidth{140 pt}
\input{imgs/ej3_2.pdf_tex}
\end{figure}
\noindent\textbf{Status:} OK. No existe solución\\

\noindent\textbf{Test$\#$4:} Piso cuadriculado.
\begin{figure}[H]
\centering
\def\svgwidth{140 pt}
\input{imgs/ej3_3.pdf_tex}
\end{figure}
\noindent\textbf{Status:} OK. No hay otra posibilidad más que colocar un sensor, vertical u horizontal en cada espacio libre.\\

\newpage
\noindent\textbf{Test$\#$5:} Casilleros importantes en esquinas.
\begin{figure}[H]
\centering
\def\svgwidth{140 pt}
\input{imgs/ej3_4.pdf_tex}
\end{figure}
\noindent\textbf{Status:} OK.\\

\noindent\textbf{Test$\#$6:} Casilleros importantes en el centro.\\
\begin{figure}[H]
\centering
\def\svgwidth{140 pt}
\input{imgs/ej3_5.pdf_tex}
\end{figure}
\noindent\textbf{Status:} OK.\\

\noindent\textbf{Test$\#$7:} Tablero de una sola fila.\\
\begin{figure}[H]
\centering
\def\svgwidth{140 pt}
\input{imgs/ej3_6.pdf_tex}
\end{figure}
\noindent\textbf{Status:} OK.\\

\noindent\textbf{Test$\#$8:} Tablero de una sola columna.\\
\begin{figure}[H]
\centering
\def\svgwidth{140 pt}
\input{imgs/ej3_7.pdf_tex}
\end{figure}
\noindent\textbf{Status:} OK.\\


\newpage
\textbf{Performance}\\
\par{Para este ejercicio, los casos evaluados se realizaron en un conjunto m\'as reducido de instancias, dada la elevada complejidad de la soluci\'on propuesta. Otra particularidad del problema es que la complejidad del algoritmo no crece en funci\'on del tama\~no de entrada (la cantidad de casilleros del piso), aunque s\'i en funci\'on de algo que puede ser acotado por \'el (la cantidad de casilleros en principio vac\'ios). Por ello se decidi\'o generar dos conjuntos de instancias. Primero se generaron instancias en las que cada una se corresponde con una cantidad de casilleros en principio vac\'ios. Luego se generaron instancias en las que cada una se corresponde con el tama\~no del piso. Para el primer conjunto, dada una cantidad $n$ de casilleros vac\'ios se genera un piso de $h$ filas y $w$ columnas con $h = \sqrt{n}$ + 1} y $w = n / \sqrt{n} +1$. Luego se determinaron al azar paredes y posiciones importantes para dejar exactamente $n$ casilleros vac\'ios. Para el segundo conjunto, dado un tama\~no $n$ se genera un piso de $h$ filas y $w$ columnas tal que $h*w = n$. Una vez generados los conjuntos, se ejecutaron ambos y se graficaron los resultados a continuaci\'on.}
\par{El siguiente gr\'afico muestra los tiempos de ejecuci\'on de las instancias generadas para el primer conjunto. Lo que nos importa ver es c\'omo se comporta el algoritmo en funci\'on de la cantidad de casilleros vac\'ios por lo que se grafic\'o en funci\'on de tal. Las series verdes corresponden a las mediciones del algoritmo sin podas, que recorre todas las posibles soluciones, mientras que las series rojas corresponden a la versi\'on del algoritmo que implementa podas. Para cada $n$ cantidad de casilleros vac\'ios, se ejecutaron 50 instancias distintas. Los cuadrados rojos y los c\'irculos verdes son los promedios de las 50 ejecuciones para el algoritmo con y sin podas respectivamente. Cada uno de estos viene acompa\~nado con intervalos verticales que denotan la varianza de cada medici\'on. Tambi\'en se graficaron con cruces rojas y verdes las mediciones m\'aximas y m\'inimas para cada $n$. La curva azul es una funci\'on del tipo $4^n$. Notar que la escala del eje vertical no es lineal sino logar\'itmica.}
\begin{figure}[H]
\centering
\def\svgwidth{140 pt}
\includegraphics{../codigo/ej3/tests/ej3a.pdf}
\caption{Resultados de las mediciones obtenidas por el script $ej3\_test$ con los par\'ametros 35 1 50 para instancias generadas en funci\'on de la cantidad de casilleros vac\'ios.}
\end{figure}
\par{Lo primero que se puede apreciar en este gr\'afico, es que la complejidad del algoritmo sin podas se ajusta bastante bien a la cota $4^n$ con $n$ la cantidad de casilleros. Si bien esa funci\'on tambi\'en acota los tiempos de ejecuci\'on del algoritmo con podas, estos son considerablemente menores y la diferencia se agranda a medida que aumenta el $n$. La complejidad del algoritmo sin podas parece estabilizarse a medida que se incrementa el tama\~no de entrada (recordar que el tama\~no de entrada fue generado para ser directamente proporcional a la cantidad de casilleros vac\'ios). Esto se deduce al ver como su varianza tiende a achicarse. El algoritmo con podas, por otro lado, parece tener intervalos en los que su complejidad no aumenta demasiado pero luego tiene saltos abruptos. Estos saltos se han identificado en las posiciones correspondientes a valores de $n = k^2$, para $k$ entero (En principio se hab\'ia ejecutado tambi\'en para $n = 36$ pero debido a su excesivo tiempo de procesamiento, se detuvo la ejecuci\'on). Estos valores tienen la particularidad de generar matrices cuadradas (de $\sqrt{n}+1$ x $\sqrt{n}+1$ casilleros). Es tambi\'en en estos saltos en los que se aprecia una mayor varianza, la cual parece disminu\'ir hasta el siguiente salto. Para la mayor\'ia de los valores de $n$, el valor m\'inimo no se ve en el gr\'afico debido a que su tiempo de ejecuci\'on fue tan bajo que se redonde\'o a $0$. Probablemente sean casos en los que, por la primera poda (evaluar si existe una posici\'on importante no vigilable) no se ejecut\'o ninguna iteraci\'on del ciclo principal.}
\par{El siguiente gr\'afico muestra los tiempos de ejecuci\'on de las instancias generadas para el segundo conjunto, es decir en funci\'on del tama\~no de entrada. En este caso, cada una de las $n$ posiciones se determin\'o al azar, aunque favoreciendo la designaci\'on de casilleros vac\'ios para que sean estos los que predominen y no las paredes. Una vez m\'as, las series verdes corresponden a las mediciones del algoritmo sin podas y las rojas al algoritmo con podas. La curva azul es exactamente la misma que la graficada en la figura anterior. Servir\'a para comparar la complejidad del algoritmo en funci\'on del tama\~no de entrada con la complejidad del algoritmo en funci\'on de la cantidad de casilleros. La varianza, los m\'aximos y los m\'inimos fueron graficados de la misma forma que la figura anterior. Tambi\'en con el objetivo de comparar ambas opciones, se mantuvieron las escalas en los ejes.}
\begin{figure}[H]
\centering
\def\svgwidth{140 pt}
\includegraphics{../codigo/ej3/tests/ej3b.pdf}
\caption{Resultados de las mediciones obtenidas por el script $ej3\_test$ con los par\'ametros 35 1 50 para instancias generadas en funci\'on de la cantidad de casilleros totales.}
\end{figure}
\par{En principio se observa que ambas series est\'an acotadas por la funci\'on que se ajustaba a la complejidad del algoritmo sin podas en funci\'on de la cantidad de casilleros vac\'ios. Incluso el algoritmo sin podas parece crecer con menor complejidad, aunque podr\'ia deberse a que el tama\~no de sus instancias incrementa en 1 para cada valor, mientras que en el caso anterior, incrementar en 1 el valor de $n$ pod\'ia incrementar en m\'as de 1 el tama\'no del piso. Es notable la gran varianza de ambos algoritmos, as\'i como el hecho que que tampoco se graficaron los valores m\'inimos. Todo esto puede ser atribu\'ido a que, dado que no se fijaron la cantidad de casilleros vac\'ios, estos podr\'ian haber sido muy pocos, haciendo muy peque\~no el conjunto de soluciones posibles.}

\newpage

\section{Conclusiones}

La realización de este trabajo práctico nos permitió familiarizarnos con el enfoque teórico-experimental que la cátedra espera dentro de los trabajos prácticos. Pudimos recorrer el proceso del análisis formal de un problema, desde su observación inicial hasta las conclusiones prácticas. 

Por el lado teórico y formal, nos planteó las dudas sobre que es necesario demostrar para justificar nuestras soluciones propuestas. Por el lado práctico, nos ayudó a complementar las nociones de algoritmos greedy y backtracking que ya habíamos visto en las clases teóricas de la materia.

También nos acercó ciertas nociones preliminares sobre cómo mostrar resultados de la manera más concreta posible y evitar los errores más comunes al hacerlo.

Para finalizar, sentimos que este trabajo nos representó un buen ejercicio en el análisis algorítmico general y nos dejó bien orientados para afrontar los siguientes proyectos dentro de la materia.

\end{document}
